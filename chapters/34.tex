\documentclass[output=paper]{langsci/langscibook}
\ChapterDOI{10.5281/zenodo.4280647}
\author{Ángel J. Gallego\affiliation{Universitat Autònoma de Barcelona}}
\title{Strong and weak \enquote{strict cyclicity} in phase theory}

% \chapterDOI{} %will be filled in at production

\abstract{This paper explores the possibility that the \gls{NTC} is eliminated
    in favor of a strong version of the \gls{PIC}.  This possibility is welcome
    on theoretical grounds, given the redundant nature of the \gls{NTC} and the
    \gls{PIC}. I review empirical evidence indicating that the (original
    formulation of the) \gls{NTC} is violated phase\is{phases}-internally, a
    possibility that does not extend to the \gls{PIC}.  In so doing, I also
    consider the weak version of the \gls{PIC} discussed in \citet{Chomsky2016}.}

\maketitle

\begin{document}\glsresetall

\section{Efficient computation}

Generative Grammar has endorsed various economy principles (from
\citeauthor{Chomsky1955}’s \citeyear{Chomsky1955} \emph{traffic convention}
to \citeapos{Chomsky1995} \emph{minimal link condition}, going through many
others).  All such proposals adhere to a “least effort” desideratum attributed
to the syntactic computation of the faculty of language.  Within the Minimalist
program (\glsunset{MP}\gls{MP}), the basic structure-building operation is
Merge -- the only one that “comes free,” without justification
\parencites[3]{Chomsky2001}[137]{Chomsky2008}.

\largerpage
Assuming it operates without bounds, \isi{Merge} takes two objects, α and β, to
construct a new object, γ. Additional applications of \isi{Merge} target γ, which is
the only object left in the derivation \citep[243]{Chomsky1995}, to yield γ$'$,
and then γ$''$, and so on and so forth -- again, without bounds:\footnote{In
    \textcites[11]{Chomsky2007}[139]{Chomsky2008} it is assumed that the free
    nature of \isi{Merge} follows from LIs having an \textsc{\gls{EF}} that is
    undeletable and can thus give rise to an unbounded application of
    \isi{Merge}.  I will not assume \glspl{EF}. Apart from the empirical
    advantage of dispensing with \glspl{EF} (they have no realization in any
    language, so they are a purely theory-internal device), this allows us to
    dispense with the technical problems discussed in \citet{Narita2014},
    related to the lack of \gls{EF} percolation.}

\ea%1
    \label{ex:34.1}
    \ea Merge(α, β) = \{α, β\}
    \ex Merge(λ, γ) = \{λ, γ\}
    \ex Merge(ψ, γ$'$) = \{ψ, γ$'$\}
    \z
\z

That α and β are no longer available was expressed in the following passage:

\begin{quote}
Applied to two objects α and β, \isi{Merge} forms the new object K,
\emph{eliminating α and β}. (\citealt[243]{Chomsky1995}, my emphasis)\end{quote}

A Merge-based system is enough to capture the property of \textsc{cyclicity},
that is, “in essence, the intuition that the properties of larger linguistic
units depend on the properties of their parts”
\citep[1]{Chomsky2012}.\footnote{As an anonymous reviewer observes, this is not
the case if \isi{Merge} allows, e.g., countercyclic infixing of SPEC-T after C has
already been merged (see \citealt{Chomsky2008}), or Parallel, Sidewards, Late,
etc. \isi{Merge}. Cf. \citet{ChoGalOtt2019} and references therein for discussion.}
It is easy to see that a cyclic system will be largely compositional
\parencites[5]{Chomsky2007}[2]{Chomsky2012}: if computation is meaningful in an
efficient manner, the interpretation of a given linguistic object will not be
changed later on, which corresponds with “the general property of
\textsc{strict cyclicity}” \citep[5]{Chomsky2007}.\footnote{Of course, the
    interpretation of “Mary” is different in \emph{Someone called Mary} and
    \emph{Mary called someone}. That the interpretation of a given
    \glsunset{SO}\gls{SO} cannot
    be changed should thus be restricted to a post-Merge scenario, a
    possibility that is not entertained in feature-based approaches to
theta-roles.} Therefore, whereas cyclicity follows from \isi{Merge} alone,
strict cyclicity requires something else -- the mere existence of such an
operation does not in and of itself guarantee the conservation of the already
assembled structure.  This is the natural scenario where \gls{MP} invokes
so-called third factor conditions, which fall into two broad categories
\citep{Chomsky2005}:

\ea%2
    \label{ex:34.2}Third-factor conditions\\
    \ea Principles of data analysis that might be used in language acquisition\is{language acquisition}
    and other domains;
    \ex Principles of structural architecture and developmental constraints
    that enter into canalization, organic form, and action over a wide range,
    including \emph{principles of efficient computation}, which would be
    expected to be of particular significance for computational systems such as
    language. It is the second of these subcategories that should be of
    particular significance in determining the nature of attainable languages.
    (\citealt[6]{Chomsky2005}, my emphasis)
    \z
\z

Different conditions have been put forward in order to capture the idea that
linguistic objects generated by the syntactic computation cannot be changed
(where \emph{change} covers a wide range of possibilities: deletion,
feature-valuation, late-insertion, tucking-in, etc.), especially by adding ad
hoc symbols or performing operations that depart from least effort metrics.
This is precisely the role played by the \glsunset{IC}\textsc{\glsdesc{IC}}
(\gls{IC}, \citealt[228]{Chomsky1995}), the
\glsunset{NTC}\textsc{\glsdesc{NTC}} (\gls{NTC}, \citealt[138]{Chomsky2008}),
and the \glsunset{PIC}\textsc{\glsdesc{PIC}} (\gls{PIC},
\citealt{Chomsky2000}). Putting details aside, \gls{IC}, \gls{NTC}\is{no
tampering condition} and \gls{PIC} all play a similar role in the current
model, which was already noted by Juan Uriagereka in his annotated version of
\citet{Chomsky2001}:

\blockquote{So the Extension Condition [still holds]. This is somewhat
    surprising, given the [adoption of] “tucking-in” in \citet{Chomsky2000}. In
    effect, \emph{we have several things ensuring the cycle}. The
    \glsunset{EC}\gls{EC}, in a radical way for the upward boundary of the
    phrase marker; the \gls{PIC} for a kind of downward boundary, beyond which
    the system doesn’t see any further operations; the idea of
    interpretation/evaluation at the strong phase\is{phases} in addition to both of these,
as the derivation unfolds; and, finally, the phase-like access to the
Numeration.  \emph{Much room for improvement and unification} \dots{}\\
\hspace*{\fill}(\citealt{Uriagereka1999b}, my emphasis)}

Such a redundant scenario is not expected, if only at a purely methodological
level. This note argues that (the strong version of) the \gls{NTC}\is{no tampering condition} can be
subsumed under the \gls{PIC}, given that local (phase-internal) modification is
possible.\footnote{Probably, the same can be said of the \gls{IC}, by simply
observing that labels, indices, traces, and similar devices are not part of any
I-language.} Discussion is divided as follows:~\Cref{sec:34.2} reviews the different
conceptions of the \gls{NTC}\is{no tampering condition} that have been entertained within \gls{MP} and the
empirical problems that have been observed for it;~\Cref{sec:34.3} turns its
attention to the \gls{PIC}, focusing on the recent possibility that the
complement of a phase\is{phases} does not leave the computation
\parencite{Chomsky2008,Chomsky2016}; in~\Cref{sec:34.4}, I argue that (the strong)
\gls{NTC} can be eliminated adopting a strong version of the \gls{PIC}, whereby
transferred computation is forgotten (literally expunged), yielding a straight
version of strict cyclicity;~\Cref{sec:34.5} summarizes the main conclusions.

\section{Merge and the NTC}\label{sec:34.2}

There is a very close relationship between \isi{Merge} and the \gls{NTC}\is{no
tampering condition} on the one hand, and between Transfer and the \gls{PIC} on
the other (as we will see in more detail in \Cref{sec:34.3}). In fact, I would
like to underscore the fact that, whereas Transfer and the \gls{PIC} (as well
as the operations of \gls{FI} and \isi{Agree})\footnote{I assume that
    \isi{Agree} actually implies a complex set of operations: Feature
    Inheritance, Match, Valuation and Deletion. Deletion is meant to cover
    erasure of uninterpretable φ-features, but it can also be applied to heads,
    as in \citeapos{Chomsky2015} analysis of \emph{that}-deletion.
    Cf.~\citet{EKS2016} alternative in terms of phase-cancellation.
Cf.~\citet{Gallego2014} for an alternative approach to \gls{FI}, with
interesting consequences for \citeapos{Chomsky2015}  analysis of the
\glsunset{EPP}\gls{EPP},\is{extended projection principle} discussed in
\citet{Gallego2017}.} apply at the phase level, \isi{Merge} and the
\gls{NTC}\is{no tampering condition} do not invariably so
\parencites[17]{Chomsky2007}[143]{Chomsky2008}[40, 42]{Chomsky2013}. I state
this correlation as follows, which I would like to build on to argue that there
is a deep connection between the phase-based architecture and the (mildly)
context-sensitive nature of the Faculty of Language
(cf.~\citealt{Chomsky1956,Uriagereka2008}):\footnote{It is typically assumed
    that all operations but \glsunset{EM}\gls{EM} apply at the phase\is{phases}
    level, simultaneously
    \parencites[116]{Chomsky2004}[19]{Chomsky2005}[17]{Chomsky2007}[155]{Chomsky2008}.
This raises questions for derivational systems, where the application of rules
is ordered, as in \citet{Chomsky2015}.}\is{NTC|see{no tampering condition}}

\ea%3
    \label{ex:34.3}
    \ea \gls{EM} = context-free
    \ex \glsunset{IM}\gls{IM}\slash \isi{Agree}\slash Transfer = (mildly) context-sensitive
    \z
\z

In what follows I would like to briefly review the different formulations of
the \gls{NTC}. As the reader will see, the conclusion will be that there are
various situations where a weak version of the \gls{NTC}\is{no tampering
condition} must be assumed, not only for operations like \gls{FI} or
\isi{Agree} (\citealt[19, fn.\ 26]{Chomsky2007}),\footnote{\gls{FI} is
    reinterpreted as copying in \citet[47]{Chomsky2013}. This also departs from
    the strong \gls{NTC}\is{no tampering condition} (unless we adopt the
formulation in \citealt{Gallego2014}).\label{fn:34.8}} but also for Merge.

In \citet{Chomsky2000,Chomsky2001,Chomsky2004,Chomsky2005}, no explicit mention
to the \gls{NTC}\is{no tampering condition} is made. Instead, the \textsc{\glsfirst{EC}} is responsible
for capturing the idea that \isi{Merge} always applies to the edge of an \gls{SO}.
Thus, \gls{EC} makes sure that, given \{α, β\}, a new element δ can only be
merged as in \REF{ex:34.4a}, not \REF{ex:34.4b}, which would be counter-cyclic.

\ea%4
    \label{ex:34.4}
	\ea \{δ, \{α, β\}\} \label{ex:34.4a}
	\ex \{\{α, δ\}, β\}\} \label{ex:34.4b}
	\z
\z

\citet[136]{Chomsky2000} discusses these options, noting that \REF{ex:34.4a} satisfies
the \gls{EC} whereas \REF{ex:34.4b} satisfies Local \isi{Merge}. In the same breath, he
notes that \blockquote[{\citealt[136]{Chomsky2000}}][.]{weaker assumptions
    suffice to bar [(\ref{ex:34.4}a)] but still allow Local \isi{Merge}
    under other conditions.  Suppose that operations do not tamper with the
    basic relations involving the label that projects: the relations provided
    by \isi{Merge} and composition, the relevant ones here being sisterhood and
    c-command} \citet[137]{Chomsky2000} goes on to argue that
    \enquote{derivations then observe the condition [\eqref{ex:34.5}], a
        kind of economy condition, where R is a relevant basic
    relation}.\footnote{This is what \citet[Ch.\ 2]{LasUri2005} and
\citet[256]{EKS2012} refer to as Law of Conservation of Relations.}

\ea%5
    \label{ex:34.5} Given a choice of operations applying to α and
    projecting its label L, select one that preserves R(L, γ)
\z

\eqref{ex:34.5} holds in general, except for head \isi{adjunction}. In the case of XP merger,
\citet{Chomsky2000} observes that \gls{EC} must be satisfied for second-Merge, but
not for subsequent applications or Merge -- the creation of specifiers, which
amounts to accepting tucking-in \citep{Richards1997}.

In \citet{Chomsky2004}, it is explicitly noted that the \gls{EC} can come in a
strong and a weak version, the latter accepting deviations from
\eqref{ex:34.5}:

\blockquote[{\citealt[109]{Chomsky2004}, my emphasis}][.]{%
Cyclicity of derivation requires that \isi{Merge} to α always be at the edge
of α, satisfying \emph{an extension condition, strong or weak}
(“tucking in”) [\dots{}] There appears to be one significant counterexample to
cyclic \isi{Merge}: late insertion of adjuncts [\dots{}] Elementary considerations of
efficient computation require that \isi{Merge} of α to β involves
minimal search of β to determine where α is introduced, as
well as least tampering with β: search therefore satisfies [Local
Merge], and \isi{Merge} satisfies an \gls{EC}, with zero search. One possibility is
that β is completely unchanged (the \emph{strong \gls{EC}}); another natural
possibility is that α is as close as possible to the head that is the label of
β, so that any Spec of β now becomes a higher Spec (“tucking in,” in Norvin
Richards’s sense). Further questions arise under \isi{Merge} with multiple Specs.
Assume some version of the \gls{EC} to hold, in accord with
\glsunset{SMT}\gls{SMT}\is{strong Minimalist thesis}}

The \gls{NTC}\is{no tampering condition} is first introduced in \citet{Chomsky2005}, when discussing
conditions of efficient computation. What I would like to capitalize on from
the following quote is how similar \gls{NTC}\is{no tampering condition} and \gls{PIC} are, in the sense
that the former appears to be related to the fact that what has been
constructed in the course of a derivation \emph{can be forgotten}; this is
relevant, since this is typically the hallmark of the \gls{PIC}.

\blockquote[{\citealt[11, 13]{Chomsky2005}, my emphasis}][.]{One natural
    property of efficient computation, with a claim to extralinguistic
    generality, is that operations forming complex expressions should consist
    of no more than a rearrangement of the objects to which they apply, not
    modifying them internally by deletion or insertion of new elements. If
    tenable, that sharply reduces computational load: \emph{what has once
    been constructed can be ``forgotten'' in later computations, in that it
will no longer be changed}.  That is one of the basic intuitions behind the
notion of cyclic computation.  The \glsunset{EST}\gls{EST}/Y-model and other
approaches violate this condition extensively, resorting to bar levels, traces,
indices, and other devices, which both modify given objects and add new
elements. A second question, then, is whether all of this technology is
eliminable, and the empirical facts susceptible to principled explanation in
accord with the ``no-tampering'' condition of efficient computation [\dots]
Assuming the \gls{NTC}\is{no tampering condition} that minimizes computational load, both kinds of \isi{Merge}
to A will leave A intact. That entails merging to the edge, the \gls{EC}, which
can be understood in different ways, including the ``tucking-in'' theory of
\textcite{Richards1997}, which is natural within the probe-goal framework of
recent work, and which can also be interpreted to accommodate head
adjunction}

Notice that what this says is that the \gls{NTC}\is{no tampering condition} is a third-factor condition on the
way \isi{Merge} operates.\footnote{This formulation states that the \gls{NTC}\is{no tampering condition} is
Merge-sensitive alone, which opens the door for conditions being sensitive to
independent operations.}  More precisely, the \gls{NTC}\is{no tampering condition} guarantees that when \isi{Merge}
applies to α and β, we obtain a new \gls{SO}, γ, which can
then be merged with further objects. So, for instance, if γ is merged
with δ, given that α and β themselves are gone from
the computation, the only way for this to happen is by forming \{γ, δ\}. This
way, \isi{Merge} must be to the edge as it cannot tamper with the objects it
applies to~-- in the case at hand, \isi{Merge} cannot break up γ or tamper with
it.

What is relevant about \citet{Chomsky2008} is the discussion of certain
situations that threaten the strong \gls{NTC}: \glsunset{FI}\gls{FI} and the
analysis of subject \isi{raising} to SPEC-T.

\blockquote[{\citealt[138, 144]{Chomsky2008}, my emphasis}][.]{A natural
    requirement for efficient computation is a \enquote{no-tampering condition}
    (\gls{NTC}): \isi{Merge} of X and Y leaves the two \glspl{SO} unchanged. If so, then
    \isi{Merge} of X and Y can be taken to yield the set \{X,Y\}, the simplest
    possibility worth considering. \isi{Merge} cannot break up X or Y, or add new
    features to them.  Therefore \isi{Merge} is invariably ``to the edge'' and we
    also try to establish the [IC] dispensing with bar levels, traces, indices,
    and similar descriptive technology introduced in the course of derivation
    of an expression [\dots{}] Note that \gls{SMT}\is{strong Minimalist thesis} might be satisfied even
    where \gls{NTC}\is{no tampering condition} is violated -- if  the violation has a principled explanation
    in terms of interface conditions (or perhaps some other factor, not
    considered here). The logic is the same as in the case of the phonological
    component, already mentioned [\dots{}] \emph{The device of inheritance}
    [\dots{}] \emph{is a narrow violation of \gls{NTC}}. The usual question
    therefore arises: does it violate \gls{SMT}?\is{strong Minimalist thesis} If it does, then the device
    belongs to \glsunset{UG}\gls{UG} (perhaps parametrized), lacking a principled
    explanation.  But the crucial role it plays at the C-I interface suggests
    the usual direction to determine whether it is consistent with \gls{SMT}\is{strong Minimalist thesis}
    though violating \gls{NTC}. If the C-I interface requires this distinction,
    then \gls{SMT} will be satisfied by an optimal device to establish it that
violates \gls{NTC}, and inheritance of features of C by the LI selected by C (namely
T) may meet that condition. If so, the violation of \gls{NTC}\is{no tampering condition} still satisfies
\gls{SMT}\is{strong Minimalist thesis}}

\citet{Chomsky2007,Chomsky2008} assumes that φ-features are generated in phase\is{phases}
heads, from which they are downloaded (downward percolation) to non-phase
heads. Following \citet{Richards2007}, the process is taken to be mandatory
under the \gls{PIC}: Since these features must be deleted, they must end up in
the Transfer domain.\footnote{As pointed out in~\cref{fn:34.8},
    \citet{Chomsky2013} suggests that \gls{FI} is actually a form of copying.
    If correct, \gls{FI} could simply be reduced under the copy theory of
movement,\is{copy theory of movement} as argued in \citet{Gallego2014}.} \gls{FI} has consequences for
the analysis of raising-to-subject,\is{raising} as discussed by \citet{EKS2012}. In
    particular, suppose the derivation of \emph{Don Quixote fought the
    windmills} is as depicted in \eqref{ex:34.6}:

\ea%6
    \label{ex:34.6}
	\ea \{Don Quixote, \{v*\{fought, \{the,windmills\}\}\}\} = v*P
	\ex \isi{Merge} (T,v*P) = \{T, \{Don Quixote, \{v*\{fought, \{the,windmills\}\}\}\}\}
	\ex \isi{Merge} (C,TP) = \{Cφ, \{T, \{Don Quixote, \{v*\{fought, \{the,windmills\}\}\}\}\}\}
	\ex \gls{FI} (C,T) = \{C, \{Tφ, \{Don Quixote, \{v*\{fought, \{the,windmills\}\}\}\}\}\}
    \ex \gls{IM} (\emph{DQ},TP) = \{C, \{Don Quixote, \{Tφ, \{\emph{t}, \{v*\{fought, \{the,windmills\}\}\}\}\}\}
	\z
\z

The problematic steps in \eqref{ex:34.6} are (d) and (e), but (e) more
clearly so. As \citet{EKS2012} discuss, the original (SPEC-less) TP must be
disconnected from C so that the \gls{EA} \emph{Don Quixote} undergoes \gls{IM}
with it; when this new (SPEC-ful) TP is created, and it is then reconnected to
C. The operation is thus ternary, in that \isi{Merge} must target the \gls{EA}, TP,
and C.  Noam Chomsky (p.c.) notes that this is a narrow extension of \isi{Merge}, but
does not depart from it in the way \isi{head movement} does, since the \gls{EA} is
merged with TP, which it is a term of.

So far, as we can see, a key trait of \gls{NTC}/\gls{IC}-constrained \isi{Merge}
(α, β) is that α and β cannot be modified: they are left unchanged, no
features, indices, etc. can be added to them by \isi{Merge}.
\citet{Chomsky2007} gives another twist by noting that while \isi{Merge} cannot
modify α or β, some subsequent operation might:

\blockquote[{\citealt[8]{Chomsky2007}, my emphasis}][.]{Merge
    (X1,\dots{},Xn) = Z, some new object. In the simplest case, n = 2, and
    there is evidence that this may be the only case (Richard Kayne’s
    “unambiguous paths”). Let us assume so. Suppose X and Y are merged.
    Evidently, efficient computation will leave X and Y unchanged (the
    no tampering condition \gls{NTC}). We therefore assume that \gls{NTC}\is{no tampering condition} holds
    unless empirical evidence requires a departure from \gls{SMT}\is{strong Minimalist thesis} in this
    regard, hence increasing the \isi{complexity} of \gls{UG}.\is{Universal
    Grammar} Accordingly, we can
    take \isi{Merge} (X,Y) = \{X,Y\}. \emph{Notice that \gls{NTC}\is{no tampering condition} entails nothing
    about whether X and Y can be modified after Merge} [\dots{}] Under \gls{NTC},
    merge will always be to the edge of Z, so we can call this an edge feature
    \gls{EF} of W}

This observation can probably be related to \citeauthor{Chomsky2015}’s
(\citeyear{Chomsky2015}: 10--11) analysis of phase-head deletion (de-phasing),
which triggers a process that makes a non-phase head inherit all the properties
of a phase\is{phases} head.  De-phasing is put forward in order to account for the fact
that subjects can be extracted from \emph{that}{}-less clauses (an \gls{ECP}
violation in earlier terminology). So, as is well-known, subject extraction
across a CP is ruled out if that is spelled out
(cf.~\citealt{Chomsky1986,Rizzi1990}):\is{empty category principle}\is{ECP|see{empty category principle}}

\ea%7
    \label{ex:34.7}
    [\textsubscript{CP} Who does the book say [\textsubscript{CP} (*that)
        [\textsubscript{TP} \emph{t}\textsubscript{Who} stabbed Caesar ]]]?
\z

\citet{Chomsky2015} reinterprets this phenomenon in order to argue that C can
undergo deletion. This makes T inherit phasehood, which makes it strong, with
no need for a DP to occupy SPEC-T for labeling\is{labelling} reasons (cf.
\citealt{Gallego2017}). More to the point,~\textcite[11]{Chomsky2015} argues
that \blockquote{The natural assumption is that phasehood\is{phases} is inherited by T
    [\dots{}] along with all other inflectional/functional properties of C
    (φ-features, tense, Q), and is activated on T when C is
deleted}.\footnote{Noam Chomsky (p.c.) elaborates on this by noting that the
\gls{NTC} states that an \gls{SO} should not be modified by \isi{Merge}, which
doesn’t literally imply that it cannot be deleted.}

\largerpage
Let us take stock. \gls{NTC}\is{no tampering condition} is the formalization of the idea that computation
applies in an efficient way, so that \isi{Merge} (α, β) cannot modify α and β
themselves. This strong formulation of the \gls{NTC}, which bars tucking in and
derives the \gls{CTM}, captures more than mere cyclicity. In particular, what I
would like to emphasize is that by not letting \isi{Merge} modify what it applies to,
the \gls{NTC}\is{no tampering condition} further captures some form of strict cyclicity too. To see this,
let us go back to \eqref{ex:34.1}, repeated as \eqref{ex:34.8} below:

\ea%8
    \label{ex:34.8}
          \isi{Merge} (α, β) = \{α, β\} = γ
\z

After \REF{ex:34.8}, the workspace contains γ and nothing else, so α and β are
no longer available \citep[243]{Chomsky1995}. At this point, we may want to
merge γ and a new object, δ:

\ea%9
    \label{ex:34.9}
          Merge(δ, γ) = \{δ, γ\}
\z

δ is either internal or external to γ. If external, δ is drawn from the
lexicon. This is \glsfirst{EM}. If internal (e.g., δ = α), then δ is a term of
γ.  Assuming the \gls{NTC}, γ cannot be modified, so it must remain \{α, β\},
which yields \{α, \{α, β\}\}, and thus two copies (occurrences) of α.  More
importantly for our purposes, the strong \gls{NTC}\is{no tampering condition} entails that \{α, β\} must be
left as it is, so merger of α will not tamper with γ by removing α.  There is
no need for an extra operation (Copy) for \gls{IM}, just like it is not needed for
\gls{EM} -- if α were taken from the lexicon, it would not be
copied.\footnote{The problem is more general if α and β remained in the
    workspace, along with γ. As Noam Chomsky (p.c.) points out, it has always
    been assumed that they do not, for the generative procedure constructs a
    single object, not a multiplicity of objects. Changing that convention
    would mean that instead of a generative process for expressions, we would
    be designing a generative process for an arbitrarily large collection of
    expressions. For instance, suppose that we hold that after \gls{EM}(α, β) =
    γ = \{α, β\}, the workspace contains α, β, γ. We then have a new question:
    what is the relation between α in the workspace (call it α1) and α in γ =
    \{α, β\} (call it α2)?  They are either copies or repetitions. If they are
    copies, everything goes haywire. Thus, if we continue to \isi{Merge} to α1
    finally yielding the finite clause FC, and to γ yielding the finite clause
    FC$'$, then the two clauses would contain the two copies α1 and α2, so one
    should be deleted, and if one enters into some relation (say anaphora) then
    the other does, etc. Things get much worse if, as this proposal allows, we
construct simultaneously indefinitely many finite clauses. This is not only
dubious, and in fact makes the notion of \enquote{copy} collapse.}

This said, there are two potentially problematic aspects about the \gls{NTC}.
The first one follows from the very fact that the strong \gls{NTC}\is{no tampering condition} runs into
the empirical problems in \eqref{ex:34.10}:\footnote{If the \gls{NTC}\is{no tampering condition} is restricted to
Merge, as Noam Chomsky (p.c.) notes, then only (10b) and (10c) are
problematic.}

\ea%10
    \label{ex:34.10} Violations of strong NTC\\
    \ea \glsdesc{FI} \parencite{Chomsky2008}
    \ex \gls{IM} to SPEC-T (after \gls{EM} (C,TP)) \parencite{Chomsky2008}
	\ex Tucking-in \parencite{Richards1997}
	\ex Head movement \parencite{Chomsky2001}
	\ex De-phasing \parencite{Chomsky2015}
    \ex Phase-cancellation \parencite{EKS2016}
	\z
\z

Apart from these \emph{local} (phase-bounded) violations of the \gls{NTC},
there is another important observation to be made about the strong \gls{NTC},
namely the redundancy between it and the \gls{PIC}, as I discuss in the
following section.

\section{Transfer and the PIC}\label{sec:34.3}

We have seen that the \gls{NTC}\is{no tampering condition} has two
formulations, strong and weak. Let me express this as follows:

\ea%11
    \label{ex:34.11}
    \ea Strong \gls{NTC}\is{no tampering condition} (NTC\textsubscript{S}) = \glspl{SO} cannot be changed
    by \isi{Merge}
    \ex Weak \gls{NTC}\is{no tampering condition} (NTC\textsubscript{W}) = \glspl{SO} can be changed
    locally, but not by \isi{Merge}
	\z
\z

What I would like to discuss is the fact that NTC\textsubscript{S} is virtually
analogous to the \gls{PIC}. The \gls{PIC} was proposed in order to capture
strict cyclicity, so that \enquote{operations cannot ‘look into’ a phase\is{phases} below}
\citep[108]{Chomsky2000}. \citet{Chomsky2004} relates the \gls{PIC} to the
operation Transfer (a wider version of Spell-out, capturing the interaction
between \glsunset{NS}\gls{NS} and both interfaces), which is defined in \eqref{ex:34.12}:

\ea%12
    \label{ex:34.12}
         Transfer hands D-\gls{NS} over to ${\Phi}$ and to ${\Sigma}$.
         \parencite[107]{Chomsky2004}
\z

In \citet{Chomsky2004}, Transfer makes it impossible for the externalization
systems to access what has been cashed out at previous phases. The possibility
that the same happens in the case of the narrow computation is not so clear:

\blockquote[{\citealt[107]{Chomsky2004}, my emphasis}][.]{When a phase\is{phases} is
    transferred to Φ, it is converted to PHON. Φ proceeds in parallel with the
    \gls{NS} derivation. Φ is greatly simplified if it can “forget about” what has
    been transferred to it at earlier phases; otherwise, the advantages of
    cyclic computation are lost [\dots{}] \gls{PIC} sharply restricts search
    and memory for Φ, and thus plausibly falls within the range of principled
    explanation [\dots{}] \emph{It could be that \gls{PIC} extends to \gls{NS} as
well, restricting search in computation to the next lower phase}}

That the \gls{PIC} does not carry over to the computation is connected to the
existence of structures, in \ili{Icelandic} or \ili{Spanish}, like those in \eqref{ex:34.13}, where T
can agree with the in-situ \gls{IA}:

\ea\label{ex:34.13}
    \{\tn{t}{T}, \{v*, \{V, \tn{ia}{\gls{IA}}\}\}\}
    \begin{tikzpicture}[remember picture, overlay]
        \draw [->, shorten <=.5mm, shorten >=.5mm] (t) -- +(0,-.5) -|
            node[below, pos=.25, font=\small] {Agree} (ia);
    \end{tikzpicture}\vspace{1.5\baselineskip}
\z

Empirically, \eqref{ex:34.13} requires the φ-probe to override the \gls{PIC} and access the
complement domain of v* (see \citealt{Richards2012}). In order to tackle this,
\textcite{Chomsky2001,Chomsky2004} adopts a weak version of the \gls{PIC}, which led to a
scenario analogous to that of the \gls{NTC}, with both strong and weak
versions:

\ea%14
	\ea\label{ex:34.14a}Strong \gls{PIC} (PIC1 or PIC\textsubscript{S})\\
        In phase\is{phases} α with head H, the domain of H is not accessible to operations
        outside α; only H and its edge are accessible to such operations.
        \parencite[108]{Chomsky2000}
	\ex\label{ex:34.14b}Weak \gls{PIC} (PIC2 or PIC\textsubscript{W})\\
        {}[Given structure [\textsubscript{ZP} Z \dots{} [\textsubscript{HP} α
        [H YP]]], with H and Z the heads of phases]: The domain of H is not
        accessible to operations at ZP; only H and its edge are accessible to
        such operations. \parencite[14]{Chomsky2001}
	\z
\z

PIC2 is incompatible with \gls{FI}, so in \citet{Chomsky2008} it is discarded.
Consider the following discussion, which suggests that phases that have been
transferred can in principle be accessed (modulo intervention effects). Chomsky
concludes that the effects of the \gls{PIC} hold for the interfaces, but not
necessarily \gls{NS}:

\blockquote[{\citealt[143]{Chomsky2008}, my emphasis}][.]{For minimal
    computation, as soon as the information is transferred it will be
    forgotten, not accessed in subsequent stages of derivation: the computation
    will not have to look back at earlier phases as it proceeds, and cyclicity
    is preserved in a very strong sense. Working that out, we try to formulate
    a \gls{PIC}, conforming as closely as possible to \gls{SMT}\is{strong
    Minimalist thesis} [\dots{}] \emph{Note that for narrow syntax, probe into
an earlier phase\is{phases} will almost always be blocked by intervention
effects}. One illustration to the contrary is \isi{agreement} into a lower
phase\is{phases} without intervention in experiencer constructions in which the
subject is raised (voiding the intervention effect) and \isi{agreement} holds
with the nominative\is{nominative case} object of the lower phase\is{phases}
(Icelandic). \emph{It may be, then, that \gls{PIC} holds only for the mappings
to the interface, with the effects for narrow syntax automatic}}

\largerpage
\citet{Chomsky2016} in fact argues that Transfer should not eliminate anything
from the \gls{NS}. Otherwise, it would not be possible to explain how the
structures in \eqref{ex:34.15} are formed:\footnote{I put aside another situation where the
    \gls{PIC} is strongly violated: covert \isi{movement}. This matter is pointed out
(not addressed) in \textcites[111]{Chomsky2004}[13]{Chomsky2005}.}

\ea%15
    \label{ex:34.15}
    \ea {}[\tss{α} The idea [\tss{β} that the Earth is round ]] was rejected t\tss{α}
    \ex {}[\tss{α} That [\tss{β} I kept my job ]] seems to t\tss{α} bother Mary
	\z
\z

The problem here is as follows: in both cases, β is a phase\is{phases}, so it should be
transferred before α is raised to matrix SPEC-T. But how can β be pronounced
along with α if it is gone from the computation? \citet{Chomsky2016} claims β
is never gone from the workspace, but rendered inaccessible by Transfer. There
are two ways to interpret this version of the \gls{PIC}, which I will call PIC3:
what’s been processed is either (i) totally inaccessible or (ii) cannot be
changed.\footnote{A reviewer points out that what I call PIC3 is actually a
    conception of Transfer and its effect on transferred material, not the \gls{PIC},
    which “describes the timing of Transfer and the size of the transferred
    object”. For the purposes of this paper, I will not dwell on this (to me,
    largely terminological) issue. The \gls{PIC} was meant to state what is
    accessible and what is not after Transfer (a mapping operation) applies.
    All I am assuming is that the PIC3 says that everything is actually
accessible after Transfer as long as it is not changed.}~Given the data in
\eqref{ex:34.15}, (i) must be dismissed. We therefore expect that violations of the
\gls{PIC} do not change whatever is inside the transferred phase\is{phases}. This
crucially allows us to change what is outside it, including the φ-probe of
matrix T in \eqref{ex:34.16}, taken from \citet{Fernandez-Serrano2016}:

\ea%16
    \label{ex:34.16}Spanish\\
    \gll Me encantan [\textsubscript{CP} PRO escuchar [\textsubscript{v*P} t\textsubscript{PRO} t\textsubscript{v*} [\textsubscript{VP} V  truenos ] ] ]\\
        to.me love-3.pl {} {} listen {} {} {} {} {} thunder\\
    \glt ‘I love to listen to thunder.’
\z

Let us therefore assume the PIC3 allows access into a lower phase\is{phases}, as long as
it is not modified. This makes it difficult to keep the copy\slash repetition
distinction. Take \eqref{ex:34.17}, call it K, where the lower phase\is{phases} complement containing
β, that is \{α, β\}, has already been transferred:

\ea%17
    \label{ex:34.17}K = \{\dots{}\{P, \{α, β\}\}
\z

Imagine we now merge β with K. β could be taken from the lexicon, so it would
be a repetition. Can it be a copy? Given that \{α, β\} is not expunged from the
derivation, the question is whether \gls{NS} can tell whether β is taken from
the lexicon or it is interpreted as an occurrence of the β contained within P’s
complement.  If \{α, β\} can be accessed, the system cannot tell the difference.
But we want to exclude this, or successive cyclic \isi{movement} would go away.
Island conditions would be affected too. Notice that the logic here is clear:
the copy\slash repetition distinction does not require changing anything within the
already passed phase\is{phases}. So, it should be possible to do that, given
\citeapos{Chomsky2016} PIC3.

A way out would be to assume, as Noam Chomsky (p.c.) suggests, that if β raises
from \{α, β\}, then both \{α, β\} and β itself have been modified: \{α, β\} by
now containing a copy that is part of chain, and β by the mere fact of becoming
a discontinuous object. Now, if this is correct, even the application of \gls{IM} to
\emph{Who} changes the v*P and \emph{Who} in \eqref{ex:34.18}.

\ea%18
    \label{ex:34.18}
          \{Who, \{Samson, \{v*, \{defeated, \emph{t}\}\}\}\}
\z

Presumably, this has not been considered problematic, for it does not violate
the \gls{PIC}, but it does the \gls{NTC}\textsubscript{S}. Now, we have seen
that \gls{NTC}\textsubscript{S} and \gls{PIC} are remarkably similar in that
they both capture strict cyclicity. If nothing else, \eqref{ex:34.18} shows
another scenario where I depart from the \gls{NTC}\textsubscript{S}. I take
this to indicate that the \gls{NTC}\textsubscript{S} is to be dispensed with
entirely.  More controversially, I also argue that the NTC\textsubscript{W} is
dispensable, \emph{if} the \gls{PIC} can play its role. Under PIC1, which I
repeat here as \eqref{ex:34.19}, this replacement is possible:

\ea%19
    \label{ex:34.19}Strong \gls{PIC} (\gls{PIC} 1 or PIC\textsubscript{S})
    In phase\is{phases} α with head H, the domain of H is not accessible to operations
    outside α; only H and its edge are accessible to such operations.
    \parencite[108]{Chomsky2000}
\z

What~\eqref{ex:34.19} says is enough to capture the effects of the
\gls{NTC}. In particular, the fact that the objects generated in the course of
the derivation cannot be tampered with. Notice that this \emph{does} allow
tampering \emph{before Transfer applies}, but we have seen that this is
empirically sustained. To the cases listed in \eqref{ex:34.10}, we can add
a sixth one, which follows from the PIC3:

\ea%20
    \label{ex:34.20}Violations of NTC\textsubscript{S}\\
    \ea \glsdesc{FI} \citep{Chomsky2008}
    \ex \gls{IM} to SPEC-T \citep{Chomsky2008}
	\ex Tucking-in \citep{Richards1997}
	\ex Head movement \citep{Chomsky2001}
	\ex De-phasing \citep{Chomsky2015}
    \ex Phase-cancellation \parencite{EKS2016}
    \ex \gls{IM} (chain creation)
	\z
\z

In the next section, I would like to summarize the main ideas of the previous
pages and, at the same time, argue that the PIC3 can be eliminated in favor of
the PIC1. In so doing, I also discuss how the data mentioned in
\citet{Chomsky2016} can be handled under such proposal. The proposal entails
that Transfer eliminates material from the workspace, yielding a more effective
reduction of computational load -- the original motivation behind phase theory
\parencite[cf.][]{Chomsky2000}.

\newpage\section{NTC eliminated: Some consequences}\label{sec:34.4}

Let me spell out the interim conclusions so far. I will phrase them as
questions:

\ea%21
    \label{ex:34.21}
    \ea Do we need both \gls{NTC}\is{no tampering condition} and the
    \gls{PIC}?\label{ex:34.21a}
	\ex If we need the \gls{PIC}, do we need the PIC3?\label{ex:34.21b}
	\z
\z

Both \gls{NTC}\is{no tampering condition} and \gls{PIC} express an efficiency
desideratum, namely that a given \gls{SO} should not be changed (manipulated,
tampered with, altered, etc.) once it has been created. This creates a
redundancy, as I have pointed out.\footnote{A reviewer does not see the
    redundancy, as (s)he takes the \gls{NTC}\is{no tampering condition} to be a
    third-factor condition on \isi{Merge} (defining a Merge-cycle that adds
    stuff to the derivation) and the \gls{PIC} to be a natural result of
Transfer (which removes stuff from the workspace). Given the (empirical)
arguments given below (and in~\citealt{ChoGalOtt2019}) it is unlikely that the
\gls{PIC} actually removes stuff from the workspace.} At the same time, we have
seen different phenomena indicating that the strong version of the \gls{NTC}
cannot be maintained. Should the weak version be? I think it should not, just
like the weak \gls{PIC} (the one in \citealt{Chomsky2001}). This raises the
more general question whether the strong \gls{PIC} could be the only cyclic
principle. If so, then the derivation can allow tampering up to the
phase\is{phases} level, when Transfer applies. Suppose the derivation has
assembled α and β to yield this:

\ea%22
    \label{ex:34.22}
    \{α, β\}
\z

Suppose next that we apply \gls{IM} to β. If the \gls{NTC}\is{no tampering
condition} does not hold, this could yield \eqref{ex:34.23}, potentially
affecting the \gls{CTM}.

\ea%23
    \label{ex:34.23}
    \{β, \{α\}\}
\z

Note that this derivation is not forced (thus, the \gls{CTM} does not go away),
but the question is whether the step in \eqref{ex:34.23} creates a problem. It is not clear
that it does, at least if something like \eqref{ex:34.23} is at stake for de-phasing
\parencite[cf.][]{Chomsky2015}.

If the only cyclic condition is the \gls{PIC}, the next question is
\REF{ex:34.21b}.  Recall that there are two empirical arguments to sustain it.
The \isi{agreement} facts (cf.~\ref{ex:34.16}) could be tackled if \isi{Agree}
takes place at the border of \gls{NS}-externalization, not in \gls{NS}. This
would have two welcome consequences. On the one hand, we could explain the
parametric nature of \isi{Agree}, which I would like to relate to
\citegen{Chomsky2014} \emph{thesis T}:

\ea%24
    \label{ex:34.24}
    Language is optimized relative to the \gls{CI} interface alone, with
    externalization a secondary phenomenon. \parencite[7]{Chomsky2014}
\z

The \emph{thesis T} tells us that efficiency of operations should be found in
the \gls{NS} $\rightarrow$ SEM channel, not in the \gls{NS} $\rightarrow$ PHON
one, which is further consistent with the claim that
\textquote[{\citealt[7]{Chomsky2014}}]{language is primarily an instrument of
thought, with other uses secondary}. If \isi{Agree} is pushed to NS ${\rightarrow}$
PHON, then the fact that its effects are subject to parametrization (as appears
to be the case), would fall into place, and would also be compatible with the
idea that language variation and parametrization are to be found only there
(\citeaposalt{Chomsky2001} uniformity principle; cf.~\citealt{Chomsky2010,BerCho2011}).

Another consequence of this concerns the very nature of \isi{Agree}, which is a
complex operation, consisting of Match, Valuation, Transfer and Deletion.
Chomsky (\citeyear{Chomsky2004} et seq.) takes these operations to somehow apply simultaneously
(at the phase\is{phases} level), but this is hardly consistent with a derivational system,
for operations must be ordered (as in \citealt{Chomsky2015}).\footnote{If
    Transfer is part of externalization, then it can be subject to
    parametrization (for the same reasons \isi{Agree} would be). This opens the
possibility that the effects of Transfer vary from language to language
(cf.~\citeaposalt{Uriagereka1999} radical or conservative Spell-out).}
Plausibly, the operations should be ordered as follows:

\ea%25 % not sure how to number subexamples
    \label{ex:34.25}
    \begin{enumerate}
        \item Match (\gls{NS})\\
        \item Valuation (\gls{NS})\\
        \item Transfer (\gls{NS} ${\rightarrow}$ SEM/PHON)\\
        \item Deletion (PHON)\\
    \end{enumerate}
\z

As noted in \citet{EpSee2002}, this timing is problematic, since it entails that
uninterpretable features will be valued before Transfer, becoming
undistinguishable from interpretable ones. Unless Deletion could apply at S\gls{EM}
too somehow deleting the uninterpretable, but valued, φ-features of v* and C,
operations would have to apply simultaneously, which, as noted, is odd within a
derivational system. A way out is at hand if the derivation can somehow
remember that φ-features were introduced as unvalued. This should be possible,
given the relevance of phase-level\is{phases} memory to distinguish trivial\slash non-trivial
chains, which in its most direct interpretation would entail revamping the
long-abandoned idea or feature chains (\cites[262, 270--271, 383, fn.\
27]{Chomsky1995}, abandoned in \citealt{Chomsky2000} due to the intricacies of
head \isi{movement}). So, if \isi{Merge} could apply not only to LIs, but also to
features -- more precisely, to their values, which is what seems to be copied from
one LI to another, then this would assimilate Valuation to \isi{Merge}, making it
possible for the system to remember that a valued feature was introduced as
unvalued, which would signal it as uninterpretable. The technical solution I am
sketching would not be too different from \gls{FI} itself. In brief, we could
dispense with the simultaneity of operations and perhaps the need for \isi{Agree} to
apply in \gls{NS} alone if \isi{Merge} could apply to LIs, features and values.

\citeapos{Obata2010} data are different. Consider \eqref{ex:34.26}:

\ea%26
    \label{ex:34.26}
    [α That [β Judas left the dinner ]] seemed [ to \emph{t}\tss{α} worry everyone ]]
\z

Here β is transferred before α is raised to matrix SPEC-T,
which makes it impossible for it to be spelled-out where we see it.~However,
even if we assumed that the \gls{PIC} leaves β accessible (through the PIC3),
this does not cover \gls{IM}.~That is, it is only α (presumably its head,
\emph{that}) that can raise to matrix SPEC-T, so how can β be
pied-piped along with α? If we allowed that, then we would also be
changing the already transferred object, as noted for \eqref{ex:34.18} above. A possible
way out for these cases is that what is transferred is turned into a pair
⟨X,Y⟩. I would like to connect this to \citegen{Chomsky2004} analysis of adjuncts,
which adopted \eqref{ex:34.27}:\footnote{Cf.~\citet[139]{Chomsky2008} for similar ideas
    in the case of \isi{Merge}.}

\ea%27
    \label{ex:34.27}
    In \tuple{α, β}, α is spelled out where β is. \parencite[199]{Chomsky2004}
\z

If Transfer converts the structure into some kind of pair, then when \gls{IM}
targets α, the actual pronunciation of β (or some part of it) could be
possible.  This would have the effect of placing β in a “secondary plane”
\citep{Chomsky2004}, but we want α (the phase\is{phases} edge), and α alone, to
remain in the primary plane.  This is what the PIC1 bought us, which brings
back the possibility that Transfer can yield \eqref{ex:34.28}, removing the
complement domain from \gls{NS} (cf.~\cite{Ott2011}):

\ea%28
    \label{ex:34.28}
	\ea \{Edge, \{P, \{β\}\}\}
	\ex Transfer (β) = \{Edge, \{P\}\} or \{Edge, P\}
	\z
\z

If Transfer applies this way, there would be tampering, but locally. \eqref{ex:34.28} would
make it possible for P to be the head of the entire phase\is{phases}, with consequences
for the v*-\gls{EA} relation \parencite[cf.][]{EpShim2015}.

\section{Conclusions}\label{sec:34.5}

This paper has discussed the nature of different conditions put forward to
capture computational efficiency within minimalism, most importantly, the
\gls{NTC} and the \gls{PIC}. Given their redundant nature (they both aim at
capturing the idea behind the strict cycle, namely that \glspl{SO} formed in
the course of a derivation cannot be changed at subsequent stages), one of them
should be dispensed with.  I have argued that strict cyclicity effects follow
from the \gls{PIC} alone. The decision is justified on methodological and
empirical grounds. The former have to do with the multiplicity of conditions
favoring strict cyclicity. The latter concern the empirical evidence showing
that the strong version of the \gls{NTC}\is{no tampering condition} cannot be maintained.

The strong \gls{PIC} (or PIC1 cf.~\citealt{Chomsky2000}), which is the one that
should be adopted, forces \gls{SCM}. Since nothing is left in the (primary
plane of) computation after Transfer, that’s the only way for a chain to be
created. It also follows that the \gls{SO} that has been cashed out cannot be
modified: it is gone from the workspace. Interestingly, there are no violations
of the \gls{PIC} analogous to those of the strong \gls{NTC}, which is another
argument to stick to the former. Interestingly, it seems that only CP and vP
give rise to SCM -- NPs, PPs and other categories lack it
\parencite[cf.][]{Gallego2012,vanUrk2016}, which may provide yet another reason
to defend that only CP and vP are phases.

\printchapterglossary{}

\section*{Acknowledgements}

I am very happy to contribute to this volume in honor of Ian Roberts, a key
figure in the field of Generative Grammar. I had the opportunity to work with
Ian back in 2008, when he supervised a British Academy visiting fellowship I
was awarded with, right after I became a doctor. I remember that experience
(with long appointments at Ian's office) as a very important one in my career
and in my personal growth too.

A previous version of this paper was presented at the University of Michigan,
in a talk organized within the UMich linguistics colloquium series on 10 March
2017. I would like to thank the audience of that talk for questions and
comments. For discussing these matters with me, I am also indebted to Noam
Chomsky, Sam Epstein, Hisa Kitahara, Dennis Ott, and Daniel Seely. This
research has been partially supported by grants from the Ministerio de Economía
y Competitividad (FFI2017-87140-C4-1-P), the Generalitat de Catalunya
(2017SGR634), and the Institució Catalana de Recerca i Estudis Avançats
(ICREA Acadèmia 2015). Usual disclaimers apply.

{\sloppy\printbibliography[heading=subbibliography,notkeyword=this]}
\end{document}
