\documentclass[output=paper]{langsci/langscibook}
\ChapterDOI{10.5281/zenodo.4280651}
\author{Ellen Brandner\affiliation{Universität Stuttgart}}
\title{Re-thinking re-categorization: Is \emph{that} really a complementizer?}

% \chapterDOI{} %will be filled in at production

\abstract{Following \citegen{Kayne2014} argumentation that the complementizer
    \emph{that} is indeed a relative pronoun and with it the complement clause
    a special type of relative clause (explicative, i.e.\ without a gap), the
    paper contributes to the discussion whether \emph{that}-complement clauses
    are also structurally \isi{relative clauses}. One consequence of this would be
    that \emph{that}-clauses should not allow long wh-extraction, contrary to
    what is observed in languages like English at first sight. However, the
    distribution of resumptive pronouns in Alemannic, a Southern \ili{German}
    dialect, indeed points into that direction. Like the \ili{Celtic} languages,
    Alemannic has a special particle for \isi{relative clauses} but can use the
    d-pronoun strategy as well. Both strategies can be used to build long
    distance dependencies alike. But \isi{resumptive pronouns} are nearly
    obligatory with \emph{that}-clauses in sharp contrast to those involving
    \isi{relative clauses}.  This difference can find an explanation, if the
    particle-strategy creates a genuine gap in the embedded clause whereas a
\emph{that}-complement clause is always a full-fledged clause and the gap in it
is only apparent, its appearance regulated by outer-syntactic criteria.}

\maketitle

\begin{document}\glsresetall

\section{Introduction}

The more or less established analysis of \isi{complementizers} of the English
\emph{that}-type is that they evolved out of pronominal elements, most commonly
the (distal) demonstrative\is{demonstratives} pronoun:

\ea\label{ex:36.1}
	\textit{That} guy over there gives me a headache (demonstrative)
\ex\label{ex:36.2}
	Do you believe \textit{that}? (anaphoric)
\ex\label{ex:36.3}
	I believe [\textit{that}\dots{}]  (complementizer)
\z

The diachronic scenario, already proposed in very early\footnote{For example
\citet{MullerFrings1959}, but the idea can already be found in very early work
from the 19th century, see \textcite{Axel2009,Axel2017} for a survey and
further references.} work, assumes that \emph{that} (and its equivalents in
the other \ili{Germanic} languages) originated as a (cataphoric) pronoun to the
following (independent) clause.  A re-bracketing of the clausal boundaries
posited the pronoun then to the left edge of the embedded clause, see e.g.\
\citet{RobRou2003} for an explicit proposal:

\ea\label{ex:36.4}
	I say \textit{that}: [ main clause ] $\rightarrow$ I say [ \textit{that}  embedded clause ]
\z

This process involves in addition to the re-bracketing a
re-categorization\is{lexical categories} of
\emph{that} such that the previously pronoun enters into the class of
C-elements and thus belongs now to the \enquote{word class} of \isi{complementizers}. As such
it occupies the C\textsuperscript{0}-position, i.e.\ it has not only changed
its word class but also its phrase structural status in that it is re-analyzed
as a head. \Citet{vanGelderen2004} takes especially this type of reanalysis
(Spec-to-head) as a hallmark of the \isi{grammaticalization} process. Evidence for
the head-status of complementizer-\emph{that} is seen in the fact that
\emph{that}-clauses allow already in the early stages (e.g.\ on Old High
German)\il{Old High German} for long \isi{wh-extraction} -- a process which must rely on an empty
specifier in the CP as an available intermediate landing site, see
\textcite{Axel2009,Axel2017} for this line of reasoning. This scenario is
assumed to not only be true of \ili{German}; the same process has taken place in
English and the other \ili{Germanic} languages.

Now various authors have cast doubt on the assumption that there is indeed such
a re-analysis process and ask whether speaking of a category C (in the sense of
a word class) is at best misleading -- in the worst case it is blurring the
actual problem to be solved, e.g.\
\textcite{Kayne2014,ManziniSavoia2003,ManSav2011}.  These authors suggest that
we should follow the “WYSWYG-principle” and under this premise \emph{that} (and
its cognates in other languages) is indeed never something else than a pronoun.
While Manzini \& Savoia remain a bit vague about its actual status -- besides
the claim that \ili{Romance} \emph{che} (`what') is a quantificational element whose
restrictor can also be a proposition (= acting then as a complementizer), Kayne
states plainly that \emph{that} is always a relative pronoun and accordingly
complement clauses are always \isi{relative clauses}, construed with a (possibly
empty) correlate pronoun in the matrix clause.

This is essentially the analysis proposed in \textcite{Axel2009,Axel2017}. She
rejects the re-bracketing analysis, based on data in \gls{OHG}\il{Old High
    German}.\footnote{Recall that
    in \gls{OHG}, there is a clear distinction between root and embedded clauses due
    to the position of the finite verb (V2 order vs.\ verb final in embedded
clauses).} Like \citet{Kayne2014}, she proposes that \emph{that} is a relative
pronoun, belonging thus to the embedded clause from the beginning on, and
assuming that there is a (possibly silent) head noun in the matrix clause.
This is in spirit very close to \citet{Kayne2014}.\footnote{The difference to a
    ``usual'' relative clause \is{relative clauses}is that there is no overtly detectable gap in it.
    This has to do with the type of the head noun that is modified by the
    relative clause:\is{relative clauses} it is clearly a kind of a direct object (realizable as a
    correlate pronoun). The semantic content of this pronoun is actually a
    proposition -- and the relative clause \is{relative clauses}is delivering the content of this
    proposition. This might be formally analysed in terms of an
    \emph{aboutness} \emph{relative}, i.e.\ a gap-less one, see
    \textcite{vanRiemsdijk2003}, \textcite{CheSyb2005}, as suggested in
\textcite{BrandnerBucheli2018}, also \textcite{Axel2009,Axel2017}.\label{fn:36:3}}

The scenario in~\eqref{ex:36.4} would then look like the one
in~(4$'$).

\begin{exe}
    \exi{(4$'$)} I say (\textit{that/it}) [ \textit{that} \dots{}embedded clause (= relative clause)]
\end{exe}

By showing that long wh-extractions\is{wh-extraction} already exist at this stage of the
language, a crucial component for her analysis is the Spec-to-head reanalysis --
as only in this configuration, long \isi{wh-extraction} is possible, due to the
now empty specifier.

On the other hand, if one follows the Kayne-analysis according to which the
\enquote{complementizer} is indeed a relative pronoun, one would expect that long
wh-extraction out of a \emph{that}-clause cannot exist at all -- given that
relative clauses are for sure one of the strongest \isi{islands} for extraction.

In this paper, I will show that there are good reasons to think that Kayne’s
position is actually correct: there is evidence from the \ili{Alemannic} dialect,
spoken in Southern Germany and Switzerland, that there is no long (cyclic)
wh-movement out of \emph{that}-type complement clauses and what looks like
extractions -- leaving behind a gap -- consists of a base-generated wh-phrase in
the matrix clause and an actually full-fledged complement clause with a
pronoun filling the \enquote{extraction-site}. This pronoun can be PF-deleted under a
rather weak principle like e.g.\ the \isi{avoid pronoun principle}
\citep{Chomsky:81}, giving thus merely the impression of actual \isi{movement}.

However, the grammar has a strategy to build long wh-dependencies
(\glsunset{LWD}\glspl{LWD}) with real gaps -- but this is only possible if the
gap in the embedded clause is a genuine gap, coming into existence via a
special type of complementizer\is{complementizers}, used normally in the formation of relative
clauses, turning the embedded clause into a predicate. The situation I am
referring to is described and analysed in \citegen{AdgRam2005} work on
\glspl{LWD} in \ili{Gaelic} (Celtic). I will present evidence here that the very same
strategy is used in some variants of \ili{Germanic} as well. But in contrast to
\citet{AdgRam2005} who suggest that there is a parametric difference
between \ili{Celtic} and \ili{Germanic} (English in this case) which allows the derivation
of genuine long wh-extractions\is{wh-extraction} in the latter, I will show that this is not true
for at least \ili{Alemannic}. Further and more detailed research -- along the lines
that will be presented here -- will be necessary to make the point valid also
for English and other \ili{Germanic} languages -- actually for all languages that have
to be claimed to exhibit long wh-extractions\is{wh-extraction}. I am aware that this is a far
reaching claim -- still the data presented should be taken to be an invitation
to re-think in general the issue of long wh-extractions\is{wh-extraction}.

The data that support this suggestion come from the Southern \ili{German} dialect
Alemannic (ALM). A large scale study about \glspl{LWD} in the whole Alemannic
speaking area revealed that this language uses the same strategy to build
\glspl{LWD} as the \ili{Celtic} languages. In addition, however -- and in contrast
to the \ili{Celtic} languages -- \ili{Alemannic} shows \glspl{LWD} with
\emph{that}-clauses, indicating that a parametric solution as proposed in
\citet{AdgRam2005} is probably not the right way to look at it. Secondly, it
will be shown below that these seemingly extractions are in reality no
extractions at all. The main evidence comes from the distribution of
\isi{resumptive pronouns} that occur in these “extractions”. They occur to such
a high percentage that it leaves no room for an actual extraction analysis.
Especially, if one assumes that resumptives\is{resumptive pronouns} are
inserted to \enquote{rescue} an otherwise impossible structure
(island\is{islands}
violations) or reduce parsing complexity, see \citet{ChaoSells1983}, it would
remain a complete mystery why the very same complexity allows or even requires
a gap when the \gls{LWD} is built via relative clause \is{relative
clauses}formation.

\section{The two strategies}\label{sec:36.2} %2. /

\glspl{LWD} in \ili{Alemannic} show up in several versions. Besides the
familiar strategies that are also found in Standard \ili{German} (or at least
the spoken variants of it), see the examples in (\ref{ex:36.5}a--c), there is a possibility
that has to my knowledge not been noted until now, see for a first description
\textcite{BrandnerBucheli2018}, illustrated with Standard \ili{German} wording
in (\ref{ex:36.5}d):

\ea\label{ex:36.5}German\\
    Wen hast du gesagt\dots{}
    \ea {}[ dass Maria \underline{\hphantom{2em}} gesehen hat ]\hfill \emph{dass}-\gls{LWD}
    \ex {}[ wen Maria\underline{\hphantom{2em}} gesehen hat ]\hfill copy const.
    \ex (was) [ wen Maria \underline{\hphantom{2em}} gesehen hat ]\hfill \emph{was}-\emph{w}-constr.
    \ex {}[wo Maria \underline{\hphantom{2em}} gesehen hat ]\hfill \emph{wo}-\gls{LWD}
    \sn \enquote*{Who did you say that Mary saw?}
    \z
\z

The interesting thing about the strategy in (\ref{ex:36.5}d) is that the
complementizer\is{complementizers} in the embedded clause corresponds to the
one used regularly in \isi{relative clauses} in this variety, cf.\
\eqref{ex:36.6}, glossed as \Rci{} (relative clause introducer); note that
the declarative complementizer\is{complementizers} in ALM is \emph{dass},
glossed as \Cci{} (complement clause introducer), like in Standard
\ili{German}:

\ea\label{ex:36.6} Alemannic\\
    \gll d'frau [ \textit{wo}-n-i geschtert \underline{\hphantom{2em}} troffe ha ] \\
    {the woman} {} {\Rci{}\hphantom{-n-}I} yesterday met have \\
\z

\ea\label{ex:36.7} Alemannic\\
    \gll mir het er gseet [ \textit{dass} er erscht schpöter kunnt ]\\
        me has he told {} \Cci{} he only later comes \\
\z

Examples like (\ref{ex:36.5}d) showed up first during the survey period of
SADS\footnote{\emph{Syntaktischer Atlas der deutschen Schweiz}, 
(\url{http://www.dialektsyntax.uzh.ch/de.html}).} where informants offered it
as one possible version to express a \gls{LWD} of the type given in
(\ref{ex:36.5}a). In the project SynAlm,\footnote{The study was conducted
    within the DFG-supported project SynAlm {\sloppy
    (\url{https://ilg-server.ling.uni-stuttgart.de/synalm/html/})}. Its funding
    time was from 2011--2015. SynAlm gathered its data via written
    questionnaires, mostly using judgments (5-point scale) for examples
    constructed as minimal pairs. Seven questionnaires were sent out. The
number of informants range from 580 to 1000. No informant was excluded but data
concerning age, social status, and origin (also of the parents) were
collected.} these were then examined in more detail and contrasted with the
\enquote{usual} strategy, i.e.\ \emph{dass}-\glspl{LWD}. It turned out that both
strategies are possible in \ili{Alemannic} and are in more or less free
variation. The large scale investigation (about 580 speakers) in the whole
\ili{Alemannic} speaking area (Switzerland, Southwest Germany, Alsatian and
Austria) conducted by SynAlm concerning \emph{wo}-\glspl{LWD} revealed the
following main results:

\begin{itemize}

    \item \emph{wo}-\glspl{LWD} were throughout accepted by more than 50\% of the
        speakers, notably the acceptance/rejection is essentially the same as
        with \emph{dass}-\glspl{LWD}\footnote{\glspl{LWD} are generally accepted only by a
        certain amount of speakers. This holds for Standard \ili{German} as well as
    for the dialects. It should also be kept in mind that there are various
strategies at the disposal (copy-construction, scope marking etc.). The
informants had always the possibility to give an own version of the sentence
asked for. In many cases, the informants judged the presented example as bad
and chose a parenthetical construction as an alternative, i.e.\ where there is
no extraction at all.}

    \item no clear areal patterns could be detected, i.e.\ it is not the case
        that there are certain (areally definable) sub-dialects of Alemannic
        that allow for \emph{wo}-\glspl{LWD} whereas others do not. Instead, it seems
        that \ili{Alemannic} speakers have simply both possibilities at their
        disposal.

    \item there was no effect with respect to age: younger speakers accepted
        the construction to the same percentage as older speakers.

\end{itemize}

Now \ili{Alemannic} is not the only language that has a special complementizer\is{complementizers} in
\glspl{RC}. The \ili{Celtic} languages are well-known for using a similar strategy
like \ili{Alemannic} by employing a specialized particle in \glspl{RC}, see e.g.
(\citealt{McCloskey2001,McCloskey:2002a} and following work) for \ili{Irish}. The
\enquote{typical} complementizer\is{complementizers} for complement clauses is illustrated in
(\ref{ex:36.8}a).  (\ref{ex:36.8}b) illustrates an \gls{RC}, compare
these with the ALM clauses in~\eqref{ex:36.6} and~\eqref{ex:36.7}:

\ea\label{ex:36.8}Irish
    \ea
        \gll Deir siad [ \textit{gur} ghoid na síogaí í ].\\
            say they {} go-\Pst{} stole the fairies her\\
        \glt ‘They say that the fairies stole her away.’
    \ex
        \gll an ghirseach [ \textit{a} ghoid na síogaí \underline{\hphantom{2em}} ]\\
        the girl {} \Rci{} stole the fairies\\
        \glt ‘the girl that the fairies stole away’
    \z
\z

The \glspl{LWD} in~\eqref{ex:36.9} and~\eqref{ex:36.10} show that it is the \Rci{}
that occurs in \glspl{LWD}, whereas \glspl{LWD} out of a \emph{go} (= \emph{gun})-clause are
impossible:

\ea\label{ex:36.9} \ili{Irish}
    \ea
        \gll Cé   a  mheas    tú   \textit{a}  chonaic  tú? \\
             who   aL   thought  you   aL  saw   you \\
        \glt ‘Who did you think that you saw?’
    \ex
        \gll Cén  t-úrscéal  a   mheas  mé  \textit{a}   dúirt  sé  \textit{a}   thuig  sé. \\
             which  novel       aL   thought I  aL   said  he  aL understood he \\
        \glt ‘Which novel did I think he said he understood?’
    \z
\z

\ea\label{ex:36.10} \ili{Irish}\\
        \gll    \llap{*}Dè      a      thuirt sibh     \textit{gun}   sgrìobh i?   \\
        what  C-\Rel{} said you       that   wrote she\\
        \glt    ‘What did you say that she wrote?’
\z

Welsh shows a comparable pattern -- although the fact that the \gls{LWD} is built on
a relative clause \is{relative clauses}can be seen here only indirectly since the relative particle
does not show up overtly: however, the embedded verb in \glspl{LWD} is in the
so-called \enquote{relative form}, the morpho-syntactic reflex of having a gap in the
clause. \ili{Welsh} examples taken from \citet[555]{Willis2000}.

\ea\label{ex:36.11} \ili{Welsh}\\
    \gll Beth ych chi ’n gredu \textit{sy} ’n wir bwysig miwn cymdeithas?\\
         what are you \Prog{} believe-\Vn{} {is-\Rel} \Pred{} truly important in society\\
    \glt ‘What do you think is truly important in society.’
\z

Even other \ili{Germanic} languages are reported to allow for structures similar to
the one in (\ref{ex:36.5}d). The following pattern is from \ili{Norwegian}
\parencite{WesVanLoh2012}:

\ea \ili{Norwegian}
    \ea
        \gll    Hvem tror du [ \textit{som} \underline{\hphantom{2em}} har gjort det ]?\\
        who  think  you {} \Rci{} {} has  done  it \\
        \glt    ‘Who do you think has done it?’ \\
    \ex
        \gll    Hvem  tror  du [ \textit{at} \underline{\hphantom{2em}} har   gjort  det ]?  \\
        who  think  you {} \textit{that} {} has  done  it \\
        \glt    ‘Who do you think has done it?’
    \z
\z

In sum, \glspl{LWD} based on an RC-structure are quite common -- also in the \ili{Germanic}
languages -- and they occur as an alternative to the (until now) more widely
attested \emph{dass}-\glspl{LWD}, together with the scope-marking and copying
constructions -- and of course with parenthetical constructions -- which seem to
be always a possibility.

In SynALm, the acceptance/rejection of \isi{resumptive pronouns} was systematically
tested against these various types of \glspl{LWD} and it is this last set of data that
gave the crucial clue for the claim from above, namely that in
\emph{dass}-clauses, there is merely an apparent \enquote{gap} and it is only in
\emph{wo}-\glspl{LWD} where genuine gaps show up.

\section{Distribution of resumptive pronouns} %3. /

Until now, we have only seen that \ili{Alemannic} is similar to the \ili{Celtic} languages
in that it allows \glspl{LWD} based on \glspl{RC}. However, the important
difference is that \ili{Alemannic} (together with \ili{Norwegian}) allows \glspl{LWD} based
on \emph{dass}-clauses as well~-- in sharp contrast to Celtic.  Given the
considerations from above, namely that \emph{dass} is a real relative pronoun,
it is the \ili{Celtic} languages that behave as expected. The possibility of
\glspl{LWD} in the \ili{Germanic} languages (including of course English) is then the
fact to be explained.

In the following, I will use the distribution of \isi{resumptive pronouns} in the
various types of \glspl{LWD} to show that \enquote{extraction} out of \emph{dass}-clauses is
indeed an illusion: all the extracted arguments can be realized as pronouns and
whether they are spelled-out overtly or not is a matter of \gls{PF} -- where
(non-syntactic) factors like distance etc.\ play a role.

\subsection{Resumptive pronouns in \ili{Alemannic} relative clauses} %3.1. /

Before going into the details of the distribution of resumptives\is{resumptive pronouns} in \glspl{LWD}, a
brief illustration of the occurrence of \isi{resumptive pronouns} in simple \glspl{RC} in
Alemannic is necessary: it has often been claimed in the literature on
Alemannic \glspl{RC} (in this case specifically on Zürich \ili{German}), that in case of
datives\is{dative case} and the oblique positions further down in the Keenan/Comrie hierarchy,
resumptives occur obligatorily, see \citet{vanRiemsdijk2003},
\citet{Salzmann2006} among others. Thus, whereas with subjects and objects,
resumptives\is{resumptive pronouns} never show up, they occur from the dative-position on, illustrated
here only with a dative-argument and a subject-rela\-ti\-vi\-za\-tion:\is{dative case}

\ea\label{ex:36.13}Zürich \ili{German}
    \ea
    \gll    der Bue [ wo ma \textit{em} s’Velo verschprooche het ] \\
    the boy {} \Rci{} one him the=bike promised has \\
    \glt    \enquote*{the boy, who was promised to get a bike}\\
    \ex
    \gll    der Bue [ wo-n\footnotemark-(*\textit{er}) zschpot kummen-isch ] \\
            the boy {}  {\Rci{}\hphantom{-n-(*}he}    too late come        is\\
    \glt    ‘The boy who arrived too late’
    \z
\z
\footnotetext{\emph{-n-} is an epenthetic consonant and is of no relevance here.}

In SynAlm, it could be shown, that this claim is empirically not tenable.
Although it is true that there never occur resumptives\is{resumptive pronouns} with subjects and
(direct) objects, one can hardly speak of \enquote{obligatoriness of
dative-resumptives} in light of an acceptance rate ranging between
9--15\%.\footnote{Many more sentences with dative-resumptives\is{dative case} were tested and
    the result was basically the same with some minor variation -- having
    probably more to do with the general naturalness of the example and other
linguistically insignificant factors.} With the oblique-positions further down
in the Keenan/Comrie hierarchy, the acceptance/requirement of a resumptive
increases accordingly. So we can safely conclude that the occurrence of
resumptives in simple \glspl{RC} follows the expected distribution -- whatever the
ultimate (syntactic) reason behind the pattern described in the Keenan/Comrie
hierarchy -- may be.\footnote{I will not take a stand here whether this has to
    do with the necessity to realize oblique\slash morphological
    case\is{case!morphological case} -- as suggested
    in \citet{Salzmann2006} or whether different factors are at stake, see for
    some speculations \textcite{BrandnerBucheli2018}. It should be noted that
    informants who did neither accept a gap nor a resumptive\is{resumptive pronouns} in the
    relativization of oblique positions adhered simply to a bi-clausal
    structure, i.e.\ the formation of an \gls{RC} was avoided.}

\subsection{Resumptive pronouns in simple LWDs}\label{sub:36.3.2} %3.2. /

Equipped with this background let us now turn to the distribution of
resumptives in \glspl{LWD}, both based on \emph{wo}-\glspl{RC} and
\emph{dass}-clauses. The expectation for the \emph{wo}-\gls{LWD}s is that they
show a comparable distribution of resumptives\is{resumptive pronouns} as in simple \glspl{RC} -- given
that they have both the same underlying syntax.\footnote{Recall that I assume
    with \citet{AdgRam2005} that the wh-phrase in the matrix is base-generated
    there and the gap in the embedded clause is licensed by a local
    configuration with the respective complementizer\is{complementizers} whose internal lexical
    specification allows/requires a gap in its complement (the so-called
    lambda-feature). I refer to their work for the technical details.} In
    \emph{dass}-clauses on the other hand, the assumption of an extraction strategy
    would one lead to expect that gaps are predominant.  However, it turns out
    that the results are essentially the opposite: resumptives\is{resumptive pronouns} are accepted to
    a much higher degree in \emph{dass}-\glspl{LWD}. The results concerning the
    acceptance of resumptives\is{resumptive pronouns} are given in \tabref{tab:36.1}.

\begin{table}
\begin{tabular}{lrrr}
\lsptoprule
{Type of \enquote{extracted} phrase} & {\emph{dass}-LWD} & {\emph{wo}-LWD} & {\emph{wo}-RC}\\
\midrule
{subject}  & {70\%} & {9\%} & --\\
{direct object} & {30\%} & {5\%} & --\\
{dative object} & {43\%} & {12\%} & {15\%}\\
{adjunct}  & {60\%} & {62\%} & {51\%}\\
\lspbottomrule
\end{tabular}
\caption{Acceptance of \isi{resumptive pronouns} in different types of \glspl{LWD} and \glspl{RC}
($n = 580$).}\label{tab:36.1}
\end{table}

\largerpage
Although there occur resumptives\is{resumptive pronouns} also with \emph{wo}-\glspl{LWD} with
subjects\footnote{The high acceptance of a resumptive in subject-\glspl{LWD} does not
    really come as a surprise -- since -- as is well known since the work by
    \citet{Engdahl1985}, resumptives\is{resumptive pronouns} in subject positions may occur to avoid an
    \glsunset{ECP}\gls{ECP}-violation (\emph{that}-trace-effect).\is{Empty Category Principle} In light of the discussion, this fact should
be reconsidered again.}  and (direct) objects to a certain extent -- whereas
they are categorically excluded in genuine \isi{relative clauses} -- the important
difference is the acceptance rate of resumptives\is{resumptive pronouns} in \emph{dass}-\glspl{LWD}. For
subjects, it is evident. The lower acceptance of resumptives\is{resumptive pronouns} (or rather the
possibility to have a gap) in direct object position may have to with the fact
that many simple transitive verbs have a grammatical output when used as a mere
activity verb (\emph{I read a book} vs.\ \emph{I read}). But this has to be investigated in more detail in future
research.

On the other hand, resumptives\is{resumptive pronouns} for datives and obliques in \emph{wo}-\glspl{LWD} show
a rather even distribution with their occurrence in \emph{wo}-\glspl{RC}. In
\emph{dass}-\glspl{LWD} again, datives\is{dative case} have a considerably higher acceptance whereas
the adjunct behaves similar under all conditions. I will not go into a thorough
discussion of these results~-- since I will take them here merely as a first
hint that the resumptives\is{resumptive pronouns} in \emph{dass}-\glspl{LWD} are maybe not really
\enquote{resumptives} -- but that the embedded clause in a \emph{dass}-\gls{LWD} is
full-fledged in the sense that there are no syntactic gaps -- but that all
positions are syntactically occupied by a co-referent pronoun -- and its
PF-realization is subject to non-syntactic conditions. The next set of data
shows this difference very clearly.

\subsection{Resumptive pronouns in LWDs across two clause
boundaries}\label{sub:36.3.3}

The acceptance of resumptives\is{resumptive pronouns} was also tested across two clause boundaries,
i.e.\ a situation where the occurrence/acceptance of resumptives\is{resumptive pronouns} can more
easily attributed to outer-syntactic (i.e.\ parsing) properties. The test
sentence is given in English wording in \eqref{ex:36.14}:

\ea\label{ex:36.14}
    Who did you say [ dass / wo Mary heard [dass/wo \underline{\hphantom{2em}} had an accident ]]
\z

\noindent We varied the \isi{complementizers} and resumptives\is{resumptive pronouns} as shown in \tabref{tab:36.2}.

\begin{table}
\begin{tabular}{lrr}
\lsptoprule
{Variation of comps} & {Acceptance (1--2)} & {Complete rejection (5)}\\\midrule
{dass\dots{}dass\dots{}gap} & {30\%} & {22\%}\\
{dass\dots{}dass\dots{}resumptive} & {70\%} & {5\%}\\
{wo\dots{}wo\dots{}gap} & {31\%} & {23\%}\\
{wo\dots{}wo\dots{}resumptive} & {8\%} & {45\%}\\
\lspbottomrule
\end{tabular}
\caption{Acceptance of gap/resumptive in subject position in \glspl{LWD} crossing two
clause boundaries ($n=580$).}\label{tab:36.2}
\end{table}

The results show clearly that the acceptance of resumptives\is{resumptive pronouns} is directly
connected to the type of the complementizer\is{complementizers}. Again: \emph{dass}-\glspl{LWD} nearly
obligatorily require an overt pronoun on the \enquote{extraction-site} (70\% with a
rejection rate of 5\%) whereas this is nearly impossible with subjects in
\emph{wo}-\glspl{LWD}. The results of this test sentence reproduces nicely a similar
result, asked in an earlier questionnaire. There, we didn’t head for \glspl{LWD} but
rather what is called \emph{long relativization}; the sentence is again given in
English wording:\largerpage

\ea
	 This is the man [dass/wo I know [dass/wo (he) lives in D.]]
\z

\noindent The results are presented in \tabref{tab:36.3}.

\begin{table}
\begin{tabular}{lrr}
\lsptoprule
{Variation of comps} & {Acceptance (1--2)} & {Complete rejection (5)}\\
\midrule
{wo\dots{}dass\dots{}gap} & {12\%} & {50\%}\\
{wo\dots{}dass\dots{}resumptive} & {87\%} & {3\%}\\
{wo\dots{}wo\dots{}gap} & {44\%} & {19\%}\\
{wo\dots{}wo\dots{}resumptive} & {5\%} & {61\%}\\
\lspbottomrule
\end{tabular}
\caption{Acceptance of gap/resumptive in long relativization (two clause
boundaries)}\label{tab:36.3}
\end{table}

The same template was used for long relativization of a dative\is{dative case} argument and
here, the acceptance of the resumptive in the \textit{wo\dots{}wo}-configuration showed
essentially the same result as with simple relativization, namely about 18\% --
where\-as the \emph{dass}-complement clause yielded a result of 83\% acceptance
for the dative\is{dative case} resumptive.

These results are more interesting than the ones from the simple \glspl{LWD} -- since
they show that the acceptance of a resumptive is not dependent on distance but
rather on the choice of the complementizer\is{complementizers}. Note that in \tabref{tab:36.2}, 
all variants with a gap reach a result of only 30\%. However, in the
case of a \emph{dass}-\gls{LWD}, the sentence can be saved by inserting the
resumptive (by a rejection rate of 5\%). This possibility is essentially
excluded for \emph{wo}-\glspl{LWD}.

\subsection{Resumptive pronouns in different shapes}\label{sub:36.3.4} %3.3. /

A final piece of evidence for the idea that the \enquote{extraction out of
    \emph{dass}-clauses} is maybe an illusion comes from the type of pronoun
    used as a resumptive. In these test-sentences, we didn’t offer the
    \enquote{usual resumptive pronoun}, namely the simple personal pronoun as
    the least marked ones available in \ili{Alemannic}, see \citet{Adger2011}
    for discussion, but a pronoun of the \emph{d}-series:

\ea
    simple pronouns: er -- (s)ie -- es; \emph{d}-series: d-er -- d-ie -- d-as
\z

The \emph{d}-series pronouns normally force a disjoint reference interpretation in a
binding configuration across a clause-boundary \parencite{Wiltschko1998}:

\ea\label{ex:36.17} \ili{German}\\
    \gll Hans\tss{i} glaubt, dass er\tss{i/j} / der\tss{*i/j} der Beste ist\\
    Hans believes that he {} \emph{d}-series the best (one) is\\
\z

Anecdotal observations about a much higher rate of d-pronouns in \ili{Alemannic} lead
us to the idea to test systematically the acceptance of these pronouns as
resumptives. And indeed, although the acceptance rate is by far lower than with
personal resumptives, it is remarkable that they show up to a much higher
degree in \emph{dass}-\glspl{LWD}, namely 35\% acceptance -- but only 15\%
with \emph{wo}-\gls{LWD}s,  This difference in acceptance co-varying with the
choice of the complementizer\is{complementizers} again hints at the conclusion that a
\emph{dass}-clause is more encapsulated with respect to its syntactic
surrounding as a \emph{wo}-clause, strengthening the idea that it is a
full-fledged clause -- even if construed with an \gls{LWD}.\footnote{Clearly,
    the impossibility of binding of the d-pronoun in \eqref{ex:36.17} must
then find a different interpretation, see \citet{vanKampen2012} for further
observations with respect to these pronouns -- where they can even act in some
cases as bound variables.}

\subsection{Resumptives in Celtic}\label{sub:36.3.5} %3.4. /

What I left out until now is a discussion of resumptives\is{resumptive pronouns} in the Celtic
languages. As discussed in McCloskey’s work, \ili{Irish} exhibits two types of RCI,
traditionally named aL and aN. While aL never allows resumptives\is{resumptive pronouns} in \glspl{RC},
aN requires them. A classical example is given below:\largerpage[2]

\ea\label{ex:36.18}Irish
    \ea
    \gll    an ghirseach a ghoid na síogaí\\
             the girl aL stole the fairies \underline{\hphantom{2em}}\\
    \ex
    \gll    an ghirseach a-r            ghoid na síogaí í\\
            the girl aN-\Pst{} stole the fairies her\\
    \z
\z

As can be seen, the \Rci{} requiring the resumptive has the tense morpheme
attached to it, indicating that it occupies a different, probably lower
position in the functional extension of the clause, i.e.\ closer to Tense, see
also \citet{Roberts2005} for such an assumption. Without committing myself to a
detailed account in terms of a split C-projection in a \citet{Rizzi1997}-style,
it is of course striking that aN shows the same behavior as the complementizer
\emph{go} -- which also combines with the tense morpheme, yielding these
different forms shown above (\emph{gu-r}, \emph{gu-n}, etc.\ depending on the
variant). Clearly, these pattern with the \emph{dass}-\glspl{LWD} in \ili{Alemannic} whereas
\emph{wo} in \ili{Alemannic} is the direct parallel to aL.

This would mean then that \ili{Alemannic} \emph{wo} and \ili{Irish} \emph{aL} are genuine
complementizers -- whereas \emph{dass/that} are indeed relative pronouns with
the head consisting of a possibly silent correlate pronoun, cf.\ the structure
given in (4$'$). This then implies that a complement clause introduced by
\emph{dass/that/go} is always an island\is{islands} and that the seemingly extraction is
not extraction at all. The data discussed here favor such an analysis.

The reason that there is no way in \ili{Celtic} to build a \gls{LWD} with a
\emph{go}-clause -- in contrast to an \ili{Alemannic} \emph{dass}-\gls{LWD} --
has probably to do with the fact that \emph{go} is originally a preposition
(see \citealt{Braesicke}; Elliot Lash, p.c.). As such, its \enquote{clausal
complement} has probably still a nominal core in it and is thus an island for
independent reasons.  Furthermore, \ili{Celtic} has to my knowledge never shown an
RC-formation strategy using pronouns. In contrast, in \ili{Germanic} (and also
\ili{Alemannic}) \glspl{RC} can be built with pronouns -- and indeed -- if not
used as an aboutness relative and thus a complement clause, as I suggested
above, cf.\ footnote \ref{fn:36:3}, it can occur with a clause-internal gap. Thus, this is
a pattern which is encountered in \ili{Germanic} -- but not in Celtic:

\ea\label{ex:36.19} \ili{German}\\
    \gll    das Buch, das   du  \underline{\hphantom{2em}}  gelesen hast, \dots{}\\
    the book that you  {}   read       have\\
    \glt    ‘The book that you’ve read, \dots{}’
\z

The exact details have to be worked out in future work -- but the difference in
building clausal complements and \isi{relative clauses} in \ili{Germanic} in \ili{Celtic} must be
the clue to understand the different behavior when it comes to \glspl{LWD}.
Alemannic is interesting as it has both strategies at its disposal for building
\glspl{RC} and \glspl{LWD} and the difference in behavior concerning
resumptives shows that there are deep syntactic differences between these
structures.

\section{Conclusion and outlook} %4. /

I started with taking seriously the doubts on \emph{dass} as having been
re-categorized to the word-class\is{lexical categories} of
complementizer\is{complementizers} (and with it its head-status, resp.
belonging to the extended projection of the verb). I asked which kind of
evidence could be relevant to show whether \emph{dass} is still what it looks
like, namely a \emph{d}-series pronoun, resp. a relative pronoun, implying that
the complement clause is essentially a relative clause,\is{relative clauses} as
assumed in \citet{Kayne2014}. The consequence of this view is that complement
clauses introduced by \emph{dass} should be opaque to extraction. And indeed, I
showed that the unexpected high acceptance rates of \isi{resumptive pronouns}
hint to the conclusion that all arguments in these embedded clauses are
syntactically present as pronouns in LWDS. However, they may be subject to a
rather \enquote{weak} principle like the \isi{avoid pronoun principle} in being merely not
pronounced if too close to the antecedent. This was contrasted with
constructions containing a genuine gap, coming into existence via a relative
clause \is{relative clauses}formation strategy involving a specialized
particle, requiring a gap in its clausal complement and thus
resumptives\is{resumptive pronouns} are
essentially not possible -- besides in those cases where they appear also in
\isi{relative clauses} -- for reasons that I did not discuss here.  If this is
on the right track, it may have far reaching consequences for a whole bunch of
assumptions about the cyclic nature of \isi{movement} (re-merge). What it essentially
means is that there is no cross-clausal \isi{movement} at all. In light of the idea
that re-merge should obey the extension condition in a strict way, this is a
welcome result -- since long cyclic-successive \isi{movement} is until now the
problematic exception to this condition.

The task for the future will then be to find more languages of the Alemannic
type to see whether the correlations outlined in \Cref{sub:36.3.4} hold as
well. The Scandinavian languages that allow \glspl{LWD} with \emph{som}
immediately come to mind. Another area of investigation would be the wh-in-situ
languages which have \glspl{LWD} but arguably no clause-internal wh-movement. A
base generation approach together with maybe different licensing conditions for
gaps/resumptives\is{resumptive pronouns} could shed new light on these long
standing issues in generative syntax.



\printchapterglossary{}

{\sloppy\printbibliography[heading=subbibliography,notkeyword=this]}

\end{document}
