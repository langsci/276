\documentclass[output=paper]{langsci/langscibook}

\author{Kleanthes K. Grohmann\affiliation{University of Cyprus, Cyprus
Acquisition Team}\and Evelina Leivada\affiliation{UiT The Arctic University of
Norway, Cyprus Acquisition Team}}

\title{Reconciling linguistic theories on comparative variation with an evolutionarily plausible language
faculty}

\rohead{\thechapter\hspace{0.5em}An evolutionarily plausible language faculty}

% \chapterDOI{} %will be filled in at production

\abstract{This work aims to reconcile the atomic objects of study typically
    assumed within comparative variation studies with an evolutionarily
    plausible faculty of language. In the process, we formulate and address the
    \emph{incompatibility problem}, the observation that studying comparative
    (micro)variation has progressively led to an evolutionarily implausible
    Universal Grammar. We identify a solution to this problem through arguing
    in favour of a so-called emergentist approach to some linguistic
    primitives. We then address the \emph{granularity mismatch problem} and
    argue on the basis of this emergentist approach firstly, that linguistic
    and neurocognitive studies of language may be brought to the same level of
    granularity, and secondly, that specific insights from comparative
    variation can inform an evolutionarily plausible approach to human
language.}

\maketitle

\begin{document}\glsresetall

\glsreset{UG}

\section{Introduction}

The topic of language variation and how it informs our study of the \gls{FL}
together with its initial state are currently at the forefront of linguistic
research (for latest overviews, see e.g. \citealt{Hinzen2014,Trettenbrein2015,BerwickChomsky2016}). 
As a matter of fact,
the exploration of variation from a comparative, cross-linguistic perspective
can be considered one of the very few topics which both linguists and cognitive
neuroscientists agree merits further attention.

A representative perspective of the first area of research is that of
generative linguist Noam Chomsky. When asked in a recent interview what the
main advantages and/or reasons to study linguistic variation are, he reiterated
a view that has been repeatedly explored in his work: In order to determine the
capacity to use and understand language, we need to know \enquote{what options
it permits} \citep{Chomsky2015b}. Put differently, if we want to understand
\gls{FL} and its initial state, \gls{UG}, we must determine what structures
\gls{UG} is capable of generating. In the same vein, we should also determine
what structures \gls{UG}\is{Universal Grammar} is \emph{not} capable of generating as striking
typological gaps across phylogenetically diverse languages call for
explanations that can enrich our theory of language (see
\citealt{BibHolRob2014} for a concrete example). From a linguistic perspective,
we will call this the \enquote{insider} view.

To pursue the analogy, the perspective of cognitive neuroscientist Peter
Hagoort can be described as the \enquote{outsider} view. Hagoort devoted part
of his plenary talk at the 47th annual meeting of the Linguistic Society of
Europe to how linguistics, once seen as a key player in the field of cognitive
science, has seen its influence fade over the years \citep{Hagoort2014}. This
alienation directly relates to how linguists have presented their discoveries
in the study of language variation. Often linguists have captured aspects of
comparative variation through postulating primitives that they did not grow or
derive in any sense, typically by assuming that a \gls{UG}-encoded feature
drives the relevant linguistic representation. Such postulations cannot be
informative in the long run. Perhaps they can be successfully employed when one
deals with some language A or B, but when the aim is broader (e.g., to approach
our language-readiness and \gls{UG}\is{Universal Grammar} as its initial
state), then such postulations are rather impeding progress.

In this context, the two most important questions to be addressed are (i) why
this alienation across disciplines is happening and (ii) whether there is a
remedy for this situation. The second question is the topic of \sectref{sec:25:2}. With
respect to the first question, it seems that the reason is in part the way the
topic of language variation has been approached over the last few years. More
specifically, discussing comparative syntax and the way parametric models
capture variation (see, for example, the recent collection of papers in
\citealt{FábregasEtAl2015}), \citet{BibHolRobShee2014} argue that linguistic
descriptions that have emerged since \citet{Chomsky1981} have achieved an
increasingly high level of descriptive adequacy, but sacrificed explanatory
adequacy due to the postulation of more and more entities in \gls{UG}. In their
words:

\begin{quote}

Arguably, the direction that [principles \& parameters] (P\&P) theory has taken
reflects the familiar tension between the exigencies of empirical description,
which lead us to postulate ever more entities, and the need for explanation,
which requires us to eliminate as many entities as possible. In other words,
parametric descriptions as they have emerged in much recent work tend to
sacrifice the explanatory power of \isi{parameters} of Universal Grammar in order to
achieve a high level of descriptive adequacy.
\parencite[104]{BibHolRobShee2014}

\end{quote}

Describing linguistic data and formulating observations or generalisations over
these data may then offer observational adequacy, possibly even descriptive
adequacy, but not explanatory adequacy.

Although \citeauthor{BibHolRobShee2014}'s point is well-taken, it is only a
part of the issue at hand. Another part is presented by \citet{Yang2004} when
he writes that

\begin{quote}

adult speakers, at the terminal state of language acquisition\is{language acquisition}, \emph{may retain
    multiple grammars, or more precisely, alternate parameter values}; these
    facts are fundamentally incompatible with the triggering model of
    acquisition\is{language acquisition} [\dots{}] \emph{It is often suggested that the individual
    variation is incompatible with the Chomskyan generative program}.
    \parencite[50--51]{Yang2004}

\end{quote}

We can thus phrase the full problem as follows:

\ea \emph{The incompatibility problem}:
Studying \isi{microvariation} has led to a model entailing an evolutionarily
    implausible \gls{UG}/\gls{FL}.
\z

Put differently, we have managed to describe many linguistic structures across
different languages, but now we have trouble explaining the ontology of the
biological \enquote{structure} underlying their existence: \gls{UG}.  Given the short
time scale typically assumed for evolution, the higher the degree of linguistic
specificity encoded in \gls{UG}, the more difficult the task of accounting for
it in evolutionary terms.

Reconciling a bottom-up approach to \gls{UG}\is{Universal Grammar} and a resulting evolutionarily
plausible \gls{FL} with the findings from the literature on language variation
has the potential to solve not only the incompatibility problem but also
\emph{Poeppel's problem}. More specifically, this reconciliation can overcome the
granularity mismatch considerations according to which linguistic and
neuroscientific studies of language operate with objects of different
granularity in a way that makes the construction of interdisciplinary bridges
particularly difficult (cf. the granularity mismatch problem in
\citealt{PoeEmb2005}). A bottom-up approach to \gls{UG}\is{Universal Grammar} entails a
non-overarticulated \gls{UG}\is{Universal Grammar} which consists of a few computational principles
(as \citealt{DiSciulloEtAl2010} have argued) only, leaving outside of this
component many of the linguistic primitives that have been ascribed to it
within comparative variation studies.

In this context, the next section discusses the importance of studying
variation from a comparative, cross-linguistic perspective while at the same
time maintaining a bottom-up approach to \gls{UG}\is{Universal Grammar} (i.e. an approach to \gls{UG}
from below that seeks to ascribe to it as little as possible, while maximizing
the role of the other two factors in language design; \citealt{Chomsky2007}).
Pursuing a bottom-up vs.\ a top-down approach matters because depending on how
much one ascribes to \gls{UG}, the plausibility of the latter from an
evolutionary perspective changes significantly. Our main aim is to offer the
following solution to the incompatibility problem: An emergentist approach to
some \gls{UG}\is{Universal Grammar} primitives can reconcile the Chomskyan generative program and the
individual variation attested in reality. \sectref{sec:25:3} then aims to
offer a concrete demonstration of how relevant findings and primitives from the
field of language variation can inform a biological approach to human language.
\sectref{sec:25:4} concludes and presents some suggestions for future work on
this topic.

\section{An emergentist approach to UG primitives}\label{sec:25:2}

The second question that arose in the context of Hagoort’s view on the
interaction of linguistics with the larger field of cognitive science is
whether there is a remedy for the observed decreased influence of linguistics.
\citet{Hagoort2014} offers five different directions for rectifying this issue.
We apply some of these directions through pursuing an approach to \gls{UG}
primitives from below \citep{Chomsky2007}, while at the same time retaining in
our theory of \gls{FL} some of the theoretical notions that pertain to the
comparative variation literature. This combination has the potential of killing
two birds with one stone, solving not only the incompatibility problem but also
doing justice to the patterns of (micro)variation that are attested across
languages in the following, two-step way:\largerpage

\begin{itemize}

    \item[I.] Disentangling variation by teasing apart the different
        contributing factors which are responsible for deriving it in a way
        that does justice to sociolinguistic and psycho-/neurolinguistic
        aspects of language use, such as mono- vs.\ bilingual acquisition\is{language acquisition}
        trajectories, the sociolinguistic status of the linguistic input, and
        the non-linguistic part of the environment.

    \item[II.] Keeping \gls{UG}\is{Universal Grammar} primitives to a minimum in order to effectively
        comply with both minimalist principles and evolutionary constraints.

\end{itemize}

Point (I) has a second part that will not be addressed in this paper but that
should be kept in mind nevertheless if the goal is to construct
interdisciplinary bridges and overcome the granularity mismatch problem:
Embedding the theory of language variation that emerges from step (I) into a
“shared context of justification” \citep{Hagoort2014} by obtaining reliable
data from different language groups, each of which may contribute its own
characteristics towards deriving variation.\footnote{\citet{Hagoort2014} argues
that running sentences in one’s head and consulting a colleague is fine for
discovering interesting phenomena and possible explanations (the \enquote{context of
discovery}), but it does not suffice as \enquote{the context of justification}, due to
innate confirmation biases and the fallibility of introspection. Thus, “to
justify one’s theory, empirical data have to be acquired and analysed according
to the quantitative standards of the other fields of cognitive science”. In the
context of addressing the incompatibility problem, Hagoort’s perspective is
relevant because it shows how findings that may target points of grammatical
(micro)variation should be analysed and interpreted.} In practice, this would
mean that careful elicitation of data should be followed by an attempt to
interpret the data through \emph{deriving} their properties rather than
assuming that they are driven by a new, ad hoc postulated feature. If
the aim is to understand \gls{FL} rather than describe structure A in language
B, then this process of interpretation should also be cautious to not rely on
assumptions that are hard to sustain in the long run and quickly decompose
under the light of interdisciplinary examination.

Talking about different contributing factors in (I) boils down to realising
that variation across developmental paths of individuals that speak the same
language can be the outcome of different modalities, environmental factors,
non-linguistic features that affect linguistic development, and so on. For
instance, research has shown that non-standard varieties allow for greater
grammatical fluidity in a way that blurs the boundaries across different
varieties. This, in turn, affects speakers’ perceptions of whether a specific
variant belongs to their linguistic repertoire or not
\parencite{CheSte1997,Henry2005}. Another contributing factor is the trajectory
of language acquisition\is{language acquisition} and subsequent development, and the circumstances in
which it takes place. For example, non-heritage speakers of a language may
differ from heritage speakers of the \emph{same} language with respect to the
amount of variation attested in their repertoire
(\citealt{Montrul2002,Montrul2008,LohWes2016}). The sociolinguistic status of
the language(s) one is exposed to (the mono- vs.\  bilingual trajectory is in
and of itself another factor that leads to variation) is yet another potential
source of variation: In the case of non-standard varieties, speakers’
perceptions about their native grammatical variants are likely to be affected
by their knowledge that many of their dialectal structures are considered
unacceptable or \enquote{incorrect} by speakers of the standard variety
(\citealt{Henry2005} for \ili{Belfast English}; \citealt{LeivadaEtAl2017a} for
\ili{Cypriot Greek}) in a way that enhances grammatical fluidity. Also, in those
cases in which a standard variety co-exists with a structurally proximal,
non-standard variety, the discreteness across grammatical variants at times
fades away by the emergence of intermediate \citep{Cornips2006} or \enquote{diaglossic}
speech repertoires \citep{Auer2005}, resulting once more in a greater degree of
variation (see also \citealt{RowGro2014} and relevant references cited for
\ili{Cypriot Greek}).

Understanding the multitude of faces that variation can acquire (for a more
extensive overview, see \citealt{Leivada2015a}) is of key importance when it
comes to approaching \gls{UG}\is{Universal Grammar} primitives from an emergentist perspective. The
reason is that cross-linguistic variation has long been described as part of
\gls{UG}, that is, deriving from \gls{UG}\is{Universal Grammar} \isi{parameters}. Showing that patterns of
variation are not as stabilised or uniform as the traditional \gls{UG}
parameters-account predicts opens the way for an emergentist approach to
linguistic primitives that were traditionally viewed as part of \gls{UG}.
Understanding what terms like \enquote{stabilised} or \enquote{uniform} refer to in the present
context requires shifting our attention to how variation \emph{within}
linguistic communities has been approached.

A crucial challenge for any approach to variation derives from the mainstream
conception of the notion of \enquote{surface variation} (i.e. grammatical variation
among speakers of the same language that is not the result of any acquired or
developmental pathology) \emph{within} a linguistic community. For example,
Chomsky’s idealised picture of a “completely homogeneous speech community” and
an “ideal speaker-listener […] who knows its language perfectly”
\citep[3]{Chomsky1965} is often assumed together with the assumption that the
so-called \enquote{linguistic genotype} is uniform across the species in the absence of
severe and specific pathology (\citealt{AndLigh2000}). Another
related idea is that attained adult performance is “essentially homogeneous
with that of the surrounding community”, unless again a pathology is present
\parencite[698]{AndLigh2000}. When translated into empirical terms,
idealisations like these, although theoretically well-argued in their original
context, paint a picture directly related to both Hagoort’s and Poeppel’s
considerations. More specifically, by not doing justice to the patterns of
surface variation that are attested in reality, theoretical linguistics may
\emph{lose} a significant part of its potential for interactions with fields
that deal with recent sign language emergence, evolutionary linguistics, or
sociolinguistics. Despite what the idealised picture suggests, variation can be
found even in the absence of any pathology, even among speakers of the same
language, and even within a native speaker who has passed the L1 acquisition\is{language acquisition}
period. The core of this idea can be analysed across two dimensions, the
linguistic dimension and the developmental one.

The developmental dimension refers to the fact that the presence of a severe
and specific pathology is not a necessary condition for obtaining variation,
even among neurotypical speakers of the same language. Individuals that share a
diagnosis of cognitive disorder (or the absence of one) are not necessarily
uniform in terms of their innate endowment: Individuals with a pathogenic
variant of a gene can be impaired in a non-uniform fashion (variable
expressivity), which may result in different cognitive phenotypes at times not
reaching a cut-off point where the diagnosis of a specific pathology is
possible. To demonstrate this with two examples, \citet{Fowler1995} observes
that there is tremendous variability with regard to language function in
individuals with Down syndrome (variable expressivity). And it has also been
observed that the existence of subsyndromal schizotypal traits in the general
population is higher than average in first-degree relatives of patients with
schizophrenia \citep{CalkinsEtAl2004}. This led to the realisation that

\begin{quote}

schizophrenia is not, despite its clinically important and reliable categorical
diagnosis [\dots], a binary phenotype (present, absent) with sudden disease
onset. \parencite[1]{EttingerEtAl2014}

\end{quote}

In other words, some pathological characteristics might be present even if the
cut-off point for reaching a diagnosis is not met -- and, on the other hand, a
diagnosis of schizophrenia might be reached, even if the pathological
characteristics manifested among individuals with the same diagnosis are far
from uniform.  Together, these two examples suggest that it is equally
plausible to expect that attained adult performance is not uniform among
members of the same linguistic community in the absence of a pathology or in
the presence of the same pathology.

With respect to the linguistic dimension, this is where factors related to
non-standard varieties and inherent grammatical fluidity enter the picture.
Evidently, not all linguistic communities are homogeneous, and in many cases
this variation goes well beyond bi- or multilingualism. Similarly, in the case
of recent language emergence de novo, as in the case of \gls{ABSL}  and other sign languages, fieldwork has shown that
not only is the development of grammatical markers subject to environmental
factors (e.g., time, distribution of speakers/signers, etc.), but also that
great grammatical fluidity is attested at the various stages in the development
of a language. In these recently emerged languages, points of variation
(\enquote{parameters} in generative terms) are \emph{not} fixed in terms of their
values, resulting in the realisation of alternate settings both within and
across speakers \parencite{Washabaugh1986,SandlerEtAl2011}.

To mention a concrete example, consider the head-directionality parameter.
S(ubject) O(bject) V(erb) is the prevalent word order among \gls{ABSL}\il{Al-Sayyid Bedouin Sign Language} signers; this
was, however, established as the prevalent order from the second generation of
signers onwards only \citep{SandlerEtAl2005}, meaning that for some time the
manifestations of this \enquote{parameter} were more fluid than what a stabilised
parameter value would permit. Even more important is the fact that variation
exists past the \enquote{stabilisation} point: \citet[2663]{SandlerEtAl2005} report the
existence of some (S)VO patterns. As \citet{Leivada2015a} argues in her
discussion of \gls{ABSL}\il{Al-Sayyid Bedouin Sign Language}, the fact that SOV
patterns became robust in the second generation of speakers illustrates that
variation is present when certain grammatical properties are still emerging.
Fluctuating parameter values within a syntactic environment are incompatible
with the idea that a parameter value is fixed past the terminal state of
acquisition. Observing that this fluctuation exists in various cases, be it
non-standard varieties or recently emerged grammars, is an indication that the
head-directionality parameter “should indeed be better viewed as a surfacey
decision that allows for varying realizations, rather than a fixed, deeply
rooted syntactic parameter” \citep[48]{Leivada2015a}. This does not mean that
points of variation are unfixed and eventually culminate in an “anything goes”
grammar, but it does mean that this surface decision is not (i) syntactic (i.e.
Chomsky in recent work has explicitly recognized that variation between
grammars is a matter of variable externalization; see
\citealt[41]{BerCho2011}), (ii) UG-encoded, or (iii) binary, as the classical
parametric approach would suggest. Non-binarity is particularly evident in case
of bidialectal speakers; their linguistic repertoire may include functionally
equivalent variants \citep{Kroch1994} with \emph{different} values that are
alternatively realized in the \emph{same} syntactic environment
\citep{LeivadaEtAl2017b}.

An emergentist approach to some linguistic primitives that were previously
thought to be parts of \gls{UG}\is{Universal Grammar} will be able to reconcile
the Chomskyan generative program (and especially \gls{UG}, as one of its main
pillars) with the patterns of variation that are attested in reality (see
\citeauthor{Yang2004}’s \citeyear{Yang2004} point mentioned earlier). Moreover,
an emergentist approach will solve the incompatibility problem, as the number
of linguistic primitives allocated to \gls{UG} will be reduced. The notion of
\emph{emergent parameters}
\parencite{RobHol2010,Roberts2012,BibRobShee2014,BibRob2017} is an important
step in this direction. The central idea behind emergent \isi{parameters} is
that instead of postulating a richly specified parametric endowment as part of
the initial state of our \gls{FL} (UG; \citealt{Chomsky1981}), \isi{parameters}
are derived (i.e. emergent) properties falling out of the interaction of
\citegen{Chomsky2005} three factors in language design
\citep{BibHolRobShee2014}. In the context of emergent \isi{parameters} in which
\gls{UG} does not provide a pre-specified \enquote{menu} of parametric choices,
\citet{BibRobShee2014} note that it is very important to provide independent
motivation for the plausibility of the \isi{parameters} that acquirers will
postulate as well as for the sequence in which each point of variation should
be considered. Here lies the solution to the incompatibility problem and a
first step towards approaching the granularity mismatch problem.

With respect to the incompatibility problem, if the points of variation that
are meaningful from a comparative (micro)variation perspective are treated as
emergent properties, they are no longer translated as innately specified
options. The consequence of this move is that \gls{UG}\is{Universal Grammar} would be considerably
deflated and much easier to discuss from an evolutionary perspective. As
\citet{Chomsky2007} has very convincingly argued, for any given component or
structure, the less attributed to structure-specific factors for determining
the development of an organism, the more feasible the study of its evolution,
hence the need for a bottom-up approach to \gls{UG}.

\largerpage
In relation to the granularity mismatch problem, the important component of the
\enquote{emergent parameters}-account lies in the element of
\emph{interaction}. As \citet{BibRobShee2014} explicitly claim, it is the
interaction of the second factor (linguistic input) and the third factor
(non-language-specific principles of cognition) plus the language-readiness
(provided by the first factor, \gls{UG}); that delivers emergent \isi{parameters}. To
illustrate this with an example, let’s return to the head-directionality
parameter, which makes reference to the position of a head in relation to its
dependents. Traditional accounts of grammar would describe Japanese as a
head-final and English as a head-initial language, with the difference between
the two explained in terms of the different value to which the
head-directionality parameter is set. The typological preference given to
harmonic orders (i.e. \emph{consistent} head-initial or head-final patterns
within a language; see \citealt{Hawkins2010}) might also be taken to suggest
that a UG-based head-directionality parameter is indeed operative and, once
set, its effects are diffused across different syntactic
environments.\footnote{A reviewer points out that this is not assumed within
    the emergentist approach just outlined. Indeed, it is not and we do not
    embrace this explanation either; we only point out that it is an
    alternative explanation, which, however, should not be preferred, since it
    does not accommodate the patterns of variation that are attested.}
    Alternatively, one could argue that the realisation of the head in relation
    to its dependents does not boil down to setting a \gls{UG}-based parameter.
    This latter approach should be preferred because it is compatible with the
    fact that variation \emph{can} be attested past the \enquote{setting} state in the
    repertoire of a neurotypical, adult speaker who has fully acquired her
    language (as suggested in the case of ABSL).  If one chooses to approach
    this parameter as an emergent parameter, the interaction of this
    grammatical choice with principles of general cognitive architecture
    becomes meaningful. For example, why are harmonic orders preferred if they
    are not \emph{imposed} by the setting of a pre-determined parameter? Of
    course, an emergent parameter would also need to be \enquote{set} in order
    to reflect the options that are permitted in the adult grammar, but
    crucially by not being encoded in \gls{UG}, its variable realizations
    within and across speakers of the same language (e.g., in the form of
    functionally equivalent variants; \citealt{LeivadaEtAl2017b}) would not be
    a problem for our theory of \gls{UG}\is{Universal Grammar} and/or \gls{FL}.

\citet{Roberts2016} suggests that these generalisation effects are related to
the computational conservatism of the learning device. This is formally
captured by his \emph{input generalisation}: “There is a preference for a
given feature of a functional head\is{functional items} F to generalise to other functional heads\is{functional items} G,
H \dots{}” (cf. \citealt[275]{Roberts2007})~-- that is, to “maximise available
features” (\citealt{BibRob2016b,Roberts2016}). This computational
conservativism is a third factor principle. If so, preference for harmonic
orders no longer amounts to a \gls{UG}-wired principle or parameter, but to the
way human memory or even learning more broadly works. It has been shown that
sequence edges are particularly salient positions and facilitate learning in a
way that gives rise to \emph{either} word-initial \emph{or} word-final
processes much more often than otherwise (see, for example,
\citealt{EndressEtAl2009} on the prevalence of prefixing and suffixing across
languages in comparison to the rarity of infixing). At the syntactic level,
\citet{Dryer1992} observes the following correlation with respect to
generalisation effects in relation to the position of the Head on the basis of
434 languages: OV languages are mostly postpositional and VO languages are
mostly prepositional. From Dryer’s dataset, \citet{Hawkins2010} calculated that
the vast majority of languages (93\%) are consistently OV-postpositional or
VO-prepositional. \citet{Hawkins2010} approaches harmonic word-orders in terms
of third factor demands, and, more specifically, a processing preference that
favours shorter processing domains. Evidently, the workings of comparative
(micro)variation which deal with headedness patterns across typologically
different languages can now be revisited and explained from a different
perspective. This perspective involves the \emph{interaction} of linguistic
patterns with the driving forces of general cognition in a way that addresses
Hagoort’s considerations. With respect to the “messy” patterns of variation
that just do not fit in the classical notion of a binary parameter, but that
are just as uncontroversially there, an emergentist approach has the potential
to cover these too. If \isi{parameters} are emergent and allow for non-binary
realizations, then the incompatibility that \citet{Yang2004} correctly observes
between these “messy” patterns and \gls{UG}\is{Universal Grammar} disappears.

Despite its theoretical and empirical benefits, this interaction may not solve
the \emph{granularity mismatch problem}. It may contribute to the
construction of interdisciplinary bridges in some respects, but still a good
portion of primitives may be left unmapped across disciplines. Put differently,
even if \isi{parameters} or other linguistic primitives are explained through an
emergentist approach, this would not entail that the granularity mismatch
problem has been solved. This could be due to the complicated nature of the
task at hand; as \citet[156--157]{Hornstein2009} argues, \blockquote{the right
    theory of grammar will be one that has (roughly) the empirical coverage of
    [government-and-binding theory], \emph{and} that \enquote{solves} Plato’s
    problem, Darwin’s problem, \emph{and} the granularity mismatch problem}
    (emphasis added).\footnote{According to \citet{Hornstein2009}, Darwin’s
    problem refers to “the logical problem of language evolution”, how language
emerged in the species (see also \citealt{BoeGro2007} on the relation between
Plato’s problem and Darwin’s problem).} In other words, given how polylithic
both the problem and its solutions are, there can be no a priori
guarantee of success.  Despite recognising this possibility, the next section
will follow \citegen{Hagoort2014} suggestion to maximise the interdisciplinary
contributions of linguistics within a larger cognitive (neuro)science
environment. We endeavour to approach a constraint, which in the linguistics
literature has been called \enquote{linguistic} or \enquote{syntactic} more
often than not, in neurocognitive terms.

\section{Levels of granularity: Anti-identity as a case study}\label{sec:25:3}

Anti-identity has received many distinct names in the linguistics literature;
consider, for example, the \emph{obligatory contour principle} in phonology
\citep{Odden1986}, \emph{identity avoidance} \parencite{vanRiemsdijk2008},
\emph{distinctness}\is{categorial distinctness} \citep{Richards2010}, \emph{X-within-X recursion}
\parencite{ArsHin2012}. This is also the basis for \emph{anti-locality}
relations in syntax (\citealt{Grohmann2003}, recently surveyed with additional
references in \citealt{Grohmann2011}). Regardless of the level of linguistic
analysis at stake, anti-identity in general describes the absence of adjacent
elements of the same category (e.g., [*XX] in syntax).

There are different ways to approach this phenomenon. In the linguistics
literature, it has been approached in terms of a UG-imposed well-formedness ban
that precludes the adjacency of same-category elements (see
\citealt{Richards2010} for a more detailed discussion). This position would
place the ban in \gls{UG}, together with the configurations of categorial
features that the ban is sensitive to. Alternatively, one could aim to keep
\gls{UG} at a minimum and see whether [*XX] can be shown to boil down to a
general, cognitive principle. A first step in this direction is made by
\citet{vanRiemsdijk2008} when he briefly argues that identity avoidance might
be \enquote{a general principle of biological organization} (p.\ 242). If so,
one expects to find its manifestations not only in language, but also in other
domains of cognition.

Taking one step back, if this comparison across cognitive domains is fruitful,
one would have successfully mapped an element that appears in the
\enquote{parts list} (i.e.\ a list that enumerates concepts canonically used in
the fields of study it represents; see \citealt{PoeEmb2005}) of two different
disciplines.  In more recent work, \citet{Poeppel2012} talks about the
\emph{mapping problem}. In his words, the mapping problem “addresses the
relation between the primitives of cognition (here speech, language) and
neurobiology.  Dealing with this mapping problem invites the development of
linking hypotheses between the domains” \citep[34]{Poeppel2012}. Developing
these linking hypotheses is the only route to potentially solving the
granularity mismatch problem. Returning now to the case at hand, linking
hypotheses \emph{can} be constructed for [*XX].

It seems to be true that humans do not like repetitions in general and that
anti-identity in language is not the result of a linguistic ban but of a bias
that finds application in other domains of human cognition too.
\citegen{Walter2007} biomechanical repetition avoidance hypothesis proposes a
\emph{physiological} motivation for this dislike: Repetition of articulatory
gestures is relatively difficult, and this difficulty results in phonetic
variation; that is, in [XX] it is likely that the two elements are not spelled
out identically. We propose the term \enquote{novel information bias}, which has a
\emph{cognitive} motivation: It refers to the well-demonstrated fact that
subjects are unable to tokenise multiple adjacent instances of the same type
(\citealt{TreKan1998}, \citealt{Walter2007}) because of a general bias in the
perceptual system to be more attentive to novel sensory information than to
repeated information \citep{Leivada2017}.

In the body of research by \citeauthor{Kanwisher1987} (\citeyear{Kanwisher1987}
et seq.), \emph{repetition blindness} has been described as the
result of difficulties in detecting repeated tokens in rapid serial visual
presentations of words.  Another illustration is the \emph{apparent motion
illusion}: Identical stimuli flashed in different locations are largely
perceived as a single moving stimulus; in other words, subjects show a clear
preference for a representation of different tokens as one moving token
\citep{VetterEtAl2012}. What this means in the context of [*XX] is that talking
about a general cognitive bias on anti-identity instead of a UG-wired
linguistic constraint that bans [*XX] explains why a limited number of [XX]
patterns do surface cross-linguistically (as shown in \citealt{Leivada2015b}).
In sum, the strong preference for anti-identity in language has to do with the
way our brain computes types and tokens, and not with a syntactic ban on
same-category embedding.

Overall, this approach to anti-identity can be extended to other \gls{UG}
primitives such as \isi{parameters} or categorial features. In line with
\citegen{PoeEmb2005} suggestion to \enquote{tak[e] linguistic categories
    seriously and us[e] them to investigate how the brain computes with such
abstract categorical representations} (p.\ 107), this approach can lead to an
evolutionarily plausible \gls{UG}, while at the same time describing and
accounting for the patterns of variation that one has to deal with in the field
of comparative variation.

\section{Outlook}\label{sec:25:4}

The approach to \gls{UG}\is{Universal Grammar} primitives advocated in this work is still in its
earliest stages. An important thing to keep in mind for future work is that
deflating \gls{UG}\is{Universal Grammar} does not equal arguing against its existence. In other
words, there can be a noticeable change in the way we treat \gls{UG}
primitives, without denying the existence of \gls{UG}\is{Universal Grammar} (for further discussion,
see \citealt{Roberts2016b} and many of the contributions to that volume). The
second important note is that achieving the right levels of abstraction and
representation in this effort is crucial: The more linguists abstain from
postulating UG-encoded primitives that are very language-specific in nature,
the more progress will be made in embedding findings from linguistics in a
productively shared context of justification. Last, a third part of this type
of approach that is worth mentioning is the conclusion reached in
\citet{BibRobShee2014}: What were previously thought to be hard-wired
properties of \gls{FL} could actually reduce to emergent properties that
feature the element of interaction among the different factors in language
design.



\printchapterglossary{}

\section*{Acknowledgements}

We thank two anonymous reviewers for their helpful comments. KKG’s contribution
was partially supported by Leventis project 3411-61041 (University of Cyprus).
EL acknowledges support from European Union’s Horizon 2020 research and
innovation programme under the Marie Skłodowska-Curie grant agreement no.
746652.

{\sloppy
\printbibliography[heading=subbibliography,notkeyword=this]
}

\end{document}
