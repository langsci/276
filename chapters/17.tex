\documentclass[output=paper]{langsci/langscibook}
\author{Liliane Haegeman\affiliation{Ghent University}}
\title{Rethinking passives: The canonical \textsc{goal} passive in Dutch and 
its dialects}

% \chapterDOI{} %will be filled in at production

\abstract{The main goal of this paper is empirical: it challenges the claim
    repeatedly found in the current generative literature
    \parencite{Aleetal2014,BroekhuisCornips2004,BroekhuisCornips2012} that
    \ili{Dutch} lacks the \textsc{goal} \isi{passive}. As will be shown, among
    other things, these claims fail to take into account the \isi{microvariation}
    already reported in the earlier generative literature.

    The paper contains a detailed discussion of the properties of \textsc{goal}
    \isi{passive} in West Flemish, showing that, based on the standard
    diagnostics, the \textsc{goal} argument has acquired subject status in the
    \isi{passive}. This conclusion thus provides a challenge for those accounts
    of \ili{Germanic} passivization which are crucially based on the claim that
    \ili{English} is the only West \ili{Germanic} languages with a canonical
    \textsc{goal} \isi{passive} (cf.\ \citealt{SteinTripsIngham2016}).}

\maketitle

\rohead{\thechapter\hspace{0.5em}Rethinking passives}

\begin{document}\glsresetall

\section{The typology of double object patterns}\label{sec:04.1}

The cross-linguistic variation in passivization of double object patterns has
recently been the source of renewed interest.\is{double object construction} It
is sometimes claimed (most recently in \citealt{SteinTripsIngham2016}) that
\ili{English} is the only West \ili{Germanic} language allowing for the passivization
of the indirect object, illustrated in \REF{ex:04.1}. The \isi{passive}
form in (\ref{ex:04.1}b) is variously referred to as the indirect object
\isi{passive}, the \textsc{goal} passive
\parencite{HaddicanHolmberg2012,HaddicanHolmberg2015}
or the \textsc{recipient} passive (\citealt{SteinTripsIngham2016}). I will use
the label \textsc{goal} passive for convenience sake, as this term allows me to
use the same term to refer to the constituent which functions as the indirect
object in the active sentence and to the constituent that becomes the subject
in the passive sentence.\footnote{I leave aside \enquote{non-canonical}
passives such as the English \emph{get} \isi{passive} and the German/Dutch non-
canonical \emph{kriegen/krijgen} (‘get’) passives (\citealt{AleScha2013}).}
\citeauthor{SteinTripsIngham2016} claim: “the recipient \isi{passive} arose in
English but \emph{not in other West \ili{Germanic} languages}”
(\citeyear{SteinTripsIngham2016}: slide 3, my italics). \ili{German} has been
reported not to have a canonical \textsc{goal} passive \REF{ex:04.2}
(\citet[70]{Anagnostopoulou2003}; \citet[9]{AleScha2013};
\citet[10]{Aleetal2014} for recent discussions). The claim that, like
\ili{German}, \ili{Dutch} lacks a canonical \textsc{goal} passive, as shown in
\REF{ex:04.3}, is also common in the literature, as in, for instance,
\textcite{BroekhuisCornips2004,BroekhuisCornips2012};
\textcite{Broekhuisetal2015}; \citet[8]{AleScha2013};
\textcite[10]{Aleetal2014}.

\ea\label{ex:04.1} \textcite{HaddicanHolmberg2012,HaddicanHolmberg2015}
    \judgewidth{\%}
    \ea[]{They gave the girl the ball.}
    \ex[]{\textit{The girl} was given the ball.}
    \ex[\%]{\textit{The ball} was given the girl.}
    \z
\ex\label{ex:04.2} \ili{German} \parencite[10]{Aleetal2014}
    \ea[]{
    \gll Sie hat     dem Mann   die Blumen   geschenkt.\\
            she has     the-\Dat{} man   the flowers   given\\
    \glt ‘She has given the man the flowers.’}
    \ex[*]{\textit{Er}   wurde   die Blumen     geschenkt.\\
    he.\Nom{} was   the.\Acc{} flowers  given\\
    ‘He was given the flowers.’}
    \ex[*]{\textit{Die} \textit{Blumen}     wurden   dem Mann   geschenkt.\\
        the.\Nom{} flowers  were     the.\Dat{} Mann  given\\
    ‘The flowers were given to the man.’}
    \z
\ex\label{ex:04.3} \ili{Dutch} \parencite[10]{Aleetal2014}
    \ea[]{
    \gll Ik heb   hem   het eten   bezorgd.\\
        I have   him   the food   delivered\\
    \glt    ‘I delivered the food to him.’}
    \ex[*]{
        \gll \textit{Hij}  werd het eten bezorgd (door mij).\\
        he   was   the food   delivered   (by me)\\
    \glt ‘He was delivered the food by me.’}
    \ex[]{
        \gll \textit{Het} \textit{eten}   werd   hem   bezorgd (door mij).\\
    the food   was   him   delivered   (by me)\\
    \glt ‘The food was delivered to him by me.’}
    \z
\z

The goal of this paper is essentially empirical: it challenges the claim that
English is the only West \ili{Germanic} languages with a \textsc{goal} \isi{passive}, and
it challenges the specific claims made in the generative literature
\parencite{BroekhuisCornips2004,BroekhuisCornips2012} that \ili{Dutch} lacks the
\textsc{goal} \isi{passive}. As I will show, among other things, such claims fail to
take into account the \isi{microvariation} reported in the earlier literature. The
paper contains a detailed discussion of the \textsc{goal} \isi{passive} in West
Flemish.

\section{The IO passive in West Flemish}\label{sec:04.2}

\subsection{The data: overview}\label{sec:04.2.1} %2.1. /
\glsunset{WF}

As shown by the examples in \REF{ex:04.4} and \REF{ex:04.5},
\glsdesc{WF}\il{West Flemish} (from now on \gls{WF}), a dialect of \ili{Dutch}
and a West \ili{Germanic} language, does have a \textsc{goal} \isi{passive}: the
definite \textsc{goal}, \emph{Valère} in active (\ref{ex:04.4}a), has been
promoted to become the subject of the \isi{passive} sentence (\ref{ex:04.4}b).
Similarly, the indefinite \textsc{goal} \emph{nen student} (‘a student’) in
active (\ref{ex:04.5}a) has been promoted to subject status in the \isi{passive}
(\ref{ex:04.5}b). The discussion in this section is based on my own dialect
intuitions; the core intuitions are corroborated in
\citet{Dhaenens2014}.\footnote{A reviewer for this volume asks whether there
    are \isi{animacy} effects for the double object pattern with verbs of motion,
like those discussed by \citet{Haddican2010}. At first sight the effect is
replicated in \gls{WF}, but this issue needs further research.}

%    For the \ili{English} verb
%    \emph{send}, \citet[2430]{Haddican2010} discusses the following contrast in
%    terms of \textsc{goal} animacy:
%
%    \begin{exe}
%        \exi{(i)}
%        \begin{xlist}
%            \ex[]{I sent Maria the letter.}
%            \ex[*]{I sent France the letter.}
%        \end{xlist}
%    \end{exe}
%
%    This specific animacy effect in (i) is replicated for \gls{WF}\il{West
%    Flemish} \emph{stieren} (‘send’):
%
%    \begin{exe}
%        \exi{(ii)} \gls{WF}\il{West Flemish}
%        \begin{xlist}
%            \ex[]{\gll K’een   Maria   dienen   brief gestierd.\\
%                {I have}   Maria   that   letter   sent\\
%            \glt ‘I have sent Maria that letter.’}
%            \ex[*]{\gll K’een Vrankrijk   dienen   brief   gestierd\\
%            {I have}   France    that   letter   sent\\}
%        \end{xlist}
%    \end{exe}
%
%    Other cases too show an animacy effect, as shown in (iii). The expression
%    \emph{de moate pakken} (‘take the measure’, ‘measure up’) allows for a
%    double object pattern with animate \textsc{goals}, though not with
%    inanimate \textsc{goals}. A \textsc{goal} \isi{passive} is available for (iiia),
%    as shown in (iiic).
%
%    \begin{exe}
%        \exi{(iii)}\gls{WF}\il{West Flemish}
%        \begin{xlist}
%        \ex[]{
%            \gll K’een   Valère   de moate gepakt.\\
%            {I have}   Valère   the measure   taken\\
%            \glt ‘I took Valère’s measurements.’}
%        \ex[*]{
%            \gll K’een   die deure de moate   gepakt.\\
%            {I have}   that door   the measure   taken\\
%            \glt}
%        \ex[]{
%            \gll Valère   is gisteren     de moate   gepakt.\\
%            Valère is   yesterday   the size     taken\\
%            \glt ‘Valère’s measurements were taken yesterday.’}
%        \end{xlist}
%    \end{exe}
%
%    Interestingly, with ditransitive \emph{geven} (‘give’) what looks like a
%    double object pattern is at first sight also available with an inanimate
%    \textsc{goal}.
%
%    \begin{exe}
%        \exi{(iv)} \gls{WF}\il{West Flemish}
%        \begin{xlist}
%            \ex
%            \gll K goan   Maria dienen boek geven.\\
%            I go   Maria that   book give\\
%            \glt ‘I will give Maria that book.’
%            \ex
%            \gll K goan   de deure een tweede loage geven.\\
%                I go   the door   a second coat give\\
%            \glt ‘I’ll give the door a second coat of paint.’
%        \end{xlist}
%    \end{exe}
%
%    However, in \gls{WF}\il{West Flemish} this particular verb does not allow a
%    \textsc{goal} \isi{passive}. In \citet{Haegeman1986b} I argue that the
%    unavailability of a \textsc{goal} \isi{passive} with \emph{geven} (‘give’) is due
%    to the presence of an underlying null preposition associated with the
%    indirect object, though the analysis presented there obviously requires
%    updating.
%
%    \begin{exe}
%        \exi{(v)} \gls{WF}\il{West Flemish}
%        \begin{xlist}
%            \ex[*]{
%            \gll dat   Maria   dienen boek gegeven is\\
%                    that Maria that   book given   is\\}
%            \ex[*]{
%            \gll dat die deure een tweede loage gegeven is\\
%            that that door a   second coat given   is\\}
%        \end{xlist}
%    \end{exe}
%
%    To the best of my knowledge, animacy is not restrictive with respect to the
%    \textsc{theme}:
%
%    \begin{exe}
%        \exi{(vi)} \gls{WF}\il{West Flemish}
%        \begin{xlist}
%        \ex
%            \gll dan ze   de studenten dienen professor/ dienen cursus angeroaden een\\
%            that   they the students that professor/  that course recommended have\\
%            \glt ‘that they recommended that professor/that course to the students’
%        \ex
%            \gll dan de studenten dienen professor/ dienen cursus  angeroaden wieren\\
%            that   the students   that professor/ that course recommended were\\
%            \glt ‘that the students were recommended that professor/that course’
%        \end{xlist}
%    \end{exe}

\ea\label{ex:04.4}\glsdesc{WF}\il{West Flemish}
    \ea
    \gll dan   ze   \textit{Valère}   die posten   beloofd   een\\
        that.\Pl{} they   Valère     those jobs   promised   have\\
    \glt ‘that they promised Valère those jobs’
    \ex
    \gll da   \textit{Valère}   die posten   beloofd   wierd / is\\
        that   Valère     those jobs   promised   ‘became’ {} is\\
    \glt ‘that Valère was promised those jobs’
    \z
\ex\label{ex:04.5}\glsdesc{WF}\il{West Flemish}
    \ea
    \gll dan   ze   \textit{nen} \textit{student}  die posten   beloofd   een\\
        that.\Pl{} they   a student   those jobs   promised   have\\
    \glt ‘that they promised a student those jobs’
    \ex
    \gll dat   *(der)   \textit{nen} \textit{student}   die posten   beloofd   wierd / is\\
    that   \hphantom{*(}ter   a student   those jobs   promised   ‘became’ {} is\\
    \glt ‘that a student was promised those jobs’
    \z
\z

Observe that the obligatory presence of expletive\is{expletives} \emph{(d)er} (‘there’) in
(\ref{ex:04.5}b) is not a property specific to the \textsc{goal} \isi{passive}. The
obligatory presence of \emph{(d)er} is fully in line with the patterns found
elsewhere in (W)F: an indefinite or a quantified subject systematically
requires that the sentence appear in the existential pattern with
\emph{(d)er-}insertion, as exemplified in active monotransitive (\ref{ex:04.6}a)
or in \isi{passive} monotransitive (\ref{ex:04.6}b).

\ea\label{ex:04.6} \glsdesc{WF}\il{West Flemish}
    \ea
\gll   dan   *(der)     \textit{drie} \textit{studenten}   dienen boek   gelezen   een\\
		    that.\Pl{} \hphantom{*(}there   three   students   that book   read     have\\
	\glt     ‘that three students have read that book’
    \ex
\gll   dan   *(der)     \textit{drie} \textit{studenten}   betrapt     zyn\\
		    that.\Pl{} \hphantom{*(}there   three   students   caught     are\\
	\glt     ‘that three students were caught’
    \z
\z

\Cref{sec:04.2.2} provides arguments to the effect that in
\gls{WF}\il{West Flemish} \textsc{goal} passives, the \textsc{goal} argument is
promoted to subject status. \Cref{sec:04.2.3} shows that
\gls{WF}\il{West Flemish} \textsc{goal} passives also comply with two specific
diagnostics for Dutch passivization set out in
\textcite{BroekhuisCornips2004,BroekhuisCornips2012}, in particular with
respect to the presence of an \textsc{agent} and the eventive interpretation.

\subsection{Subject diagnostics for the \textsc{goal}
passive}\label{sec:04.2.2}

In the \gls{WF}\il{West Flemish} \textsc{goal} passives (\ref{ex:04.4}b) and
(\ref{ex:04.5}b), the promoted \textsc{goal} acquires the syntactic properties
of the \gls{WF}\il{West Flemish} subject, both when definite (\ref{ex:04.4}b)
and when indefinite (\ref{ex:04.5}b) (for early diagnostics, cf.\
\citealt{Haegeman1986a,Haegeman1986b}).

\subsubsection{Agreement}\label{ssub:agreement}

In the \textsc{goal} \isi{passive}, the \textsc{goal} DP agrees for person and
number with the finite verb and (in the relevant contexts) with the
complementizer (\ref{ex:04.7}--\ref{ex:04.8}). (\ref{ex:04.7}a)
illustrates a \isi{passive} with a definite \textsc{goal}: the finite
auxiliaries \emph{wierden}/\emph{woaren} (‘were’) are plural, as is the
complementizer \emph{dan} (‘that’), and they thus can be seen to agree with the
plural DP \emph{de studenten} (‘the students’). Neither complementizer\is{complementizers} nor
auxiliary can be singular (\ref{ex:04.7}b--d). In (\ref{ex:04.8}a) agreement is
triggered by the plural indefinite \emph{drie studenten} (‘three students’).
Again the agreement is mandatory (\ref{ex:04.8}b--d). The patterns in \REF{ex:04.7} and
\REF{ex:04.8} also entail that, in the \isi{passive} sentences, singular
agreement with the \textsc{theme} \emph{dienen bureau} (‘that office’) would be
ungrammatical, cf.\ (\ref{ex:04.7}d) and (\ref{ex:04.8}d).

\ea\label{ex:04.7} \glsdesc{WF}\il{West Flemish}
    \ea[]{
	\gll   dan     \textit{de} \textit{studenten}   dienen   bureau beloofd   {wierden / woaren}\\
		  that.\Pl{}  the students   that   office  promised   were-\Pl{}\\
    \glt   ‘that the students were promised that office’}
    \ex[*]{
	\gll  dat     \textit{de} \textit{studenten}   dienen bureau beloofd   {wierden / woaren}\\
		  that.\Sg{}  the students   that office  promised   were-\Pl{}\\
    \glt}
    \ex[*]{
	\gll   dan     \textit{de} \textit{studenten}   dienen bureau beloofd   {wierd / was}\\
		  that.\Pl{}  the students   that office  promised   was-\Sg{}\\
    \glt}
    \ex[*]{
	\gll   dat     \textit{de} \textit{studenten}   dienen bureau beloofd   {wierd / was}\\
		  that.\Sg{}  the students   that office  promised   was-\Sg{}\\
    \glt}
    \z
\ex\label{ex:04.8} \glsdesc{WF}\il{West Flemish}
    \ea[]{
	\gll   dan   *(der)    \textit{drie} \textit{studenten} dienen bureau beloof   {wierden / woaren}\\
		  that.\Pl{} \hphantom{*(}there  three students   that office  promised were-\Pl{}\\
    \glt   ‘that three students were promised that office’}
    \ex[*]{
	\gll  dat    *(der)    \textit{drie} \textit{studenten} dienen bureau beloofd    {wierden / woaren}\\
		  that.\Sg{} \hphantom{*(}there  three students   that office  promised  were-\Pl{}\\
    \glt}
    \ex[*]{
    \gll dan   *(der)    \textit{drie} \textit{studenten} dienen bureau beloofd    {wierd / was}\\
		  that.\Pl{} \hphantom{*(}there  three students   that office  promised  were-\Sg{}\\
      \glt}
    \ex[*]{
	\gll  dat   *(der)    \textit{drie} \textit{studenten} dienen bureau beloofd    {wierd / was}\\
		  that.\Sg{} \hphantom{*(}there  three students   that office  promised  were-\Pl{}\\
    \glt}
    \z
\z

\subsubsection{Case}

When pronominal, the \textsc{goal} DP is realised as a nominative\is{nominative
case}, and, like other nominative\is{nominative case} pronouns, it allows for
pronoun doubling. In (\ref{ex:04.9}a) the strong nominative\is{nominative
case} pronoun \emph{zie} is a doubler for the weak form \emph{ze}. For full
discussion of \gls{WF}\il{West Flemish} subject pronouns I refer to my earlier
work \parencite{Haegeman1991,Haegeman1992,Haegeman2004}. In the Flemish
regiolect, the subject of the \textsc{goal} \isi{passive} can be the impersonal
pronoun \emph{men} (‘one’), which is restricted to subject position of a finite
clause (\ref{ex:04.9}b).\footnote{This property cannot be tested for the
dialect because the impersonal pronoun \emph{men} is not used.}\largerpage[-2]

\ea\label{ex:04.9}\glsdesc{WF}\il{West Flemish}
    \ea
    \gll   da   \textit{ze} \textit{(zie)}   die posten     beloofd   wierd\\
		    that   she (she)   those positions   promised   was\\
	\glt     ‘that she was promised these jobs’
    \ex
	\gll   Het   komt   veel   voor  dat   \textit{men}   die   behandeling afgeraden     wordt.\\
		it   comes often   for   that   one   that   treatment disrecommended   is\\
	\glt ‘It is quite common that one is advised against that treatment.’
    \z
\z

\subsubsection{Relativization}

Like canonical definite subjects, relativized\is{relative clauses}
\textsc{goal} DPs are associated with relativizer\is{relative clauses}
\emph{die} (\ref{ex:04.10}a) and with \emph{dat/die} alternations
(\ref{ex:04.10}b). These properties are characteristic of subject relativization
in \gls{WF}\il{West Flemish} (\ref{ex:04.10}b), and they are unavailable in
object relativization (\ref{ex:04.10}c). See
\textcite{Haegeman1984,Haegeman1992}.

\ea\label{ex:04.10}\glsdesc{WF}\il{West Flemish}
    \ea
	\gll   Dat   zijn   de studenten   \textit{dien}   die posten   beloofd   woaren.\\
		    that   are   the students   \emph{die}{}-\Pl{}   those jobs   promised   were\\
	\glt     ‘Those are the students that were promised those jobs.’
    \ex
	\gll   Dat   zijn   de   studenten   dan-k   peinzen \textit{dien}   die   posten   beloofd woaren.\\
		that   are   the   students   that-I   think \emph{die-}\Pl{}   those   jobs   promised were\\
	\glt     ‘Those are the students that I think were promised those jobs.’
    \ex
	\gll   Dat zijn   de boeken   dan-k   peinzen da / *die   Valère     besteld   eet.\\
    that are   the books   that-I   think that {/} \hphantom{*}\emph{die}   Valère   ordered   has\\
	\glt     ‘Those are the books that I think that Valère has ordered.’
    \z
\z

\subsubsection{Existential patterns}\largerpage[1]

When the \textsc{goal} is an indefinite nominal (\ref{ex:04.5}b), a numeral
(\ref{ex:04.11}a) or a \emph{wh}{}-constituent (\ref{ex:04.11}b), and is promoted
to becoming the subject of the \isi{passive},  \emph{(d)er}{}-insertion is
obligatory.

\ea\label{ex:04.11}\glsdesc{WF}\il{West Flemish}
    \ea
	\gll   dan *(der) $\varnothing$ / \textit{drie} \textit{studenten}   dienen post   beloofd zyn\\
    that \hphantom{*}\emph{ter} {} {} three students   that job   promised are\\
	\glt     ‘that (three) students were promised that job’
    \ex
	\gll   Kweeten   niet   \textit{wien}   dat *(er)   dienen post   beloofd   is.\\
    {I know}  not   who   that \hphantom{*(}there   that job   promised   is\\
	\glt     ‘I don’t know who was promised that job.’
    \z
\z

Obligatory \emph{(d)er-}insertion is associated with indefinite or quantified
subjects and not with objects.

\subsubsection{Distribution}\largerpage[-1]

Like canonical definite subjects, the definite \textsc{goal} DP in the
\textsc{goal} \isi{passive} has to be linearly adjacent to the complementizer\is{complementizers}
\emph{dat} (‘that’)\footnote{In \gls{WF}\il{West Flemish} the complementizer\is{complementizers}
    \emph{dat} is obligatorily present in all embedded clause, frequently
leading to doubly filled Comp positions.} in embedded clauses
\REF{ex:04.12} and to the finite verb in root clauses \REF{ex:04.13}.
In (\ref{ex:04.12}a), adjuncts such as \emph{gisteren} (‘yesterday’) or
\emph{verzekerst} (‘probably’) cannot intervene between the complementizer\is{complementizers}
\emph{dat} (‘that’) and the \textsc{goal} \emph{Valère}. In
(\ref{ex:04.12}b), the \textsc{theme} \emph{die posten} (‘those jobs’)
cannot intervene between the complementizer\is{complementizers} \emph{dat} (‘that’) and the
\textsc{goal} \emph{Valère.} In \REF{ex:04.13}, the same adjacency
requirement is illustrated for root clauses in which the finite verb, here the
auxiliary \emph{wierd} (‘was’), has moved to C.  \REF{ex:04.14} and
\REF{ex:04.15} show that identical adjacency restrictions apply to definite
subjects of transitive sentences.

\ea\label{ex:04.12}\glsdesc{WF}\il{West Flemish}
    \ea[]{
	\gll    dat (*gisteren / verzekerst) \textit{Valère}   die posten   beloofd   wierd\\
        that \hphantom{(*}yesterday {} probably  Valère     those jobs   promised   was\\
    \glt    ‘that Valère was (probably) promised those jobs (yesterday).’}
    \ex[*]{
	\gll    dat   die posten   \textit{Valère}   beloofd   wierd\\
		    that   those jobs   Valère     promised   was\\
    \glt}
    \z
\ex\label{ex:04.13}\glsdesc{WF}\il{West Flemish}
    \ea[]{
	\gll   Daarom     wierd  (*gisteren / verzekerst) \textit{Valère}   die posten   beloofd.\\
    {for that reason}   is \hphantom{(*}yesterday {} probably Valère   those jobs   promised\\
	\glt     ‘For that reason, Valère was (probably) promised those jobs (yesterday).’}
    \ex[*]{
	\gll   Daarom     wierd   die   posten   \textit{Valère}   beloofd.\\
    {for that reason}   was   those   jobs   Valère     promised\\
       \glt}
    \z
\ex\label{ex:04.14}\glsdesc{WF}\il{West Flemish}
    \ea[]{
	\gll   dat   (*gisteren / verzekerst)   \textit{Valère} die posten   beloofd eet\\
		    that   \hphantom{(*}yesterday {} probably     Valère those jobs   promised has\\
    \glt     ‘that (probably) Valère promised those jobs (yesterday).’}
    \ex[*]{
	\gll   dat   die posten   \textit{Valère}   beloofd   eet\\
		    that   those jobs   Valère     promised   has\\
    \glt}
    \z
\ex%15
    \label{ex:04.15}\glsdesc{WF}\il{West Flemish}
    \ea[]{
	\gll   Daarom     eet   (*gisteren / verzekerst) \textit{Valère}   die posten   beloofd.\\
		    {for that reason}   has   \hphantom{(*}yesterday {} probably Valère     those jobs   promised\\
	\glt     ‘For that reason, Valère probably promised those jobs (yesterday).’}
    \ex[*]{
	\gll   Daarom   eet   die posten   \textit{Valère}   beloofd.\\
		    {for that reason} has   those jobs   Valère     promised\\
    \glt}
    \z
\z

\subsubsection{Non-finite clauses}

The \textsc{goal} \isi{passive} is available in non-finite \isi{control}
clauses, in which case the \textsc{goal} will be a controlled PRO
(\ref{ex:04.16}a). The goal subject of a \isi{passive} clause may undergo
raising in \emph{te} infinitives (\ref{ex:04.16}b).

\ea%16
    \label{ex:04.16}\glsdesc{WF}\il{West Flemish}
    \ea
	\gll   Me   [PRO]  dienen   anderen   post   beloofd   te   zyn, goa-se     niet   veruzen.\\
    with {}    that   other     job   promised   to   be goes-she   not   move\\
	\glt   ‘Having been promised that other job, she’s not going to move house.’
    \ex
	\gll   Ze   pleegdege   zie   zukken   medicamenten voorengeschreven   te zyn.\\
		    she   used     she  such   medications prescribed     to be\\
	\glt     ‘She used to be prescribed that medication.’
    \z
\z

\subsubsection{Coordination}

That it is the \textsc{goal} nominal which is promoted to subjecthood in the
\textsc{goal} \isi{passive} is confirmed by \isi{coordination} data. For instance, an
active clause can coordinate with a \textsc{goal} \isi{passive} clause under
one shared subject (\ref{ex:04.17}a); a clause with a \textsc{theme}
\isi{passive} of a transitive verb can coordinate with a \textsc{goal}
\isi{passive} clause under one shared subject DP (\ref{ex:04.17}b).

\ea%17
    \label{ex:04.17}
    \ea
	\gll   dan   die   twee   studenten   nor us     mochten en   da medicament    neu  niet   meer voorengeschreven   goan   worden.\\
		  that  those   two   students   to home   might and   that medication   now  no   more prescribed     go   be\\
	\glt ‘that those two students were allowed to go home and now will no longer be prescribed that medication.’
    \ex
	\gll   da   Valère  eerst  vur   een interview     utgenodigd   is en   doa   toen  dienen   post   beloofd   is\\
		that  Valère  first  for   an interview    invited   is and   there  then  that   job   promised   is\\
	\glt     ‘that Valère was first invited for an interview and was promised the job there.’
    \z
\z

\subsection{The \textsc{agent} in the \textsc{goal}
passive}\label{sec:04.2.3}

As in other \isi{passive} sentences, in a \textsc{goal} \isi{passive} sentence, the
\textsc{agent} can be overtly expressed \REF{ex:04.18}.

\ea%18
    \label{ex:04.18}\glsdesc{WF}\il{West Flemish}\\
    \gll dan-k    dienen velo   aangeraden   zyn   \textit{door} \textit{twee} \textit{collega's}\\
    that-I   this bicycle   recommended am   by   two   colleagues\\
    \glt ‘that I was recommended that bike by two colleagues.’
\z

An implied \textsc{agent} can be modified by an adjunct: in \REF{ex:04.19},
\emph{per ongeluk} (‘unintentionally’) or \emph{espres} (‘intentionally’)
modify the understood \textsc{agent}.

\ea%19
    \label{ex:04.19}\glsdesc{WF}\il{West Flemish}\\
    \gll dat   Valère    \textit{per} \textit{ongeluk} / \textit{espres} te vele  cortisonepillen   voorengeschreven   wier\\
    that   Valère    by   accident {} intentionally too   many   cortisone.pills     prescribed     was\\
    \glt ‘that Valère was prescribed too many cortisone pills by accident /
    intentionally.’
\z

\subsection{Event passive}\label{sec:04.2.4}\largerpage %2.4. /

Based on the diagnostics in
\textcite{BroekhuisCornips2004,BroekhuisCornips2012}, I conclude that the
\gls{WF}\il{West Flemish} \textsc{goal} \isi{passive} can have an eventive
reading both with the auxiliary\is{auxiliaries} \emph{worden} (‘become’) and
with the -- probably much more common -- alternative \emph{zijn} (‘be’).
Temporal specifiers modifying the event time are compatible with the
\textsc{goal} \isi{passive} \REF{ex:04.20}.

\ea%20
    \label{ex:04.20}\glsdesc{WF}\il{West Flemish}\\
    \gll    dat   Valère    \textit{gisteren} te   vele   cortisonepillen voorengeschreven   is\\
            that   Valère    yesterday too   many   cortisone.pills prescribed     is\\
    \glt    ‘that Valère was prescribed too many cortisone pills yesterday.’
\z

\subsection{Conclusion: WF has a \textsc{goal} passive}\label{sec:04.2.5}

All the diagnostics discussed above converge and point clearly towards the
conclusion that WF, a dialect of \ili{Dutch} and a West \ili{Germanic} language, has
a productive \textsc{goal} \isi{passive}, contrary to claims in the current
generative literature.

Whether the emergence of the \textsc{goal} \isi{passive} in \gls{WF}\il{West
Flemish} can also be attributed to contact with \ili{French}, as argued for
\ili{English} by \citet{SteinTripsIngham2016}, is a question that needs to be
addressed. It is true that the \gls{WF}\il{West Flemish} lexicon provides
strong evidence of contact of \ili{French} as shown in \citet{Haegeman2009}.
An alternative hypothesis might be that the emergence of the \textsc{goal}
\isi{passive} is due to Ingvaeonic influence (see \citealt{Dhaenens2014}). I do
not further speculate on this issue here.

\section{Conclusion}\label{sec:04.4} %3. /

This paper provides empirical evidence against persistent claims in the formal
literature to the effect that \ili{English} is the only West \ili{Germanic} language
with a \textsc{goal} \isi{passive}, showing that at least the West Flemish
dialect of Dutch has a productive canonical \textsc{goal} \isi{passive}. The
\gls{WF}\il{West Flemish} data strongly challenge the claims in the current
literature that \ili{Dutch} lacks a canonical \textsc{goal} \isi{passive},
since at least one \ili{Dutch} dialect does display the pattern.

\printchapterglossary{}

\section*{Acknowledgements}\largerpage

I dedicate this paper to Ian. I have known Ian from the earliest stages of his
career and I have been lucky enough to be able to work with him in Geneva. I
admire the tenacity with which Ian has continued to rethink the linguistic
themes that had initially preoccupied him in his early research and the way in
which his research has developed into a full-fledged research programme that
allows us to attain a deeper understanding of core issues of comparative
syntax.

{\sloppy\printbibliography[heading=subbibliography,notkeyword=this]}

\end{document}
