\documentclass[output=paper]{langsci/langscibook}
\author{Kari Kinn\affiliation{University of Bergen}}
\title{High and low phases in Norwegian nominals: Evidence from ellipsis,
    psychologically distal demonstratives and psychologically proximal
possessives}

\renewcommand{\lsCollectionPaperHeaderTitle}{\thechapter\hspace{0.5em}High and
low phases in \ili{Norwegian} nominals}

% \chapterDOI{} %will be filled in at production

\abstract{This squib discusses the idea of a high and a low phase in \ili{Norwegian}
nominals. I argue that ellipsis phenomena and syntactic constructions yielding
speaker-perspective meanings corroborate the proposal that nominals may have a
biphasal structure.}

\maketitle

\begin{document}\glsresetall

\rohead{\thechapter\hspace{0.5em}High and low phases in Norwegian nominals}

\section{Introduction}

This squib picks up on an idea most recently proposed by e.g.\
\citet{cornilescunicolae2011nominal}, \citet{simpsonsyed2016biphasal},
\citet{simpson2017parallels}, \textcite{SyedSimpson2017} and
\citet[161]{roberts2017finaloverfinalDP}, namely that the extended nominal
projection may consist of two \isi{phases}. If on the right track, this
proposal gives us a new type of evidence for parallel structure in nominals and
clauses (e.g.\ \citealt{Abney1987}; \citealt{Szabolcsi1994}).\footnote{On
    \isi{phases} in the clausal domain, see \citet{Chomsky2000} and much
subsequent work.}

While \citet{cornilescunicolae2011nominal} and the studies by Simpson and Syed
focus on \ili{Romanian} and \ili{Bangla},  I will discuss the idea of a high and a low
nominal phase\is{phases} in \ili{Norwegian}. Previously, \citet{julien2005nominal} has made a
case for biphasal nominals in Scandinavian on the basis of  case-licensing and
definiteness phenomena in certain possessive constructions.\footnote{Julien
argues for a low phase\is{phases} in addition to the more standardly assumed high phase;
see \citet[4--5, 73, 202, 219]{julien2005nominal} for details.} I will
introduce two types of data that are new in the context of \ili{Norwegian}: first,
like \citet{simpson2017parallels} and \textcite{SyedSimpson2017}, I will look at
ellipsis. Then I will consider speaker-perspective\is{speaker perspective}
meanings, which I, drawing on work by e.g.\  \citet{sigurdsson2014context},
take to be derived via syntactic operations at the phase\is{phases}
edges.\footnote{\citet[40]{cornilescunicolae2011nominal} mention
    speaker-perspective\is{speaker perspective} meanings (``judgements by the
    speaker'') as a characteristic of the higher nominal phase\is{phases}, but
    not of the lower one.  Their arguments for a biphasal structure are based
    on the properties of prenominal adjectives and the so-called adjectival
    article construction.  The main data discussed in
    \citet{simpsonsyed2016biphasal} are blocking effects on nominal-internal
movement. \citet{roberts2017finaloverfinalDP} proposes a biphasal structure in
a discussion of the \isi{Final-over-Final
Condition}\is{FOFC|see{Final-over-Final Condition}} in DP.} The
speaker-perspective\is{speaker perspective} meanings to be considered are i)
psychologically distal demonstratives\is{demonstratives!psychologically distal
demonstratives} (e.g.\ \citealt{johannessen2008psycological}) and ii) a
possessive construction that I describe as psychologically proximal.

I assume the following  structure of the extended nominal domain in
Norwegian, as proposed by \citet{julien2005nominal}:

\ea\label{ex:julien-structure}
    {}[\tss{QP}... [\tss{DemP}... [\tss{DP}... [\tss{CardP}... [\tss{α{}P}...
    [\tss{\emph{n}P}... [\tss{NumP}... [\tss{NP}...]]]]]]]]
\z

\noindent In this hierarchy, QP hosts strong quantifiers, DemP
\isi{demonstratives},
CardP numerals/weak quantifiers, and α{}P adjectives (adjectives are sitting in
the specifier of the α head).  DP and \emph{n}P both contribute to
definiteness; the definite suffix originates in \emph{n}P; D mostly probes
and attracts lower material, or, in the case of modified nouns, can be
lexicalised by a pre-adjectival definite determiner which comes in addition to
the definite suffix (so-called \emph{double definiteness}).  Example
(\ref{ex:nominals-structure}a) illustrates the order of different elements in
the nominal phrase (quantifier -- demonstrative -- numeral -- adjective -- noun
with definite suffix); Example (\ref{ex:nominals-structure}b) shows double
definiteness with a pre-adjectival definite determiner.

\ea\label{ex:nominals-structure} \ili{Norwegian}
    \ea
    \gll alle disse tre gode bøk-ene\\
    	all these three good book-\Pl.\Def{}\\
    	\glt `all these three good books'
    \ex	\gll den nye bok-a \\
    	the new book-\Def{} \\
    	\glt `the new book'
	\z
\z

\noindent  On Julien's (\citeyear[12]{julien2005nominal}) analysis, DP,
\emph{n}P, NumP and NP are present in every DP, whereas CardP and α{}P are only
merged when they contain lexical material. I take it that this also applies to
QP and DemP.

\section{Ellipsis}

Like \citet{simpson2017parallels}, I adopt \citegen{Boskovic2014} proposal that
ellipsis is constrained by phases; more precisely, \isi{ellipsis} can affect either
i) the phase\is{phases} itself, or ii) the complement of the phase\is{phases} head (see
\citeauthor{Boskovic2014}'s paper and references there for cross-linguistic
evidence). On this approach, \isi{ellipsis} of complements of non-phase heads is
disallowed \citep[42]{Boskovic2014}. For illustration, compare
(\ref{ex:contrastfrombosc}a) and  (\ref{ex:contrastfrombosc}b) (from
\citealt[56]{Boskovic2014}; \isi{ellipsis} is marked by strikethrough):

\ea\label{ex:contrastfrombosc}
	\ea Betsy must have been being hassled by the police, and Peter must have been \sout{being hassled \dots}
    \ex *Betsy must have been being hassled by the police, and Peter must have been being \sout{hassled \dots}
	\z
\z

\noindent In (\ref{ex:contrastfrombosc}a), the complement of a phase\is{phases} head is
elided (the phase\is{phases} head is Asp1, spelt out by \emph{been}; see
\citealt[62]{Boskovic2014} for the full syntactic structure).  In
(\ref{ex:contrastfrombosc}b), on the other hand, not only \emph{been}, but also
\emph{being} is stranded; this would involve \isi{ellipsis} of the complement of a
non-phase head, which is not acceptable.

Some languages seem to disallow \isi{ellipsis} for independent reasons even
under the appropriate phasal conditions \citep[48]{Boskovic2014}; thus,
\isi{ellipsis} being impossible does not necessarily exclude the presence of a
phase\is{phases}.  However, according to \citeauthor{Boskovic2014}'s analysis,
the possibility of \isi{ellipsis} can be taken as an indication of phasehood.

\subsection{Ellipsis in the higher phase}

Ellipsis data suggest the presence of a phase\is{phases} in the higher nominal domain in
Norwegian. It is, for example, possible to strand a prenominal possessive
pronoun while the rest of the nominal phrase is elided, as illustrated in
Example (\ref{ex:possessors}) (the relevant nominals are in boldface):\newpage

\ea\label{ex:possessors} \ili{Norwegian}
		\ea
		 \gll Han er min beste venn, og jeg er \textbf{hans} \sout{\textbf{beste} \textbf{venn}}.\\
		 he is my best friend and I am his \sout{best friend}\\
		 \glt `He is my best friend, and I am his.'

		\ex
		\gll Jeg kom i min fineste kjole, og Anne kom i \textbf{sin} \sout{\textbf{fineste} \textbf{kjole}}\\
		I came in my nicest dress and Anne came in her.\textsc{refl} \sout{nicest dress}\\
		\glt `I was wearing my nicest dress, and Anne was wearing hers.'

		\z
\z

\noindent I follow \citet[207, 210]{julien2005nominal}, who argues that
prenominal possessive pronouns are first-merged in Spec-NP and move to Spec-DP
(via intermediate positions). What we have in example (\ref{ex:possessors})
then, is \isi{ellipsis} of everything below D (α{}P, \emph{n}P, NumP and NP).
The most obvious analysis that presents itself is that D is a phase\is{phases}
head whose complement is elided. The analysis is illustrated (somewhat
simplified) in (\ref{ex:illustDPell}):

\ea\label{ex:illustDPell}
	hans \sout{beste venn}

	[\tss{DP} [\tss{\sout{α{}P}} [\tss{\sout{\emph{n}P}} [\tss{\sout{NumP}} [\tss{\sout{NP}}]]]]]
\z

\noindent It is worth noting that not only DP, but also projections located
even higher in the nominal phrase can license \isi{ellipsis}. This lends support to
\citeauthor{Boskovic2014}'s (2014) proposal that \isi{phases} are contextually
defined: the edge of the phase\is{phases} is constituted by the highest functional
projection present. Thus, in a structure where a QP is merged above DP, Q will
be the phase\is{phases} head. An example of \isi{ellipsis} with a stranded QP element (the
strong quantifier \emph{alle} `all') is provided in Example
(\ref{ex:QPellipsis}):\footnote{It is also possible to strand a strong
    quantifier and a demonstrative:\is{demonstratives} \emph{Alle disse \sout{bøkene} er solgt},
lit.\ `all these \sout{books} are sold'.  Many such cases can be
straightforwardly analysed as \isi{ellipsis} in the lower phase\is{phases}, which is discussed
in the next section. An issue that invites further research, both
empirically and theoretically, concerns \isi{ellipsis} of a noun modified by an
adjective in such contexts (an elided adjective would be higher than
\emph{n}P).  I leave that  aside here.}

\ea\label{ex:QPellipsis} \ili{Norwegian}
    \ea
    \gll Det er noen ekstra skruer i skuff-en, men ikke ta  \textbf{alle}
    \textbf{\sout{de ekstra skru-ene i skuff-en}}\\
	there are some spare screws in drawer-\Def{} but not take all \sout{the spare screw-\Pl.\Def{} in drawer-\Def{}}\\
    \glt `There are some spare screws in the drawer, but don't take all of them.'
    \ex alle \sout{de ekstra skruene i skuffen}
		[\tss{QP} [\tss{\sout{DP}} [\tss{\sout{α{}P}} [\tss{\sout{\emph{n}P}} [\tss{\sout{NumP}} [\tss{\sout{NP}}]]]]]]
    \z
\z

\subsection{Ellipsis in the lower phase}

While the data presented above seem to indicate a phase\is{phases} headed by the topmost
projection in the nominal domain, \ili{Norwegian} also allows \isi{ellipsis} exclusively
targeting material in the lower part of the nominal. The perhaps clearest
evidence of this is \isi{ellipsis} following adjectives, as illustrated in
(\ref{ex:nP-ellipsis}):

\ea\label{ex:nP-ellipsis} \ili{Norwegian}
    \ea
    \gll Vi har vanligvis t-skjorter i alle farger, men \textbf{de}
    \textbf{svarte}  \textbf{\sout{t-skjort-ene}} er utsolgt akkurat n\aa{}.\\
        we have usually t-shirts in all colours but the black
        \sout{t-shirt-\Pl.\Def{}} are sold.out just now\\
    \glt `We normally have t-shirts in all colours, but the black ones are sold out right now.'
    \ex
    \gll Jeg har funnet de fleste nøkl-ene vi mistet, men \textbf{alle}
    \textbf{de} \textbf{fire} \textbf{sm\aa{}} \textbf{\sout{nøkl-ene}} er fortsatt borte.\\
    I have found the most key-\Pl.\Def{} we lost but all the four small
    \sout{key-\Pl.\Def{}} are still missing\\
    \glt `I have found most of the keys that we lost, but all of the four
    small ones are still missing.'
    \z
\z

\noindent Recall that adjectives are located in α{}P, a projection
below DP and CardP. On the assumption that \isi{ellipsis} can only affect \isi{phases} and
complements of phase\is{phases} heads, the examples in (\ref{ex:nP-ellipsis}) cannot be
licensed by the topmost functional projection. In Example
(\ref{ex:nP-ellipsis}a), the highest element present is a pre-adjectival
definite determiner, and the phase\is{phases} head would be D. The elided material, a
noun with a definite suffix, is located in \emph{n}P, which is a
complement of α{}, i.e. a non-phase head. In (\ref{ex:nP-ellipsis}b),
the highest element present is a strong quantifier, and the phase\is{phases} head would
be Q. Again, the elided material is  located in \emph{n}P, a complement of
α{}, and in addition to  α{}P, both CardP and DP intervene
between the \isi{ellipsis} site and the highest phase\is{phases} head. To account for the data,
I propose, consistently with \citet{julien2005nominal} (who reaches this
conclusion on different grounds), that \emph{n}P is a phase\is{phases} and that the
examples in (\ref{ex:nP-ellipsis}) are phasal \isi{ellipsis} of
\emph{n}P.\footnote{\citet{simpson2017parallels}, citing
    \citet{ruda2016NPellipsis}, makes a similar proposal for \ili{Polish} and
\ili{Hungarian}.} The analysis is illustrated in (\ref{ex:nP-ellipsis-analysis}):

\ea\label{ex:nP-ellipsis-analysis}
		\ea
		de svarte \sout{t-skjortene}

	[\tss{DP} [\tss{α{}P} [\tss{\sout{\emph{n}P}...} ]]]

	\ex
	alle de fire små \sout{nøklene}

		[\tss{QP} [\tss{DP} [\tss{CardP} [\tss{α{}P} [\tss{\sout{\emph{n}P}...} ]]]]]

		\z

\z

\noindent Having looked at some \isi{ellipsis} data, we now turn to
speaker-perspective meanings.

\section{Speaker-perspective meanings}

There is now a significant body of work developing formal syntactic accounts of
phenomena related to speech acts, indexicality and speaker perspective, going
back to \citegen{ross1970declarative} (e.g.\ \citealt{SpeaTenn2003};
\citealt{giorgi2010about}; \citealt{hill2013vocatives};
\citealt{sigurdsson2014context}; \citealt{wiltschkoheim2016confirmationals}).
While many works focus exclusively on the left periphery of  CP,
\citet[179]{sigurdsson2014context} connects \isi{speaker perspective} (and
indexicality more generally) to \isi{phases} and argues that edge linkers, a
type of feature that enables narrow syntax to link to context and that includes
speaker and hearer features, must be present in \emph{any phase} (although some
phases may not have a full set). This proposal, which I adopt here, is
consistent with the idea that \isi{phases} have a parallel structure
\citep{poletto2006parallel}. The edge linkers most relevant for the present
discussion are the following:

\ea
	\ea  Λ\tss{A}, representing the logophoric agent (speaker).
	\ex   Λ\tss{P}, representing the logophoric patient (hearer).
	\z
\z

\noindent If there is evidence that speaker-perspective\is{speaker perspective} meanings can arise from
syntactic operations both in  the higher and the lower part of the nominal
domain, it could be taken to suggest that there are two nominal \isi{phases}.

\subsection{Speaker-perspective meanings in the higher phase}

In the higher nominal domain, a clear example of speaker-perspective\is{speaker perspective} meanings
is provided by so-called \glspl{PDD},\is{demonstratives!psychologically distal demonstratives} most elaborately described by
\citet{johannessen2008psycological} (see also further references cited
there).\footnote{Other relevant speaker-perspective\is{speaker perspective} phenomena are possibly the
\gls{EAC} \parencite[294--297]{halmoy2016nominal} and certain uses of \emph{sånn}
`such' \parencite{johannessen2012modale}.}  The
\gls{PDD}\is{demonstratives!psychologically distal demonstratives} itself has the same phonological form as a 3rd person personal
pronoun, but when it combines with a (human) noun, it conveys a particular
meaning: it signals psychological distance.  This  sets it apart from regular
\isi{demonstratives}. Often, the \gls{PDD} is used when the speaker does not know the
person under discussion personally, or when they want to signal a negative
attitude towards that person (cf.\ examples (\ref{ex:PDD}a,b)).\footnote{All
examples in (\ref{ex:PDD}) are from \citet{johannessen2008psycological};
notation and translations slightly adapted.} The reference point may also be
with the hearer: the speaker uses the \gls{PDD} to introduce someone that they
are familiar with themselves, but that the \emph{hearer} might not know
personally  (cf.\ \ref{ex:PDD}c).\is{demonstratives!psychologically distal
demonstratives}

\ea \label{ex:PDD} \ili{Norwegian}
    \ea
    \gll jeg og Magne vi sykla jo og \textbf{han} \textbf{Mikkel} da\\
    I and Magne we cycled then and he Mikkel then\\
    \glt `Me and Magne and that guy Mikkel, we rode  our bikes' (NoTa, M, 36, \citealt[164]{johannessen2008psycological}	)

\ex
\gll \textbf{hun} \textbf{dam-a} hun blei jo helt nerd da\\
she woman-\Def{} she became yes totally nerd then\\
\glt `That woman, she became a complete nerd, you know.' (NoTa, M, 18, \citealt[166]{johannessen2008psycological})

	\ex
\gll du vet \textbf{han} \textbf{kj\o{}rel\ae{}rer-en} jeg har?\\
you know he driving.teacher-\Def{} I have\\
\glt `You know that driving instructor I have?' (NoTa, F, 18, \citealt[164]{johannessen2008psycological})
		\z
\z

\noindent \citet[178]{johannessen2008psycological} shows that the \gls{PDD} in
Norwegian cannot co-occur with the pre-adjectival definite determiner in double
definiteness constructions (example (\ref{ex:nominals-structure}b)); the most
obvious interpretation of this  is that the
\gls{PDD}\is{demonstratives!psychologically distal demonstratives} is a D
element.\footnote{Norwegian differs from \ili{Swedish} and \ili{Danish} in this
respect; in \ili{Swedish} and \ili{Danish} the \gls{PDD} seems to be merged
higher \citep[175--176]{johannessen2008psycological}, probably in DemP.} Since
no higher projections are merged in the examples in (\ref{ex:PDD}), DP is a
phase\is{phases} and will contain speaker and hearer features (Λ\tss{A} and
Λ\tss{P}).

I propose that the encoding of psychological distance in relation to the
speaker or hearer is achieved in a way similar to that of deictic gender
control  \citep[185--186]{sigurdsson2014context}. An example of deictic gender
control is given in (\ref{ex:deicticgendercontrol}), where the \ili{Icelandic} 1st
person pronoun triggers agreement in gender (fem. or masc., depending on the
speaker's gender), although the pronoun itself does not exhibit any
overt gender distinctions.

\ea\label{ex:deicticgendercontrol} \ili{Icelandic}
\citep[185]{sigurdsson2014context}\\
	\gll Ég ger\dh{}i \th{}etta sjálfur / sjálf / *sjálft\\
    I did this self.\M{} {} self.\glossF{} {} \hphantom{*}self.\glossN{}\\
	\glt `I did this myself.'
	\z

\noindent Deictic gender control, according to Sigurðsson, involves gendering
of the speak\-er/hear\-er features. In an example such as
(\ref{ex:deicticgendercontrol}), the speaker feature at the C-edge will have
the value  Λ\tss{A/M} if the speaker is male and Λ\tss{A/F} if she is female;
the value is passed down to the pronoun \emph{ég} \enquote*{I} via Agreement
with the gendered speaker feature and triggers gender agreement in
\emph{sjálfur/sjálf} \enquote*{myself}.  In a similar fashion, I propose that
the PDDs in (\ref{ex:PDD}a) and (\ref{ex:PDD}b) get their
psychologically distal meaning via a speaker feature at the D-edge with the
specification Λ\tss{A/PSYCH-DIST}. The \gls{PDD}\is{demonstratives!psychologically distal demonstratives} in (\ref{ex:PDD}c) differs in
that the hearer, not the speaker, is the reference point; in this case, the
syntactic source of the psychologically distal meaning would be the hearer
feature, with the specification Λ\tss{P/PSYCH-DIST}.

\subsection{Speaker-perspective meanings in the lower phase?}

The next question is whether there is any evidence for
speaker-perspective\is{speaker perspective} meanings arising  in the
\emph{lower} nominal domain.  I would like to draw attention to a particular
possessive construction that might instantiate this.  The construction
involves a proper or common noun  and a postposed 1st person possessive
pronoun, and it contrasts with  the \gls{PDD}\is{demonstratives!psychologically
distal demonstratives} in  that it does not convey psychological distance; on
the contrary, it yields a very affectionate reading and is only appropriate in
intimate contexts.\footnote{This description is based on my intuitions as a
native speaker of \ili{Norwegian}.} The construction seems to be primarily used
in vocatives, and to my knowledge, it has not been discussed much in the
previous literature, although it is very briefly touched upon by
\citet{julien2016predicationalvocatives}.\footnote{\citet[90]{julien2016predicationalvocatives}
    writes: ``The use of first person possessive pronouns in vocatives would be
    an interesting topic in itself, especially since it often appears to add a
    flavour of endearment to the utterance, but I will leave this topic aside
    here.''}\textsuperscript{,}\footnote{The construction bears some
    resemblance to the \glsreset{EAC}\gls{EAC}
    \citep[294ff]{halmoy2016nominal}, which consists of an adjective and a noun
    with a definite suffix.  However, there are important differences. While
    the \gls{EAC} is characterised by the presence of an adjective, the
    construction to be discussed here does not necessarily contain other
    modifiers than the possessive. The \gls{EAC} occurs independently of
    possessive pronouns. Moreover, the \gls{EAC} does not necessarily convey
affection; it can also express negative feelings.}

Because the construction  conveys the opposite	 of psychological distance,
namely  psychological proximity, I refer to it as the \emph{\gls{PPP}
    construction}. Some authentic examples are given in
    (\ref{ex:affectiontepossessive}):\footnote{Some speakers report that they
    do not use the construction with proper names, but they generally seem to
be familiar with it.}

\ea\label{ex:affectiontepossessive}
	\ea
		\gll Natt'a, \textbf{Anne} \textbf{min}. Jeg f\aa{}r vel kalle deg det?\\
		night-night Anne my I get well call you that\\
		\glt `Night-night, my dearest Anne. I suppose I can call you that?' (The novel \emph{St\o{}rst av alt}, Lillian Wirak Skow, 2010)

		\ex
		\gll \textbf{S\o{}te } \textbf{H\aa{}kon} \textbf{v\aa{}r} du fyller 8 \aa{}r den 18. juni, hipp hurra for deg!\\
		sweet H\aa{}kon our you fill 8 years the 18 June, hip hooray for you\\
	\glt `Our sweet H\aa{}kon, you turn 8 on  18th  June, hip hooray for you!'
    (Birthday greeting in local newspaper,
2013)\footnote{\url{https://www.an.no/vis/personalia/greetings/3561747} (accessed 22/11/2017).} %

		\ex

\gll Jeg vil for alltid b\ae{}re med meg minne om deg \textbf{Kari} \textbf{min} i mitt hjerte\\
I will for always carry with me memory about you Kari my in my heart\\
\glt `I will carry with me the memory of you in my heart for ever, my dearest
Kari'. (Memorial webpage, 2017)\footnote{\url{https://wang.vareminnesider.no/}
(accessed 22/11/2017; full URL omitted because of the sensitive nature of this
example).}

    \ex
\gll  [...] du vil aldri bli glemt, \textbf{Godgutt-en} \textbf{min}\\
{} you will never be forgotten good.boy-\Def{} my\\
\glt `You will never be forgotten, my good boy' (Kennel webpage,
2015)\footnote{\url{http://kennelulwazi.com/våre\%20hunder/gandhi/index.html} (accessed 22/11/2017).}

    \ex
\gll [...] Elsker deg masse \textbf{venn-en} \textbf{min} :-)\\
{} love you lots friend-\Def{} my\\
\glt `I love you a lot, sweetie!' (Text message)\footnote{\url{http://www.p4.no/underholdning/p4-lytternes-beste-kjerlighetsmeldinger/artikkel/336327} (accessed 22/11/2017).}

    \z
\z\largerpage[-2]

\noindent The examples in (\ref{ex:affectiontepossessive}a--c) illustrate the
PPP construction with proper names.   (\ref{ex:affectiontepossessive}a) is
taken from a novel, more precisely from a scene in which a new couple are
saying good night to each other. Note that the person who addresses his
girlfriend as \emph{Anne min} (lit.\ `Anne my') explicitly  asks for permission
to do so; this highlights the intimate style  of the construction. Example
(\ref{ex:affectiontepossessive}b) is from a birthday greeting to a young boy
from his  parents;   (\ref{ex:affectiontepossessive}c) is taken from a memorial
webpage. The examples in (\ref{ex:affectiontepossessive}d,e) illustrate the
PPP construction with common nouns;  (\ref{ex:affectiontepossessive}d) is a
greeting addressed to a dog on a kennel web page;
(\ref{ex:affectiontepossessive}e) is from a text message exchange between
spouses. Note that when  the noun in a PPP construction is modified by an
adjective, like in (\ref{ex:affectiontepossessive}b), there is no
pre-adjectival definite determiner (i.e. no double definiteness); this is a
characteristic of the PPP construction (and vocatives in
general).\footnote{Occurrences of what looks like the PPP construction can be
    found in non-vocative contexts too: \emph{[\dots{}] ta godt vare på
    \textbf{Håkon} \textbf{vår}} `take good care of our dearest Håkon'
    (\url{http://www.torgeirogkjendisene.no/10/48/2/bangkok-og-cha-am-thailand-19-29-september/},
    accessed 28/11/2017). However, in this paper, I limit my attention to
    vocatives. Postposed possessive pronouns are regularly used in \ili{Norwegian},
    and in non-vocative contexts a postnominal 1st person possessive does not
    necessarily yield an affectionate reading; a statement like \emph{Jeg skal
besøke \textbf{broren} \textbf{min} } `I am going to visit my brother' comes
across as neutral. }

Now,  it could  be argued that  the psychologically proximal meaning of the PPP
construction is a pragmatic (i.e. non-syntactic) phenomenon that automatically
follows when certain nouns (including proper nouns) are  combined with a 1st
person  possessive pronoun. However, although possessives are regularly
postposed, \ili{Norwegian} also allows preposed possessive pronouns, and, in
these contexts, the degree of affection and intimacy associated with the PPP
construction does not  arise. Imagine a situation in which a highly respected
senior member of staff in a company is about to retire and a more junior
member of staff is giving a  speech. The speaker could be expected to say
something along the lines of (\ref{ex:speech}a), with a preposed possessive
pronoun. The minimally different example in (\ref{ex:speech}b), on the other
hand, with a postposed possessive, would come across as inappropriate; the PPP
construction conveys too much intimacy in the given context.\footnote{Again,
this description is based on my native-speaker intuitions; I have consulted
other native speakers who agree.}

\ea\label{ex:speech} \ili{Norwegian}
    \ea[]{
		\gll \textbf{Vår} kjære \textbf{Anne}, vi ønsker deg alt godt i år-ene som kommer.\\
		our dear Anne we wish you all good in year-\Pl.\Def{} that come\\
    \glt `Our dear Anne, we wish you all the best in the years to come.'}

    \ex[\#]{
	\gll Kjære \textbf{Anne} \textbf{vår}, vi ønsker deg alt godt i år-ene som kommer.\\
	dear Anne our we wish you all good in year-\Pl.\Def{} that come\\
    \glt intended meaning: `Our dear Anne, we wish you all the best in the years to come.'}
	 		\z
\z

\noindent With regard to the examples with common nouns  in
(\ref{ex:affectiontepossessive}d,e), one might perhaps wonder if the proximal,
affectionate reading is simply due to the lexical semantics of the cited nouns;
the nouns used in the PPP construction often have a ``pet-name-like''  feel
even in other contexts. Note, however, that nouns that are neutral with
respect to such inherent properties can also be used, and the  proximal reading
still arises, as illustrated in (\ref{ex:brannmann}):

\ea\label{ex:brannmann} \ili{Norwegian}\\
	\gll  Gratulerer masse med dagen  lille \textbf{brannmann-en} \textbf{vår}!\\
	congratulations much with day.\Def{} little fire.man-\Def{} our\\
    \glt `Happy birthday, our little fire man!' (Birthday greeting in local
    newspaper)\footnote{\url{http://www.f-b.no/vis/personalia/greetings/7330499}
    (accessed 22/11/2017).}
\z

\noindent Also, note that nouns whose lexical semantics are at odds with
notions such as intimacy and affection seem inappropriate in the PPP
construction. Cf.\ the contrast between (\ref{ex:infelicitous}a) and
(\ref{ex:infelicitous}b):\footnote{Example (\ref{ex:infelicitous}b)
    would sound stylistically marked even with a prenominal possessive pronoun,
    but not as inappropriate as it does with a postnominal possessive,
according to my judgement.}

\ea\label{ex:infelicitous} \ili{Norwegian}
\ea[]{
		\gll Kom hit, kjærest-en min!\\
		come here girlfriend/boyfriend-\Def{} my\\
    \glt `Come here, my love!'}

    \ex[\#]{
		\gll Gå bort, fiend-en min!\\
		go away, enemy-\Def{} my\\
    \glt intended meaning:`Go away, my enemy!'}
				\z
\z

\noindent The data presented  in  (\ref{ex:speech})--(\ref{ex:infelicitous})
seem to suggest that the speaker-perspective\is{speaker perspective} meaning of
the PPP construction follows from its  syntax, not from pragmatics or lexical
semantics.  I  propose the following analysis of the PPP construction.

\emph{n}P is a phase\is{phases} and thus contains edge linkers. In the \gls{PPP}
construction, the Λ\tss{A} feature of \emph{n}P is equipped with a proximal
counterpart of the \textsc{psych-dist} specification responsible for the
\gls{PDD}\is{demonstratives!psychologically distal demonstratives} construction (see above); I call this Λ\tss{\textsc{a/psych-prox}}.
Now, just as in regular possessive constructions, postposing of the possessive
pronoun follows from movement of the noun from its NP-internal position past
the possessive, which is first-merged  in Spec-NP
\citep[143]{julien2005nominal}, and up to the edge of \emph{n}P. The difference
is that in the PPP construction, the possessive pronoun Agrees with
Λ\tss{\textsc{a/psych-prox}}; this yields the psychologically proximal reading.
A sketch of the relevant pieces of structure  is given in
(\ref{ex:derivationPPP}) (for convenience I mark movement with traces and the
Agreement relation between the possessive and the edge linker with an
arrow):\footnote{I follow \citet{julien2005nominal} in analysing the movement
of the noun as \isi{head movement}.}

\ea\label{ex:derivationPPP}
    Anne min\\
    {}[\tss{\emph{n}P}
        [\tss{\emph{n}} Λ\tss{\textsc{a/psych}\tn{a}{}-\textsc{prox}} Anne\tss{i} ]
        [\tss{NumP}
            [\tss{Num} t\tss{i} ]
            [\tss{NP} min\tss{\textsc{a/psych}\tn{b}{}-\textsc{prox}}
                [\tss{N} t\tss{i} ]%
            ]%
        ]%
    ]
    \tikz[remember picture, overlay, baseline]
    \draw [arrow, <->, shorten >= 1mm, shorten <= 1mm] (a) -- +(0,-.5) -| (b);
    \vspace{.5\baselineskip}
\z

\noindent Admittedly, it is a challenge to show unequivocally that a syntactic
operation in \emph{n}P is  responsible for the speaker-perspective\is{speaker
perspective} meaning in the PPP construction;  it does not have overt,
phase-internal morphological or syntactic effects (unlike the
\gls{PDD}\is{demonstratives!psychologically distal demonstratives} in the DP
phase\is{phases}, which has a special form).  A full investigation into this
issue must be left for future research; in particular, it is important to
consider possible interactions with the higher phase, for which the concept of
speaker/hearer-perspective is currently more established.\footnote{In
    vocatives, the higher phase\is{phases} is probably not DP
    \citep{Longobardi1994}; the lack of a D-layer in \ili{Norwegian} vocatives
    is evidenced by the lack of a pre-adjectival definite determiner with
    modified nouns (cf.\ Examples \ref{ex:affectiontepossessive}b and
    \ref{ex:brannmann}).  One could perhaps  argue that vocatives are small
    (reduced) nominals, a parallel to \isi{small clauses}
    \citep{Pereltsvaig2006}, consisting of the lower phase\is{phases} only.
    However, recent research argues for a Voc projection that encodes the
    vocative function (e.g.\ \citealt{hill2007vocatives, hill2013vocatives};
    \citealt{espinal2013vocatives}; \citealt{stavrou2013aboutthevocative};
\citealt{julien2014vokativar, julien2016predicationalvocatives}). VocP would be
a phase\is{phases} if \isi{phases} are contextually defined. } However, I would
like to point out some  possible indications that the PPP construction indeed
gets its speaker-perspective meaning from an edge linker in  \emph{n}P.

First, as shown in Example (\ref{ex:affectiontepossessive}b), repeated below in
(\ref{ex:re_adj}), the PPP construction is compatible with a prenominal
adjective:

\ea\label{ex:re_adj} \ili{Norwegian}\\
	\gll \textbf{S\o{}te } \textbf{H\aa{}kon} \textbf{v\aa{}r} du fyller 8 \aa{}r den 18. juni, hipp hurra for deg!\\
	sweet H\aa{}kon our you fill 8 years the 18 June, hip hooray for you\\
	\glt `Our sweet H\aa{}kon, you turn 8 on  18th  June, hip hooray for you!' (Birthday greeting in local newspaper)
\z

\noindent Since adjectives are merged in Spec-α{}P (cf.\ Example
\eqref{ex:julien-structure}), this suggests that the noun does not leave
\emph{n}P, and that the postnominal possessive pronoun stays in an even lower
position, in Spec-NP.  This does not in itself  exclude the possibility of
interaction with edge linkers in the higher phase\is{phases}, but it is certainly
compatible with \emph{n}P as  the locus of the
\textsc{Λ\tss{a/psych-prox}} feature.  Second, in terms
of its meaning, the PPP construction bears  resemblance to diminutives;
cross-linguistically it is common for diminutives to mark affection  (see
\citealt{jurafsky1996diminutives} and references there).  Diminutive
formation is often thought to take place in a low position in the nominal;
\citet{wiltschko2006diminutives} proposes, on independent grounds, that
diminutives (e.g.\ in German) are light nouns in n, comparable to \emph{n} in
the framework adopted here. To me it seems plausible that the PPP construction
and diminutives have structural similarities, so that arguments for diminutive
formation in \emph{n}P are also relevant for the PPP construction. I
hypothesise that a speaker-perspective\is{speaker perspective}
\emph{n}-edge-linker is involved in diminutives marking affection, and  that
the PPP construction arises via syntactic operations involving the same
feature. The similarity between the PPP construction and diminutives finds some
support in orthography: the PPP construction can occasionally  be found with a
hyphen linking the noun and the possessive pronoun, as shown in
(\ref{ex:Marianne-min}):\footnote{I have only seen this orthographic pattern in
PPP constructions involving proper names.}

\ea\label{ex:Marianne-min} \ili{Norwegian}\\
	\gll 	Gratulerer med dagen, kjære søte fine nydelige \textbf{Marianne-min}\\
	congratulations with day.\Def{} dear sweet lovely beautiful Marianne-my\\
	\glt `Happy birthday, my dear, sweet, lovely, beautiful Marianne'
	 (Birthday greeting on Facebook, 2017)
\z

\noindent The hyphen suggests a tight connection between the noun and the
possessive; it could mean that the possessive pronoun in the \gls{PPP} construction
is a diminutive suffix (see also~\citealt{Lodrup2011}
and~\citealt{Svenonius2017}).

Many \ili{Norwegian} speakers can use the suffixes \emph{-mor} `mother' and
\emph{-far} `father' to  form what can be described as affectionate diminutive
forms of proper names. Interestingly, some of the speakers that I have
informally consulted  report a reluctance to use the diminutive forms in the
PPP construction (I share this intuition); cf.\
(\ref{ex:diminutive}):

\ea\label{ex:diminutive}
\ea[]{\gls{PPP} construction}
		\gll Anne min\\
	Anne my {}\\
\glt

\ex[]{Diminutive}
	\gll Annemor \\
	Anne.\Dim{}\\
\glt

\ex[]{Diminutive used in \gls{PPP} construction}
    \gll \llap{??}Annemor min\\
	    {}Anne.\Dim{} my\\
    \glt
	\z
\z

\noindent There are also  speakers who accept (\ref{ex:diminutive}c); clearly,
further investigations into the inter-speaker variation and its underlying
reasons are needed. However, a possible interpretation of the dubious status of
(\ref{ex:diminutive}c) could be that it is not possible for both the diminutive
suffix \emph{-mor} and the possessive pronoun of the  PPP to enter into a
relationship with the Λ\tss{\textsc{a/psych-prox}} feature at the \emph{n}-edge
at the same time.

\section{Conclusion}

In this squib, I have discussed the idea that \ili{Norwegian} nominal phrases, like
clauses,  can consist of both a high and a low phase\is{phases}. I have shown that
Norwegian allows \isi{ellipsis} both in the higher and lower nominal domain;
according to \citet{Boskovic2014}, \isi{ellipsis} is an indication of phasehood.
Moreover, inspired by \textcite{sigurdsson2014context}, I have argued that
speaker-perspective meanings arise via syntactic operations in the higher
nominal domain (psychologically distal
demonstratives,\is{demonstratives!psychologically distal demonstratives}
\citealt{johannessen2008psycological}), and, somewhat more tentatively, also in the lower part of the nominal
(\emph{n}P)  (in the Psychologically Proximal Possessive construction).
Assuming that speaker-perspective\is{speaker perspective} meanings are related
to edge-linkers at phase\is{phases} edges \citep{sigurdsson2014context}, this
also corroborates a biphasal structure.

\section*{Acknowledgements}

I would like to thank Ian Roberts, whom I had the pleasure of working with on
the ReCoS project at the University of Cambridge, and who remains a great
inspiration. This paper came about from ideas that were discussed during the
6th CamCoS conference. For comments and suggestions, I thank the editors of
this volume, two anonymous reviewers, Janne Bondi Johannessen, Jan Terje
Faarlund and Per Erik Solberg.  Any remaining errors are my own.

\printchapterglossary{}

{\sloppy
\printbibliography[heading=subbibliography,notkeyword=this]
}

\end{document}
