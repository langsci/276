\documentclass[output=paper]{langsci/langscibook}
\ChapterDOI{10.5281/zenodo.4280657}

\title{V3 in urban youth varieties of Dutch}
\author{Marieke Meelen\affiliation{University of Cambridge}\and
    Khalid Mourigh\affiliation{Leiden University}\lastand
    Lisa Lai-Shen Cheng \affiliation{Leiden University}}

% \chapterDOI{} %will be filled in at production

\abstract{In this paper we compare new data from \ili{Dutch} urban youth varieties to emerging
varieties in other \ili{Germanic} languages like \ili{German} and \ili{Norwegian}. We argue that,
unlike previously thought, V3\is{V3 word order} word orders can be found in urban youth varieties
of \ili{Dutch} as well and present data from our new corpus. The V3\is{V3 word order} patterns in our
dataset share most characteristics of the optional V3\is{V3 word order} innovations observed in
other \ili{Germanic} urban youth varieties: the sentence-initial constituent is a
frame-setter of any category and the preverbal constituent is mainly the
subject that functions as a \emph{familiar topic}. We adopt Walkden's (2017)
analysis and extend it by adding an additional FrameP so that preverbal
constituents that do not function as familiar topics could be accounted for as
well. Following Wolfe's cline of possible V2-languages, we argue that the \ili{Dutch}
urban youth varieties can best be analysed as \enquote{Force-V2 system 1} grammars with
V-to-Force movement + an additional FrameP. They thus differ from Standard
Dutch, which is argued to be a \enquote{Force-V2 system 2} based on the fact that only
hanging or left-dislocated\is{left dislocation} topics can be found in sentence-initial position of
superficial V3\is{V3 word order} patterns. This data thus presents an interesting case of
syntactic change in the opposite direction: from strict V2\is{V2 word order} to V2\is{V2 word order} with optional
V3 orders.}


\begin{document}\glsresetall
\maketitle
\section{Introduction}\label{sec:intro}


Main clauses in Modern \ili{Dutch} are characterised by the verb-second
(V2) constraint (cf.\ \citealt{Zwart:1997}). Just like in Modern \ili{German} and
Scandinavian languages, the finite verb linearly follows a variety of
sentence-initial constituents, as shown in \eqref{exV2} for subjects,
objects and adjuncts.\footnote{Throughout this article the inflected verbs in
    the examples will be indicated in \textit{italics}. Unless specified
    otherwise, all examples are from a small corpus of a \ili{Dutch} urban youth
    variety compiled by Khalid Mourigh in 2013--2017, recorded in Gouda (see
also \Cref{sec:lingset} and the Appendix).}

\ea\label{exV2} Standard \ili{Dutch}
    \ea
    \gll Ian \textit{vierde} zijn verjaardag gisteren.\\
    Ian celebrated his birthday yesterday\\
    \ex
    \gll Zijn verjaardag \textit{vierde} Ian gisteren.\\
    His birthday celebrated Ian yesterday\\
    \ex
    \gll\label{exV2c}Gisteren \textit{vierde} Ian zijn verjaardag.\\
    Yesterday celebrated Ian his birthday\\
    \trans \enquote*{Ian celebrated his birthday yesterday.}
    \z
\z

\noindent All three options are grammatically correct in Standard \ili{Dutch}, but
the choice of sentence-initial constituent is pragmatically conditioned.
Verb-third (V3) orders as seen in the \ili{English} translation of example
\eqref{exV2c}, are not allowed in Standard Modern \ili{Dutch}:

\begin{exe}
\ex\label{exV3} Standard \ili{Dutch}\\
    \gll \llap{*}Gisteren Ian \textit{vierde} zijn verjaardag.\\
yesterday Ian celebrated his birthday\\
\trans Intended: `Ian celebrated his birthday yesterday.'
\end{exe}

\noindent Recently, some varieties of \ili{Germanic} V2\is{V2 word order} languages have been reported
to exhibit V3\is{V3 word order} orders alongside the standard V2\is{V2 word order} patterns (see, among others,
\citealt{Freywaldetal:2015}, \citealt{Wiese:2013}, \citealt{WieseRehbein:2016}
and \citealt{Walkden:2017}). These new \ili{Germanic} varieties have emerged in
multilingual settings in large cities in various countries in
Europe.\footnote{The term \enquote{urban youth varieties} will be used for
    these varieties of \ili{Dutch} throughout this paper, because it has the least
    pejorative connotation and it captures the sociolinguistic characteristics
    of being spoken by young people in urban, multilingual settings. Other
    terms for these varieties of \ili{Danish}, \ili{Norwegian}, \ili{Swedish}
    and \ili{German}, such as
    \enquote{ethnolect}, \enquote{multiethnolect}, \enquote{Kiezdeutsch} (`neighborhood German') or
    \enquote{Kebab Norwegian} are problematic because they do not characterise the
    exact nature of the varieties and often have strong derogatory overtones
    (cf.\ \citealt{Walkden:2017}; \citealt{Aarsaether:2010}).} Various examples
    of these unexpected V3\is{V3 word order} or XSV orders in these languages that usually exhibit the V2\is{V2 word order} constraint have been
cited by \citet{Freywaldetal:2015} and \citet{Walkden:2017}:\footnote{Since the preverbal constituent is
        usually the subject of the sentence, \citet{Freywaldetal:2015} refer to
        them as \enquote{XSV} with any type of constituent \enquote{X}
        preceding the subject and the verb. In our present corpus, we only find
        preverbal subjects as well.  \citet{Walkden:2017}, however, presents
        some examples of light adverbials in the \ili{German} urban youth variety
        \enquote{Kiezdeutsch}. The lack of light adverbials like \emph{hier} `here' and
        \emph{da} `there' in our present corpus is presumably the result of our
        small dataset rather than the result of a structural restriction. The
        \ili{Dutch} adverbs (\emph{hier} and \emph{daar}) are functionally equivalent
        to their \ili{German} counterparts and we therefore have no reason to assume
    urban varieties of \ili{Dutch} differ in this respect from Kiezdeutsch. The \ili{Dutch}
urban dialect could in theory be different, however. Therefore, we continue to
use the term \enquote{V3} to refer to these innovative word order patterns.}

\ea
    \ea \ili{German} urban youth variety \parencite[787]{Wiese:2009}\\
    \gll morgen ich \textit{geh} Arbeitsamt\\
    tomorrow I go job.centre\\
    \trans \enquote*{Tomorrow I will go to the job centre.}\\
    \ex \ili{Norwegian} urban youth variety \parencite[133]{Opsahl:2009a}\\
    \gll n\aa{} de \textit{f{\aa}r} betale\\
    now they get pay\\
    \trans \enquote*{Now they have to pay.}\\
    \ex Danish urban youth variety \parencite[47]{Quist:2008}\\
    \gll normal man \textit{g{\aa}r} p{\aa} ungdomsskolen\\
    usually one goes to youth.club\\
    \trans \enquote*{Normally you attend the youth club.}\\
    \ex \ili{Swedish} urban youth variety \parencite[53]{Ganuza:2008}\\
    \gll d\aa{} alla \textit{börja(de)} hata henne\\
    then everyone started hate her\\
    \trans \enquote*{Then everyone started hating her.}
    \z
\z
%\end{exe}                   % sets up the top-level example environment


\noindent \citet{Appel:1984}, \citet[91]{AppelMuysken:1987} and
\citet{SchwartzSprouse:2000} have reported that adult L2 learners of \ili{Dutch}
produce adverb-subject-verb orders (XSV or AdvSV) as well:

\begin{exe}                   % sets up the top-level example environment
\ex \ili{Dutch} L2 learner \parencite{Appel:1984}\\
    \gll En dan hij \textit{gaat} weg.\\
    and then he goes away\\
    \trans \enquote*{And then he goes away.}
\end{exe}                   % sets up the top-level example environment

\noindent However, according to \citet{Freywaldetal:2015}, there are very few
violations of the V2\is{V2 word order} constraint found in three case studies of \ili{Dutch} they
examined: bilateral interviews with a mixed groups of young people from Lombok
\citep{Cornips:2002}, interviews with four male adolescents of Surinamese,
Creole descent \citep{CornipsDeRooij:2013} and in- and out-group conversations
in the classical Labovian method with speakers from a \ili{Dutch}, Moroccan-Dutch,
and \ili{Turkish}--\ili{Dutch} background. The only three examples are the following (cited
by \citealt[86--87]{Freywaldetal:2015}):

\ea
    \ea Utrecht/TCULT corpus, Badir\\
    \gll toen we \textit{hadden} eerst twee autos\\
    then we had first two cars\\
    \trans \enquote*{Then, we first had two cars (and later only one).}\\
    \ex Utrecht/TCULT: Badir\\
    \gll daarom ik \textit{heb} dat probleem niet\\
    {that's why} I have that problem not\\
    \trans \enquote*{That's why I don't have that problem.}\\
    \ex Adam-Nijmegen/etnolects project: Hassan, see \citet{Lukassen:2011}\\
    \gll daarom Nederland \textit{is} niet echt meer van eh\\
    {that's why} {the Netherlands} is not really more like eh\\
    \trans \enquote*{That's why the Netherlands is no longer more like eh \dots}
    \z
\z

\noindent They conclude from this that the \ili{Dutch} urban youth variety, unlike
its V2\is{V2 word order} neighbours in Germany, Denmark, Norway and Sweden, \enquote{does not
allow loosened grammatical restrictions in respect to the XSV order}
\citep[88]{Freywaldetal:2015}.

In this article we first present new data from a \ili{Dutch} urban youth variety
spoken by \ili{Dutch} teenagers with a Moroccan heritage in Gouda
(\Cref{sec:lingset} and \Cref{sec:data}). We argue that these new data show that this
Dutch urban youth variety indeed exhibits violations of the strict V2\is{V2 word order}
constraint. V3\is{V3 word order} orders are attested in our dataset and we suggest this is an
indication that \ili{Dutch} urban youth varieties show the same characteristics as
their \ili{Germanic} neighbours (\Cref{sec:data}). We then proceed to
consider these V3\is{V3 word order} orders in their syntactic context. Although our present
dataset is still quite limited, we will present a tentative synchronic
analysis, elucidating this optional variation in the context of the Standard
Dutch C-domain (\Cref{sec:ana1}). We then sketch a possible scenario of
\isi{language change} and how this relates to the diachronic analyses that have been
proposed for this phenomenon in other \ili{Germanic} urban vernaculars (section
\Cref{sec:ana2}). Finally, we define some areas of future work, based on the
need for different types of data collection and other syntactic deviations from
Standard \ili{Dutch} that affect the C-domain (\Cref{sec:fut}).

%%%%%%%%%%%%%%%%%%%%%%%%%%%%%%%%%%%%%%%%%%%%%%%%%%%%%%%%%%%%%%%%%%%%%%%%%
\section{Linguistic setting}
\label{sec:lingset}
%%%%%%%%%%%%%%%%%%%%%%%%%%%%%%%%%%%%%%%%%%%%%%%%%%%%%%%%%%%%%%%%%%%%%%%%%
\noindent The present study is based on a corpus of oral interviews conducted
by one of the authors with Moroccan \ili{Dutch} teenagers in Gouda. Gouda, which is a
rather small city with 71,105 inhabitants, has the largest Moroccan \ili{Dutch}
population in the Netherlands with 6,892 members. About half of the Moroccan
population in Gouda belong to the second generation, meaning that they were
born in the Netherlands and have at least one parent who was born in Morocco.
According to the people interviewed in Gouda, most members of the local
Moroccan \ili{Dutch} community originate from the region of Nador in North Morocco,
more specifically from Ayt Said, making this linguistically a tight-knit group.

This means that a large percentage of its members have Riffian \ili{Berber} as their
heritage language (98.5\% of the population of the countryside of Ayt Said
speaks Tarifiyt Berber\footnote{Statistics from \url{www.hcp.ma}, last
accessed on 13 December 2017. Tarifiyt \ili{Berber} is one of the three major Berber
languages spoken in Morocco.}). Dialectal \ili{Arabic} also plays an important role
as a lingua franca in general. While it is not used for everyday communication,
Standard \ili{Arabic} still plays an important role in religious life and in the
media. People who were born and raised in the Netherlands primarily use \ili{Dutch}
in daily life (already in the 1980s, cf.\ \citealt{DeRuiter:1989}). With their
parents they often speak \ili{Berber} or (dialectal) Arabic, or they
code-switch between one of these languages and \ili{Dutch}. Therefore,
\ili{Berber} and \ili{Arabic} can be considered heritage languages (cf.\
\citealt{Montrul:2016}).

The total corpus consists of roughly thirteen hours of interviews with
thirty-one people (see the Appendix for a full overview of speaker
codes we use in our examples, including interview settings and language
backgrounds, based on \citealt{Mourigh:fc}). The interviews were conducted in
groups of at least two people with the interviewer always present. All
interviews were conducted with male teenagers except for two teenage girls who
have the same ethnic background. The teenagers share a similar socio-economic
and educational background. At the time of recording they either attended
secondary school (VMBO) or lower vocational training (MBO). The interviews were
conducted at different places in informal settings such as the hallway of a
sports club, a cultural centre, close to the school and in the town centre.
All interviews were conducted in \ili{Dutch} with occasional code-switching to Berber
or Arabic.

The interviews inevitably suffer from the observers' paradox, and even though
the interviewer shares the ethnic background of the interviewees, he does not
share other characteristics such as age and place of residence. The interviewer
had the impression that many interviewees were quite comfortable. However, the
lack of certain lexical elements, such as \ili{Berber} and \ili{Arabic}
discourse markers, which are typical for Moroccan \ili{Dutch} discourse
indicate that their speech was somewhat influenced \citep{Kossmann:fc}. This
might also be a reason for the infrequent occurrence of V3\is{V3 word order}
order in the corpus. In general, even in the corpora of other \ili{Germanic}
urban varieties, V3\is{V3 word order} occurrences are quite rare, both in
interviews and in self-recordings (cf.\ \citealt{Ganuza:2008}).

In addition to the corpus, from which most of the examples were drawn, some
data originate from videoclips that Moroccan \ili{Dutch} youngsters themselves put on
YouTube.\footnote{Data taken from videos on the following channels:
    \url{https://www.youtube.com/watch?v=acFL0W3Y1ZY} and
\url{https://www.youtube.com/user/Youstoub}, last accessed on 13 December
2017.} These are not from Gouda and therefore indicate that it is a more
widespread phenomenon.

%%%%%%%%%%%%%%%%%%%%%%%%%%%%%%%%%%%%%%%%%%%%%%%%%%%%%%%%%%%%%%%%%%%%%%%%
\section{Describing the V3 data}
\label{sec:data}\is{V3 word order}
%%%%%%%%%%%%%%%%%%%%%%%%%%%%%%%%%%%%%%%%%%%%%%%%%%%%%%%%%%%%%%%%%%%%%%%%

\noindent In this section we present the data that show deviations from the
Standard \ili{Dutch} V2\is{V2 word order} pattern. We describe this data in terms of the initial
constituent (the \enquote{X} in XSV orders), the preverbal constituent (the
subject) and, finally, the distribution of possible V3\is{V3 word order} orders. Before moving on
to the aberrant V3\is{V3 word order} orders in these urban varieties, however, we must discuss
the superficial V3\is{V3 word order} orders that are in fact allowed in the Standard \ili{Dutch} V2\is{V2 word order}
grammar.

The occurrence of such V3\is{V3 word order} orders in our urban vernacular data would not be
unexpected if these sentences are acceptable in Standard \ili{Dutch}. Therefore
sentences like examples~\eqref{exHT-NW} and~\eqref{exHT-boeken} with hanging
topics are excluded:

\ea Standard \ili{Dutch}
    \ea
    \gll\label{exHT-NW}Noord-Wales, dat \textit{is} echt een mooie plek om op vakantie te gaan.\\
    {North Wales}, that is really a lovely place to on holiday to go.\Inf{}\\
    \trans \enquote*{North Wales, that}s a really lovely place to go on holiday.'
    \ex
    \gll\label{exHT-boeken}Die boeken, die \textit{moet} je zorgvuldig behandelen.\\
    those books those must you carefully treat.\Inf{}\\
    \trans \enquote*{As for those books, you should treat those with care.}
    \z
\z

\noindent \citet{GrecoHaegeman2020} discuss another type of V3\is{V3 word
order} order in Standard \ili{Dutch} that appears in the context of
circumstantial frame-setters.  Frame-setting topics are usually adjuncts in
sentence-initial position. They set the scene and/or delimit the space or time
in which the event described in the following comment takes place. These
frame-setters can be combined with non-subject initial orders or
non-declaratives, as shown in examples~\eqref{exFSanswer} and~\eqref{exFSprob},
respectively.

\ea Standard \ili{Dutch}
    \ea
    \gll\label{exFSanswer}Als je haar iets vraagt, nooit \textit{antwoordt} ze op tijd.\\
    if you her something ask.\Ssg{} never reply.\Tsg{} she on time\\
    \trans \enquote*{If you ask her something, she never replies on time.}
    \ex
    \gll\label{exFSprob}Als er morgen een probleem is, MIJ \textit{moet} je niet bellen.\\
    if there tomorrow a problem is me must you not call\\
    \trans \enquote*{If there is a problem tomorrow, don't call ME!}
    \z
\z

\noindent Because these are allowed in Standard
Dutch\footnote{\citet{GrecoHaegeman2020} note that sentences with
    subject-initial V3\is{V3 word order} orders and \emph{circumstantial} frame-setters are
    acceptable in the West-Flemish dialect of \ili{Dutch}, but not in Standard \ili{Dutch}.

\begin{exe}                   % sets up the top-level example environment
    \exi{(i)} OK in West-Flemish; but * in Standard \ili{Dutch}\\
    \gll \llap{*}Als mijn tekst klaar is, ik \textit{zal} hem opsturen.\\
    when my text ready is I shall it send\\
    \trans \enquote*{When my text is ready, I will send it.}
\end{exe}                   % sets up the top-level example environment

They argue, however, that these V3\is{V3 word order} orders systematically differ from the V3\is{V3 word order}
orders innovated by young \ili{Germanic} speakers in urban settings discussed in the
present paper. We will leave this discussion for future research.} as well,
this paper about the \ili{Dutch} youth varieties from Gouda is not concerned with
these types of V3\is{V3 word order} orders. In the following sections we will present the data
and describe their characteristics in terms of type of initial constituent,
preverbal constituent and distribution in a wider context.

%%%%%%%%%%%%%%%%%%%%%%%%%%%%%%%%%%%%%%%%%%%%%%%%%%%%%%%%%%%%%%%%%%
\subsection{The sentence-initial constituent}
\label{sec:dataic}
%%%%%%%%%%%%%%%%%%%%%%%%%%%%%%%%%%%%%%%%%%%%%%%%%%%%%%%%%%%%%%%%%%

There seems to be no categorial restriction on the initial constituent in the
Dutch urban vernacular dataset. There are determiner phrases (DPs),
prepositional phrases (PPs), adverbial phrases (APs) or entire clauses (CPs)
shown in examples \eqref{exDP1}, \eqref{exPP}, \eqref{exAP} and~\eqref{exCP}
respectively:

\ea
    \ea MD-A\label{exDP1}\\
    \gll Een keertje ik \textit{was} gewoon aan het fietsen\\
    one time I was just on the cycle.\Inf{}\\
    \trans \enquote*{One time I was just cycling.}
    \ex YouTube video Maisdokter\\
    \gll\label{exPP}Op een gegeven moment hij \textit{douwt} zo'n mais in zijn kont.\\
    at a given moment he pushes such.a corn.cob in his butt\\
    \trans \enquote*{At some point he pushes a corn cob in his butt.}
    \ex MD-I\\
    \gll\label{exAP}Hier je \textit{bent} verzekerd.\\
    here you are insured\\
    \trans \enquote*{Here you are insured.}
    \ex MD-B\\
    \gll\label{exCP}Wanneer we hem slaan, hij \textit{gaat} gelijk huilen.\\
    when we him beat he goes straight cry.\Inf{}\\
    \trans \enquote*{If we beat him he immediately starts to cry.}
    \z
\z

\noindent This lack of categorial preference for the sentence-initial
constituent corresponds to the V3\is{V3 word order} patterns found in urban varieties of
Norwegian, \ili{Swedish} and \ili{German}. \citet{Walkden:2017} illustrates this with
examples from Kiezdeutsch in particular, but the same seems to hold for the new
V3 patterns observed in \ili{Norwegian} and \ili{Swedish} urban youth varieties.

%%%%%%%%%%%%%%%%%%%%%%%%%%%%%%%%%%%%%%%%%%%%%%%%%%%%%%%%%%%%%%%%%%
\subsubsection{Sentence-initial frame-setters}
%%%%%%%%%%%%%%%%%%%%%%%%%%%%%%%%%%%%%%%%%%%%%%%%%%%%%%%%%%%%%%%%%%

Although our dataset is limited, we still find such categorial variety. All
these initial constituents are adjuncts indicating a specific time or location.
This is exactly what has been observed in other \ili{Germanic} urban youth varieties
(see \citealt[84]{Freywaldetal:2015} and \citealt{Walkden:2017}).
\citet{Freywaldetal:2015} characterise this type of initial constituent as
\enquote{an interpretational frame or anchor} for the immediately following
proposition.  This type of \enquote{frame-setter} (cf.\ \citealt{Chafe:1976})
thus provides a certain limitation in terms of time or
place.\footnote{\textcite{Freywaldetal:2015} add a \enquote{conditional}
    function to temporal or locational functions of these frame-setters.
    However, in light of the possible V3\is{V3 word order} orders with
    conditional frame-setters in Standard \ili{Dutch} discussed above, we leave
    the \enquote{conditional} specification in \ili{Dutch} urban vernaculars
out of the present discussion.} As \textcite{Walkden:2017} points out, it is
important to note that this type of frame-setter may also occur as the initial
constituent in regular V2\is{V2 word order} structures in the standard
varieties of \ili{Germanic} V2\is{V2 word order} languages.
Example~\eqref{exDP}, in Standard \ili{Dutch}, would have subject-verb inversion
as expected in V2\is{V2 word order} languages:\footnote{The use of the
diminutive \emph{keertje} `small time' is actually a further characteristic of
non-standard \ili{Dutch}.}

\begin{exe}                   % sets up the top-level example environment
\ex Standard \ili{Dutch}\\
\gll\label{exDP}Een keertje \textit{was} ik gewoon aan het fietsen\\
one time was I just on the cycle.\Inf{}\\
\trans \enquote*{One time I was just cycling.}
\end{exe}                   % sets up the top-level example environment

%%%%%%%%%%%%%%%%%%%%%%%%%%%%%%%%%%%%%%%%%%%%%%%%%%%%%%%%%%%%%%%%%%
\subsubsection{Other sentence-initial constituents}
%%%%%%%%%%%%%%%%%%%%%%%%%%%%%%%%%%%%%%%%%%%%%%%%%%%%%%%%%%%%%%%%%%

\noindent Apart from these adjuncts of time and location, there are some other
types of initial constituents in V3\is{V3 word order} structures in our dataset. These can be
grouped into three categories, which we briefly discuss below. These examples
are less straightforward, because the direct equivalent with subject-inversion
in Standard \ili{Dutch} does not exist. We therefore do not take these into
consideration in our analysis in \Cref{sec:ana}.

The first group consists of examples with \emph{omdat} `because', as shown
in~\eqref{exomdat} and~\eqref{exomdat2}:

\ea
    \ea MD-K\\
    \gll\label{exomdat}Omdat ik \textit{vind} het niet goed.\\
    because I find it not good\\
    \trans \enquote*{Because I don't think it's right.}
    \ex MD-K\\
    \gll\label{exomdat2}Omdat hij \textit{is} Marokkaan natuurlijk.\\
    because he is Moroccan obviously\\
    \trans \enquote*{Obviously because he is Moroccan.}
    \z
\z

\noindent These examples are difficult because \emph{omdat} introduces a
subordinate clause in Standard \ili{Dutch}. Subordinate clauses have SOV order and
therefore the Standard \ili{Dutch} equivalent of~\eqref{exomdat} and~\eqref{exomdat2}
would have SOV order following \emph{omdat}:

\ea Standard \ili{Dutch}
    \ea
    \gll \dots{} omdat ik het niet goed \textit{vind}.\\
    {} because I it not good find\\
    \trans \enquote*{\dots{} because I don't think it's right.}
    \ex
    \gll \dots{} omdat hij Marokkaan \textit{is} natuurlijk.\\
    {} because he Moroccan is obviously\\
    \trans \enquote*{\dots{} obviously because he is Moroccan.}
    \z
\z

\noindent In the examples from the \ili{Dutch} urban youth varieties dataset,
\emph{omdat} seems to behave like another \ili{Dutch} conjunction with the same
meaning: \emph{want} `because'. The conjunction \emph{want} is typically
followed by matrix-clause V2\is{V2 word order} syntax, as shown in example~\eqref{exwant}:

\ea Standard \ili{Dutch}\\
    \gll\label{exwant}Want ik \textit{vind} het niet goed.\\
    because I find it not good\\
    \trans \enquote*{Because I don't think it's right.}
\z

\noindent If the conjunction \emph{omdat} in the \ili{Dutch} urban youth varieties
indeed has the syntactic specifications of Standard \ili{Dutch} \emph{want}, the
superficial V3\is{V3 word order} order we observe here is not unexpected. If \emph{want} is
followed by subordinate-clause syntax, not the lack of V2\is{V2 word order} with
subject-inversion, but the lack of SOV order is unexpected. According to
\citet[123--125]{Zwart:2011}, \emph{omdat} can be followed by V2\is{V2 word order} in the
contexts of bridge verbs like \emph{zeggen} `to say' as well. We therefore do
not consider \emph{omdat}-clauses in our urban varieties corpus as part of
our proper V3\is{V3 word order} dataset. We will briefly discuss the implications for subordinate
clauses in \Cref{sec:fut} below.

%\largerpage[2]
The second group of examples with superficial V3\is{V3 word order} orders in the \ili{Dutch} urban
youth varieties involve code-switching from \ili{Dutch} to \ili{Berber} and/or Arabic.

\ea
    \ea MD-E\\
    \gll he, weet je, bhal jij \textit{gaat} naar hun\\
    hey know you bhal you go to them\\
    \trans \enquote*{Hey, you know, \emph{bhal} you go to them.}
    \ex MD-I\\
    \gll eentje hoor je van die: qa ik \textit{heb} vandaag uh\\
    one hear you of those qa I have today uh\\
    \trans \enquote*{You hear one of those: \emph{qa} I have today uh}
    \z
\z

\noindent There are also examples of code-switches or Arabic/Berber
interjections with V2\is{V2 word order} and the expected subject-verb in the urban youth
varieties, as shown in example~\eqref{excs3}.

\ea From YousToub channel\\
    \gll\label{excs3}En inshallah \textit{haal} je goede cijfers.\\
    and inshallah get you good grades\\
    \trans \enquote*{And, \emph{inshallah}, you'll get good grades.}
\z

\noindent These sentences with \ili{Berber} or \ili{Arabic} discourse markers,
however, cannot be compared to Standard \ili{Dutch} either; we leave them out
of the present analysis.

Finally, there is one category of adverbials that do not normally occur in
sen\-tence-initial position in Standard \ili{Dutch}, but that do occur several
times in our dataset of superficial V3\is{V3 word order} orders in the
\ili{Dutch} urban youth varieties:

\ea
    \ea MD-L\\
    \gll zogenaamd je \textit{hebt} geen geld meer\\
    as-if you have no money anymore\\
    \trans \enquote*{As if you no longer have any money (left).}
    \ex MD-R\\
    \gll \dots{} maar wel ik \textit{begrijp} alles.\\
    {} but still I understand everything\\
    \trans \enquote*{\dots but I do understand everything}
    \z
\z

\noindent The adverbs \emph{zogenaamd} `as-if' and \emph{wel} `still,
nonetheless' cannot occur in sen\-tence-initial position in Standard \ili{Dutch}. In
their Standard \ili{Dutch} equivalents, they would follow the inflected verbs, as
shown in examples~\eqref{exzoge} and~\eqref{exwel}, respectively:

\ea Standard \ili{Dutch}
    \ea
    \gll\label{exzoge}je \textit{hebt} zogenaamd geen geld meer\\
    you have as-if no money anymore\\
    \trans \enquote*{As if you no longer have any money (left).}
    \ex
    \gll\label{exwel}\dots{} maar ik \textit{begrijp} wel alles.\\
    {} but I understand still everything\\
    \trans \enquote*{\dots{} but I do understand everything}
    \z
\z

\noindent Again, because these sentence-initial constituents with superficial
V3 orders in our dataset do not have a direct equivalent, we cannot compare
them to Standard \ili{Dutch} V2\is{V2 word order}. We will exclude these from our analysis presented in
\Cref{sec:ana} below.

%%%%%%%%%%%%%%%%%%%%%%%%%%%%%%%%%%%%%%%%%%%%%%%%%%%%%%%%%%%%%%%%%%
\subsection{Preverbal constituent}
\label{sec:dataprev}
%%%%%%%%%%%%%%%%%%%%%%%%%%%%%%%%%%%%%%%%%%%%%%%%%%%%%%%%%%%%%%%%%%
%Pronouns, Noun phrases

\noindent The next crucial element in the superficial V3\is{V3 word order} orders is the
preverbal constituent. In Standard \ili{Dutch} V2\is{V2 word order} order, the preverbal constituent is
the sentence-initial constituent and it can be an argument or adjunct of a wide
variety of phrase types. The V3\is{V3 word order} orders in the \ili{Dutch} urban youth varieties
mostly exhibit arguments, or, more specifically, subject pronouns in all
persons and number, as shown in examples~\eqref{exsbjpro1},~\eqref{exsbjpro2}
and~\eqref{exsbjpro3}:

\ea
    \ea 24 maart interiew\\
    \gll\label{exsbjpro1}Soms ik \textit{gooi} iets op de grond.\\
    sometimes I throw something on the floor\\
    \trans \enquote*{Sometimes I throw something on the floor.}
    \ex MD-C\\
    \gll\label{exsbjpro2}\'e\'en keer we \textit{zaten} bij big Mo film te, televisie te kijken\\
    one time we sat.\Pl{} at big Mo film to tv to watch.\Inf{}\\
    \trans \enquote*{Once we were watching a film, tv at big Mo}s.'
    \ex MD-A\\
    \gll\label{exsbjpro3}Toen ze \textit{vroegen} ID.\\
    then they asked.\Pl{} ID\\
    \trans \enquote*{Then they asked for ID.}
    \z
\z

\noindent The second-person singular pronoun has stressed and unstressed
variants in Standard \ili{Dutch}: \emph{je} (unstressed) vs. \emph{jij}
(stressed). Both occur as the subject in our V3\is{V3 word order} dataset, as shown in
examples~\eqref{exAPje} (repeated from~\ref{exAP}) and~\eqref{exAPjij}:

\ea MD-I
    \ea
    \gll\label{exAPje}Hier je \textit{bent} verzekerd.\\
    here you are insured\\
    \trans \enquote*{Here you are insured.}
    \ex
    \gll\label{exAPjij}Daarna jij \textit{ging} mee.\\
    afterwards you went along\\
    \trans \enquote*{Afterwards you went along.}
    \z
\z

\noindent From a cross-linguistic perspective, the occurrence of the stressed
pronoun \emph{jij} `you' is unexpected. \citet[84]{Freywaldetal:2015} observe
that preverbal constituents in urban youth varieties in Germany, Norway or
Sweden are \enquote{virtually always unaccented} (see also
\citealt{Walkden:2017}).  Cross-linguistically, the preverbal element is usually
the subject of the clause, but as \textcite{Walkden:2017} points out, this is a
\enquote{strong tendency rather than a requirement}. In the \ili{Dutch} urban
youth varieties dataset, we also find some examples of non-pronominal subjects
in preverbal position:

\ea\label{ex:16.19}
    \ea\label{exsbjnp} MD-I\\
    \gll\label{exsbjnpa}vroeger mensen \textit{gingen} lopend\\
    in.the.past people went.\Pl{} on.foot\\
    \trans \enquote*{In the past people would go on foot.}
    \ex MD-I\\
    \gll\label{exsbjnpb}daarna die, die leraar \textit{heeft} niet meer lesgegeven\\
    afterwards that that teacher has no longer taught\\
    \trans \enquote*{Afterwards that, that teacher hasn't taught anymore.}
    \ex YouTube video Maisdokter\\
    \gll\label{exsbjnpc}Op een gegeven moment iemand \textit{zegt} tegen hem je moet naar Fez\\
    at a certain time someone says to him you must to Fez\\
    \trans \enquote*{At some point someone says to him: you must go to Fez.}
    \ex MD-I\\
    \gll\label{exsbjnpd}daarna de rest \textit{zegt} ik ga niet\\
    afterwards the rest says I go not\\
    \trans \enquote*{Afterwards the rest says: I'm not going.}
    \ex YousToub\\
    \gll\label{exsbjnpe}Vaak het probleem \textit{is} dat ze met de jaren verwachten ze meer.\\
    often the problem is that they with the years expect.\Pl{} they more\\
    \trans \enquote*{Often the problem is that they -- as the years go by -- expect more.}
    \z
\z

\noindent According to \textcite{Freywaldetal:2015}, a common denominator of
these preverbal constituents lies in their information-structural nature: they
are all \emph{familiar topics} that refer to a contextually given or salient discourse
referent. Not all examples in the \ili{Dutch} urban youth varieties data presented in
\eqref{ex:16.19} contain familiar topics, however. The subjects of
examples~\eqref{exsbjnpa} and~\eqref{exsbjnpb} could indeed be argued to be linked
to the \emph{common ground}, either because they are generic concepts (like
\emph{mensen} `people') or because they have been explicitly mentioned in the
preceding discourse (like \emph{die leraar} `that teacher'). The teacher is the
topic of the preceding sentences (all in Berber), in which a boy is being
beaten by his teacher, but later comes back to seek revenge and hits the
teacher.

The subject of example~\eqref{exsbjnpc}, \emph{iemand} `someone', is technically
inert and would function more as a \emph{shift topic} than a familiar topic. The
referential status of the subject in~\eqref{exsbjnpd}, \emph{de rest} `the rest',
can be inferred from the context, but it clearly indicates a contrast between
this subject and the topic in the immediately preceding discourse.
Example~\eqref{exsbjnpe} is a copular\is{copulas} clause in which \emph{het
probleem} `the problem' in preverbal position could be argued to be the
predicate, with the \emph{dat}-clause as its subject. The analysis of these
types of copular\is{copulas} clauses goes beyond the scope of the present
paper, but the fact that a noun phrase like \emph{het probleem} `the problem'
can occupy the preverbal position cannot be ignored. This phrase is certainly
not a familiar topic. We will come back to these subtle information-structural
differences in \Cref{sec:ana} below.

%%%%%%%%%%%%%%%%%%%%%%%%%%%%%%%%%%%%%%%%%%%%%%%%%%%%%%%%%%%%%%%%%%%%%%%%
\subsection{Distribution of V3\is{V3 word order} orders}
\label{sec:datadis}
%%%%%%%%%%%%%%%%%%%%%%%%%%%%%%%%%%%%%%%%%%%%%%%%%%%%%%%%%%%%%%%%%%%%%%%%%%%%%%%%
%V2-V3 in sequential sentences, V3\is{V3 word order} as option, not the norm, no elicited data on the limitations of V3\is{V3 word order}

The V3\is{V3 word order} orders in our data do not occur in every main clause. Just like in other
Germanic urban youth varieties, the V3\is{V3 word order} orders are optional deviations from the
regular V2\is{V2 word order} patterns. V3\is{V3 word order} orders can be found immediately preceding or following
regular V2\is{V2 word order} sentences uttered by the same speaker in the same type of context.
Example~\eqref{exdisV3} immediately follows another clause with the same
sentence-initial constituent \emph{toen} `then'. The first clause exhibits
regular V2\is{V2 word order} order, whereas the second clause is V3\is{V3 word order}:

\ea\label{exdisV3} MD-A\\
    \gll Toen \textit{gingen} we wegrennen. Toen ze \textit{vroegen} ID.\\
    then went.\Pl{} we run.away.\Inf{} then they asked.\Pl{} ID\\
    \trans \enquote*{Then we ran away. Then they asked for ID.}
\z

\noindent The V3\is{V3 word order} orders do not occur very often and when they do, they are
found alongside very similar sentences with Standard \ili{Dutch} V2\is{V2 word order} order. Since our
current data consists of non-elicited sentences only, we cannot check the
(un)grammaticality of certain types of V3\is{V3 word order} orders in different contexts. This is
difficult to verify in general, because we are dealing with a non-standard
variety of the language which is subject to stylistic variation. The young
people who speak this variety often change to Standard \ili{Dutch} in the presence of
people who are not from their peer group.

\citet[109--130]{Ganuza:2008} discusses the same sociolinguistic conditions for
her focus group speaking \ili{Swedish} urban varieties. \citet{Walkden:2017}, based on
previous work on Kiezdeutsch by Wiese and \ili{Swedish} urban varieties by Ganuza,
notes that there are three contexts in which these types of V3\is{V3 word order} orders are not
allowed. These are sentences in which the preverbal constituent is the object
(rather than the subject), \emph{wh}-interrogatives and subordinate clauses.
All examples in our current urban vernacular dataset of \ili{Dutch} have preverbal
subjects and none of the examples are wh-interrogatives. This might be due to a
limited dataset, but since these options seem to be excluded in other urban
vernaculars, the same generalisation might hold for the \ili{Dutch} urban vernacular.
We have already briefly mentioned our examples with subordinate clauses
introduced by \emph{omdat} `because'. \citet{Walkden:2017} notes that there are
occasional examples of V3\is{V3 word order} in clauses introduced by the \ili{German} \emph{weil}
`because', but that \enquote{this is a context in which it is well known that
main clause word order may occur in colloquial usage} \citep{Walkden:2017},
which is reminiscent of the above-mentioned \emph{omdat}-clauses in \ili{Dutch} we
left out of our proper V3\is{V3 word order} dataset for now (see also
\citealt{AntomoSteinbach:2010} and \citealt{Reis:2013}).

%%%%%%%%%%%%%%%%%%%%%%%%%%%%%%%%%%%%%%%%%%%%%%%%%%%%%%%%%%%%%%%%%%%%%%%%
\section{Analysis}
\label{sec:ana}
%%%%%%%%%%%%%%%%%%%%%%%%%%%%%%%%%%%%%%%%%%%%%%%%%%%%%%%%%%%%%%%%%%%%%%%%

%blurb about synchronic and diachronic analysis
%
\noindent Although our current dataset is still fairly limited, we will attempt
to offer a preliminary synchronic analysis of these V3\is{V3 word order} orders in \ili{Dutch} urban
youth varieties. Until we collect more data, this analysis is necessarily
preliminary, but it will help our attempts to sketch a diachronic analysis of
ongoing syntactic change in \ili{Dutch}.

%%%%%%%%%%%%%%%%%%%%%%%%%%%%%%%%%%%%%%%%%%%%%%%%%%%%%%%%%%%%%%%%%%%%%%%%
\subsection{Synchronic analysis}
\label{sec:ana1}
%%%%%%%%%%%%%%%%%%%%%%%%%%%%%%%%%%%%%%%%%%%%%%%%%%%%%%%%%%%%%%%%%%%%%%%%

It is important to emphasise that the synchronic analysis of the V3\is{V3 word order} patterns
should be compatible with a V2\is{V2 word order} grammar as well, because these V3\is{V3 word order} orders are
only \emph{optional} variants of the Standard \ili{Dutch} V2\is{V2 word order}. In other words, all
speakers with innovative V3\is{V3 word order} patterns also (indeed, mostly) utter V2\is{V2 word order} sentences
that are the norm in Standard \ili{Dutch}. Although the V2\is{V2 word order} constraint observed in
various languages shares two crucial characteristics (verb-movement to the
C-layer accompanied by the merger of a phrasal constituent, cf.\
\citealt{Holmberg:2013} and \citealt{Wolfe:2015b}), V2\is{V2 word order} languages can differ in the
way they exhibit these characteristics. Apart from a traditional distinction
based on whether V2\is{V2 word order} is limited to main clauses (as in \ili{Dutch}, \ili{German} and
Mainland Scandinavian) or  appears in subordinate clauses as well (as in
\ili{Icelandic} or \ili{Yiddish}) (cf.\ \citealt{Holmberg:2013}), languages also appear to
differ in terms of their CP structure.

Recently, the typology of different types of V2\is{V2 word order} languages was further developed
by \citet{Wolfe:2017} on the basis of the availability of pro-drop and optional
V3 orders. In this typology of V2\is{V2 word order} languages,
\textcite[31]{Wolfe:2017} distinguishes three types of V2\is{V2 word order}
systems named after the landing site of the verb, based on the landing site of
the finite verb (Fin or Force):

\begin{description}

    \item[Fin-V2:] Frame-setter + topic\is{topic} + focus\is{focus} (\ili{Old English},
        \ili{Middle Low German}, etc.)

    \item[Force-V2 system 1:] Frame-setter + topic/focus (Later
        \ili{Old French}, \ili{Spanish}, etc.)

    \item[Force-V2 system 2:] Frame-setter\tss{HT/LD} + topic/focus\is{focus}
                              (Modern \ili{Dutch} and \ili{German}, etc.)

\end{description}

\noindent Standard \ili{Dutch} is classified by \citet{Wolfe:2017} as a
\enquote{Force-V2 system 2} language, because regarding V3\is{V3 word order} orders, Standard
Dutch can only accommodate hanging (\glsunset{HT}\gls{HT}) or
left-dislocated\is{left dislocation}
(\glsunset{LD}\gls{LD}) topics\is{topic} as a sentence-initial constituent.
V3/XSV orders found in urban youth varieties are ungrammatical in the standard
language.

\ea
    \ea\label{exdisV3-ht} Standard \ili{Dutch} -- HT\\
    \gll\label{exHT-NWa}Kaapstad, dat \textit{is} echt een mooie plek om op vakantie te gaan.\\
    {Cape Town}, that is really a lovely place to on holiday to go.\Inf{}\\
    \trans \enquote*{Cape Town, that's a really lovely place to go on holiday.}
    \ex Standard \ili{Dutch} -- *V3, but probably OK in urban varieties\\
    \gll \llap{*}In de zomer Kaapstad \textit{is} echt een mooie plek om op vakantie te gaan.\\
    in the summer {Cape Town} is really a lovely place to on holiday to go.\Inf{}\\
    \trans intended: \enquote*{In summer, Cape Town is really a lovely place to go on holiday.}\label{exHT-NW2}
    \z
\z

\noindent The Standard Modern \ili{Dutch} V2\is{V2 word order} order with
V-to-Force movement is shown in \eqref{exv2SD1}:

\begin{multicols}{2}
\ea\label{exv2SD1}
\begin{tikzpicture}[baseline=(forcep.base)]%(current bounding box.north)]%-.79\baselineskip]
\tikzset{every tree node/.style={align=center,anchor=north}}
    \Tree
        [.\node(forcep){ForceP};
    		[.SpecForce\\\emph{Toen} ]
    		[.Force'
                \node(v2){Force\\\emph{vroegen$_i$}};
                [.FinP \edge[roof]; \node(t){\emph{ze t$_i$ ID}}; ] ] ]% ] ]
    \draw[arrow, bend left=45] (t.south) to (v2.south);
\end{tikzpicture}
\ex\label{exsbjpro4} Standard \ili{Dutch}\\
    \gll Toen \textit{vroegen} ze ID.\\
    then asked.\Pl{} they ID\\
    \trans \enquote*{Then they asked for ID.}
\z
\end{multicols}

\noindent As described in \Cref{sec:dataic} above, the sentence-initial
constituents in the superficial V3\is{V3 word order} orders in \ili{Germanic} urban youth varieties
function as a frame- or scene-setter. The initial constituents are not
arguments, but adjuncts with a temporal or locational meaning such as
\emph{toen} `then', \emph{een keer} `one time' or \emph{hier} `here'. The
superficial order of constituents in these sentences is thus: Frame -- Subject
-- Verb. In line with \textcite{Walkden:2017}, we assume general \is{verb movement}V-to-C movement
in standard modern \ili{Germanic} V2\is{V2 word order} clauses in general and therefore Standard \ili{Dutch}
as well. If the inflected verb moves to a C-head and the subject moves to its
specifier, the easiest analysis for the urban vernacular V3\is{V3 word order} sentences would
involve an extra structural layer to host this frame-setting sentence-initial
constituent. Independent evidence for extra structural layers in the C-domain
is abundantly found in \ili{Romance} languages, upon which \citet[283]{Rizzi1997}
based his split CP:\is{topic}\is{focus}

\ea
    {}[Frame\dots{} [Force\dots{} [Topic\dots{} [Focus\dots{} [Fin\dots{} [TP\dots{} ]]]]]
\z

\noindent Variations on this were further developed by
\citet[71]{BenincaPoletto:2004} and by
\citet[112--113]{FrascarelliHinterholzl:2007}, who later apply this to early
\ili{Germanic} \parencite{HinterholzlPetrova:2009}:

\ea ForceP $>$ ShiftP $>$ ContrP $>$ FocP $>$ FamP* $>$ FinP
\z

\noindent As \citet{Roberts:1996} already observed, analysing V3\is{V3 word
order} orders in Old English, we need to postulate at least one extra layer in
the CP if we assume V-to-C movement always occurs in these V2\is{V2 word order}
languages. \textcite{Roberts1996} assumed a distinction between Fin and
Focus/Force as the landing site of the finite verb in these cases. Until we
have evidence for a further split, we will assume a simple split of the CP into
two layers. Note that the so-called \enquote{bottle-neck effect} in strict
V2\is{V2 word order} languages like Standard \ili{Dutch} and German uses
locality to prevent movement of more than one constituent into the C-domain
(cf.\ among others \citealt{Roberts:2004} and \citealt{Mohr:2009}). From this
perspective a V2\is{V2 word order} language with multiple constituents in the
C-domain is unexpected and needs to be explained. We follow
\citegen{Walkden:2017} assumption, based on earlier work by
\textcite{Rizzi1997} and \citet{Haegeman:1995}, which states that certain heads
may be associated with criteria requiring them to enter into a spec-head
configuration with an appropriate XP\@. This then motivates
interpretively-driven movement such as topicalisation\is{topicalization},
focalisation\is{focalization}, wh-questions, etc. Languages with syncretised
left peripheries, such as Standard \ili{Dutch}, only allow one criterion to be
active, resulting in the movement of one (and only one) constituent to the
C-domain.  With \textcite{Walkden:2017}, we assume that V3\is{V3 word order}
orders arise when not one but two of these criteria are to be satisfied.

Since the sentence-initial constituent in \ili{Dutch} urban youth varieties is
always clearly a frame- or scene-setter, it seems appropriate to add an
additional FrameP on top of the Standard \ili{Dutch} ForceP to accommodate the
V3\is{V3 word order} orders in urban youth varieties. Compare
example~\eqref{exv2SD1} above to the innovative V3\is{V3 word order} option
from our dataset of \ili{Dutch} urban youth varieties with similar \is{verb
movement}V-to-Force movement, but an added FrameP to host the temporal
frame-setter \emph{toen} `then' in~\eqref{exv3UD}:\largerpage[3]

\begin{multicols}{2}\raggedcolumns
\ea\label{exv3UD}
\begin{tikzpicture}[baseline=(root.base)]%-.79\baselineskip]%,tree,scale=1.0,every tree node/.style={scale=1}]
\tikzset{every tree node/.style={align=center,anchor=north}}
\Tree
    [.\node(root){FrameP};
	[.{\emph{Toen}} ]
	[.ForceP
		[.SpecForce\\\emph{ze} ]
		[.Force'
            \node(v2){Force\\\emph{vroegen$_i$}};
            [.FinP \edge[roof]; \node(t){\emph{t$_i$ ID}}; ] ] ] ] ]
    \draw[arrow, bend left] (t.west) to (v2.south) ;
\end{tikzpicture}\columnbreak
\ex\label{exsbjpro4-ft} Standard \ili{Dutch} -- familiar topic\\
    \gll Toen ze \textit{vroegen} ID.\\
    then they asked.\Pl{} ID\\
    \trans \enquote*{Then they asked for ID.}
\z
\end{multicols}

\noindent Wolfe's typology assumes a cartographic CP-structure based on
\citet{Rizzi1997} with a FrameP on top of ForceP, followed by TopP, FocP and
FinP. Since urban youth varieties of \ili{Dutch} allow various kinds of frame-setters
(e.g.\ \emph{daarna} `afterwards', \emph{soms} `sometimes', etc.) and only one
preverbal topic/focus, the grammar of these varieties can therefore be best
described as \enquote{Force-V2 system 1} in Wolfe's typology. Speakers
with optional V3\is{V3 word order} orders have access to two registers of \ili{Dutch}: Standard \ili{Dutch}
with strict V2\is{V2 word order} (\enquote{Force-V2 system 2}) and urban varieties with optional
additional frame-setters (\enquote{Force-V2 system 1}). We assume that
style-shifting occurs in more formal contexts, e.g.\ writing, speaking to
non-peers, etc.  Wolfe's V2\is{V2 word order} typology is ultimately a diachronic typology. In
the next section, we will turn back to his typology in the light of our
diachronic analysis.

%%%%%%%%%%%%%%%%%%%%%%%%%%%%%%%%%%%%%%%%%%%%%%%%%%%%%%%%%%%%%%%%%%%%%%%%
\subsection{Diachronic analysis}
\label{sec:ana2}
%%%%%%%%%%%%%%%%%%%%%%%%%%%%%%%%%%%%%%%%%%%%%%%%%%%%%%%%%%%%%%%%%%%%%%%%
%Walkden's scenario from CP-V2 > L2 SVO > L1 Split CP
%Wolfe's cline of V2\is{V2 word order}: Fin-V2 > Force-V21 > Force-V2 2
%Analysis for \ili{Dutch}: failed \is{verb movement}V-to-C unlikely for Moroccan L1s? position of the V in Berber/Arabic also in C presumably

\ili{Old English} was already analysed as a V2\is{V2 word order} language by
\citet{VanKemenade:1987}.  In 1996, Ian Roberts makes inferences based on this
and work on \ili{Gothic} by, among others, \citet{Kiparsky:1994} and observed
that ``residual V2'' in Present-day \ili{English} is a misleading term for the
actual state of affairs.  Comparing characteristics of \ili{Old English}
V2\is{V2 word order} and V3\is{V3 word order} orders, it appears that ``Full
V2'' of Modern German and \ili{Dutch} is better described as an innovation: a
stage of \enquote{strict V2} that \ili{English} has never reached.
\textcite{Roberts1996} suggests that the V2\is{V2 word order} and V3\is{V3 word
order} orders in Old English can be analysed with a ``split-Comp'' structure
allowing multiple landing sites for the verb in the left periphery.

To our knowledge, \citegen{Walkden:2017} paper on \ili{Germanic} urban youth
varieties (or \enquote{urban vernaculars} as he calls them) presents the only
comprehensive diachronic analysis of these innovative types of V3\is{V3 word
order} orders. In addition to the urban vernacular data, he draws on insights
from, among others, \textcite{Roberts1996} to develop a similar account for the
situation in Old English. Walkden's analysis is based on a scenario of
imperfect L2 acquisition\is{language acquisition} of the standard V2\is{V2 word
order} language by speakers from a different linguistic background (e.g.\
immigrants from Turkey, Morocco, etc.\ moving to Germany, or, in our case, the
Netherlands). He proposes three separate stages for the development of optional
V3\is{V3 word order} orders (cf.\ \citealt{Walkden:2017}):\largerpage

\begin{description}
    \item[Stage 1:] L2 learners of standard \ili{Germanic} V2\is{V2 word order} fail to acquire
        \isi{verb movement} to C, resulting in SVO orders

    \item[Stage 2:] L1 learners (e.g.\ children of first-generation
        immigrants) attempt to reconcile mixed input of SVO and \is{verb movement}V-to-C,
        resulting in a split-CP (CP1 \& CP2) that allows for the observed
        optional V3\is{V3 word order} structures in the urban vernaculars

    \item[Stage 3:] V3\is{V3 word order} structures are propagated across communities and
        successive generations increase their use
\end{description}

\noindent These diachronic developments are straightforward and they fit the
overall sociolinguistic situation with first- and second-generation immigrants
in the Netherlands as well. Through socio-historical circumstances, certain
areas of the country had a high proportion of L2 learners. Let us go through
the implications for the analysis of the \ili{Dutch} urban vernacular V3\is{V3 word order} sentences
stage by stage.

\textit{Stage 1} of the analysis hinges on the failure of the acquisition\is{language acquisition} of
\isi{verb movement} to C. This is necessary for the subsequent stage in which the
second generation attempts to make sense of a mixed SVO/V-to-C input. The
question is whether this scenario of failure of the acquisition\is{language acquisition} of \is{verb movement}V-to-C
movement is likely for the Moroccan immigrants in the Netherlands. The native
language of this first-generation L2 learners is \ili{Berber} or Moroccan Arabic,
although all of them have a good understanding of Standard \ili{Arabic} as well. Both
Berber and \ili{Arabic} are VSO languages with optional SVO orders. Verb movement in
pragmatically neutral matrix clauses in these languages is usually argued to be
limited to \is{verb movement}V-to-T or \is{verb movement}V-to-AgrSP (cf.\ amongst others \citet{Benmamoun:1992a},
\citet{Jouini:2014} and \citet{Shlonsky:2000} for \ili{Arabic} and \citet{Choe:1987}
for Berber). In both languages, sentence-initial frame-setters can occur with
following VSO orders as well. In a corpus study of child-directed \ili{Dutch},
\citet{MacWhinneySnow:1985} observed that only 23\% of the input was
non-subject initial. Although this is apparently enough for \ili{Dutch} L1 learners
to acquire the V2\is{V2 word order} constraint (see also \citet[114]{Yang:2000} for a full
discussion), L2 learners might initially interpret the non-subject initial
orders in a way that is compatible with the grammar of their first language. We
would thus hypothesise that they do not postulate a phi-probe in the C domain
resulting in \is{verb movement}V-to-C movement because they do not require this phi-feature on C
to yield XVS orders in their native language. With the next generation, they
use their mixed input, leading to Stage 2 in Walkden's proposal. Although at
home they might also speak \ili{Berber} or Moroccan \ili{Arabic}, \ili{Dutch}
is frequently used in the Moroccan community; there are multiple dialects and
languages that are not always mutually intelligible. Since our current number
of examples of V3\is{V3 word order} order are still fairly limited and we have
not collected any specific acquisitional data of these L2 learners yet, we
leave a further exploration of this hypothesis for future research.

Assuming \textit{Stage 1} has resulted in the failed acquisition\is{language acquisition} of \is{verb movement}V-to-C
movement, in \textit{Stage 2} the next generation consisting of L1 learners of
Dutch attempt to reconcile their mixed SVO/V2 input. They acquire \is{verb movement}V-to-C
successfully and their language, the urban youth variety under discussion, has
a V2\is{V2 word order} grammar. To reconcile this V2\is{V2 word order} grammar with the SVO input as well, they are
forced to postulate a split of the CP to accommodate additional frame-setters.

In \textit{Stage 3} this split is then postulated to be propagated throughout
the community. The V3\is{V3 word order} orders in our data are not limited to a
single speaker, but found in interviews with various teenagers from Gouda. In
addition to this, we found several examples of these V3\is{V3 word order}
innovations in YouTube videos of young speakers with a Moroccan heritage from
other parts of the country. This is a clear indication that the new split-CP
grammar has spread amongst teenagers with a Moroccan background in the
Netherlands at the very least. The young people with optional V3\is{V3 word
order} orders seem to be aware of the fact that this grammar is associated with
a specific register, as they are able to switch to a purely V2 grammar in
formal contexts or simply when talking to \ili{Dutch} speakers outside of their
Moroccan \ili{Dutch} community.\footnote{As we have only collected data from
    young people with a Moroccan background, at this stage we cannot comment on
    how widespread this phenomenon is outside the Moroccan community in the
Netherlands. In addition, more data is needed on the socio-linguistic
parameters associated with the possible switch in register. This, however, goes
beyond the scope of the present paper and we leave this for future research.}

%%%%%%%%%%%%%%%%%%%%%%%%%%%%%%%%%%%%%%%%%%%%%%%%%%%%%%%%%%%%%%%%%%%%%%%%
\subsubsection{V3 innovations in a diachronic typology of V2}
%%%%%%%%%%%%%%%%%%%%%%%%%%%%%%%%%%%%%%%%%%%%%%%%%%%%%%%%%%%%%%%%%%%%%%%%

Recall Wolfe's typology of V2\is{V2 word order} languages from \Cref{sec:ana1}, which we present in
\figref{fig:16:1}.\largerpage[-2]

\begin{figure}[h]
\caption{V3* in V2\is{V2 word order} languages \parencite[31]{Wolfe:2017}\label{fig:16:1}}
\begin{tikzpicture}[node distance=.66cm, text width=3.5cm,
                    every node/.style={%
                        draw,
                        align=center,
                        minimum height=5.75cm,
                        rounded corners=8pt,
                        font=\small,
                        inner sep=1mm}]

    \node (fin) {%
    \begin{minipage}[t][4.5cm]{3.5cm}
        \centering
        \textit{Fin-V2 system}\\[\baselineskip]
        Frame-setter + topic + focus\\[\baselineskip]
        \emph{Early Medieval \ili{Romance},
        Later Old Occitan and Sicilian;
        \ili{Middle Low German},
        Early \ili{Old High German}, \ili{Old English}}
    \end{minipage}
    };

    \node [right=of fin] (force) {%
    \begin{minipage}[t][4.5cm]{3.5cm}
        \centering
        \textit{Force-V2 system 1}\\[\baselineskip]
        Frame-setter + topic/focus\\[\baselineskip]
        \emph{Later \ili{Old French}, \ili{Spanish}, \ili{Venetian};
            Later Old\il{Old High German}, Middle\il{Middle High German}, New
            High \ili{German};
        Sumeiran and Vallader Rhaeto Romance}
    \end{minipage}
    };

    \node [right=of force] (force2) {%
    \begin{minipage}[t][4.5cm]{3.5cm}
        \centering
        \textit{Force-V2 system 2}\\[\baselineskip]
        Frame-setter$_{\textsc{ht/ld}}$ + topic/focus\\[\baselineskip]
        \emph{Modern \ili{German}, \ili{Dutch};
        San Leonardo Rhaeto-Romance}
    \end{minipage}
    };

    \draw [thin, ->, shorten <=.5mm, shorten >=.5mm] (fin.east) to (force.west);
    \draw [thin, ->, shorten <=.5mm, shorten >=.5mm] (force.east) to (force2.west);

\end{tikzpicture}
\end{figure}

\noindent \citet{Wolfe:2017} argues that older \ili{Germanic} varieties provide more
options for V3\is{V3 word order} orders. Early Medieval \ili{Romance} and Early \ili{Old High German} allowed
both topics and foci in sentence-initial position and are thus classified as a
\enquote{Fin-V2} system. In later \ili{Old French} and Spanish and New High \ili{German},
on the other hand, only a frame-setter and either a topic\is{topic} or a focus\is{focus}
constituent was found sentence-initially, making them \enquote{Force V2\is{V2 word order} system 1}
languages.  In both \ili{Germanic} and \ili{Romance}, Wolfe thus observes a change from
Fin-V2 to Force-V2 (and within Force-V2 from system 1 to system 2, which
ultimately happened in Modern \ili{Dutch} and \ili{German}).

From this perspective, the optional V3\is{V3 word order} orders in the
\ili{Dutch} urban varieties could indicate that this variety of \ili{Dutch} is
in transition (again) from a Force-V2 system 2 (back) to system 1. Would this
typology be appropriate for the scenario of language contact and change
proposed by \citet{Walkden:2017}? A crucial aspect of Walkden's scenario is
that the CP cannot be split in the standard V2\is{V2 word order} language. The
simple non-cartographic synchronic analysis with a single CP in Standard
\ili{Dutch} splitting into a CP1 and CP2 would therefore work.  In the grammar
of \ili{Dutch} urban youth varieties, the outer CP2 is reserved for any type of
frame-setter and the inner CP1 hosts the verb and any type of preverbal
constituent. These labels need no further specification, although the outer CP2
could be seen as a FrameP since it always hosts a frame- or scene-setter. This
consistency provides a good argument for the  mapping of information-structural
features to a further-defined hierarchical structure in the left periphery, at
least for FrameP and ForceP.

If we were to assume the CP of Standard \ili{Dutch} is already split into
further layers of ForceP, FocP, FinP etc.\ and we thus take a cartographic
approach, Walkden's diachronic scenario can only work if the verb in Standard
\ili{Dutch} is in the left-most possible position. If the verb were in a lower
position, the need to postulate more structure to reconcile the SVO/V2 input
would not arise, so the split sketched by Walkden would not be motivated. The
left-most position would be Force in a \enquote{ForceP system 2} type of
language, which is indeed the position in which the verb lands according to
\citet{Wolfe:2017}. If Walkden's scenario is correct this implies there might
be diachronic evidence in addition to Wolfe's synchronic V3\is{V3 word order}
analysis to motivate \is{verb movement}V-to-Force movement in Standard
\ili{Dutch}. The forced split of the CP (or ForceP) Walkden describes could
result in the creation of extra structure in the form of a FrameP that can host
any type of frame-setter in a \enquote{Force-V2 system 1} type of grammar.

\textcite{Walkden:2017}, however, suggests this split CP conflates
information-struc\-tural layers as follows:

\begin{itemize}

    \item CP2 = ForceP, ShiftP, ContrP and FocP (for sentence-initial
        frame-setters)

    \item CP1 = FamP and FinP (for preverbal subjects)
\end{itemize}

\noindent CP2 does not include FrameP in this system, forcing the
sentence-initial frame-setter to occur lower in the structure, in ForceP,
ShiftP or ContrP. CP1 is reserved for FamP and FinP as these host the preverbal
subject that are (almost) always familiar topics in the data Walkden discusses.
Recall, however, that preverbal subjects in \ili{Dutch} urban varieties are not
always familiar topics:

\ea Standard \ili{Dutch}
    \ea Shift topic\\
    \gll\label{eexsbjnpc}Op een gegeven moment iemand \textit{zegt} tegen hem je moet naar Fez\\
    at a certain time someone says to him you must to Fez\\
    \trans \enquote*{At some point someone says to him: you must go to Fez.}
    \ex Contrastive topic\\
    \gll\label{eexsbjnpd}daarna de rest \textit{zegt} ik ga niet\\
    afterwards the rest says I go not\\
    \trans \enquote*{Afterwards the rest says: I'm not going.}
    \ex Shift topic?\\
    \gll\label{eexsbjnpe}Vaak het probleem \textit{is} dat ze met de jaren verwachten ze meer.\\
    often the problem is that they with the years expect.\Pl{} they more\\
    \trans \enquote*{Often the problem is that they -- as the years go by -- expect more.}
    \z
\z

\noindent These types of contrastive or shift topics in preverbal position
would be in CP2 in Walkden's split CP if we take the information-structural
labels of the split CP seriously. Walkden's mechanism of change can thus only be
extended to the \ili{Dutch} urban varieties if the CP is split differently. We
therefore propose the following split:

\begin{itemize}

    \item CP2 = FrameP (for sentence-initial frame-setters)

    \item CP1 = ForceP (for preverbal subjects with any information-structural
        status)

\end{itemize}

To conclude, we adopt Walkden's diachronic scenario resulting in a situation in
which second-generation L1 speakers of \ili{Dutch} solve their ambiguous SVO/V2 input
by creating additional structure in the C-domain. If we confine ourselves to an
analysis of \ili{Dutch} only, it would suffice to postulate a single CP in Standard
Modern \ili{Dutch} that is subsequently reanalysed by the speakers of urban youth
varieties as a simply binary split into CP1 and CP2. From a cross-linguistic
perspective, however, it might be desirable to adopt a cartographic layering of
the CP that can account for the observed differences in terms of pro-drop,
optional V3\is{V3 word order} orders and the landing site of the verb, as proposed by
\citet{Wolfe:2017}. If we combine Walkden's diachronic scenario with
\citegen{Wolfe:2017} typology of V2\is{V2 word order} grammars, the \ili{Dutch} urban youth varieties are moving away
from a \enquote{Force-V2 system~2} (Standard Modern Dutch) to a \enquote{Force-V2 system~1}
with an additional FrameP. Although Wolfe's typology is also based on
diachronic syntactic changes,  both the \ili{Romance} and \ili{Germanic} languages he
studied have moved from \enquote{Fin-V2} to \enquote{Force-V2 system 1} and, in
the case of \ili{Dutch} and \ili{German}, all the way to \enquote{Force-V2 system 2}. The
innovative V3\is{V3 word order} orders in urban youth varieties present an interesting case of
syntactic change in the opposite direction, i.e.\ from \enquote{Force-V2 system
    2} to \enquote{Force-V2 system 1}.\footnote{A reviewer speculates this type of
    change in the opposite direction might be associated with language contact
    and L2 acquisition\is{language acquisition}, whereas change from \enquote{Force-V2 system 1} to
\enquote{Force-V2 system 2} might be ``the more natural `endogenous' change''.
This is an interesting suggestion that we would like to explore in future
research.}

%%%%%%%%%%%%%%%%%%%%%%%%%%%%%%%%%%%%%%%%%%%%%%%%%%%%%%%%%%%%%%%%%%%%%%%%%%%%%
\section{Future work}
\label{sec:fut}
%%%%%%%%%%%%%%%%%%%%%%%%%%%%%%%%%%%%%%%%%%%%%%%%%%%%%%%%%%%%%%%%%%%%%%%%%%%%%

\noindent Some issues discussed in the present paper provide interesting
pathways for future work. The generalisations and analyses presented here are
based on a small dataset. It would first of all be important to extend our
dataset in both qualitative and quantitative ways. The quality of our current
data is limited to interview settings with young people from Gouda and some
videos in which \ili{Dutch} teenagers with a Moroccan heritage present themselves and
discuss their lives. As mentioned by \citet{Freywaldetal:2015}, these methods
do not necessarily get the best results, because young people change to a more
formal (i.e.\ more Standard Dutch) register whenever an interviewer is present.
In our future attempts at data collection, we will therefore aim to leave the
recorder with the young people and let them speak without any interference.

From a synchronic point of view, there are some more observations in our
current dataset that warrant further discussion. One pattern that is repeatedly
found in these urban youth varieties, but not in Standard \ili{Dutch}, is
\emph{dat}-deletion, as shown in~\eqref{exdatd}:\largerpage

\ea
    \ea\label{exdatd} MD-C\\
    \gll Denk je hij weet Gouda uit zijn hoofd?\\
    think you he knows Gouda from his head\\
    \trans \enquote*{Do you think (that) he knows Gouda by heart?}
    \ex Standard \ili{Dutch}\\
    \gll Denk je dat hij Gouda uit zijn hoofd weet?\\
    think you that hij Gouda from his head knows\\
    \trans \enquote*{Do you think (that) he knows Gouda by heart?}
    \z
\z

\noindent Both the deletion of the complementiser and the lack of subordinate
word order (SOV in Standard Dutch) need to be addressed in any future
discussions on the C-domain of these urban youth varieties.

From a diachronic perspective, there are numerous strands for future research,
especially from a cross-linguistic perspective. To mention just one in \ili{Dutch}
alone: a more thorough study of the process of L2 acquisition\is{language acquisition} would be
beneficial to provide further evidence for the scenario sketched by
\citet{Walkden:2017}.

%%%%%%%%%%%%%%%%%%%%%%%%%%%%%%%%%%%%%%%%%%%%%%%%%%%%%%%%%%%%%%%%%%%%%%%%%%%%%
\section{Conclusion}
\label{sec:con}
%%%%%%%%%%%%%%%%%%%%%%%%%%%%%%%%%%%%%%%%%%%%%%%%%%%%%%%%%%%%%%%%%%%%%%%%%%%%%

%V3 does exist in urban varieties of \ili{Dutch}
%Tentative synchronic and diachronic analyses

\noindent In this paper we compared new data from \ili{Dutch} urban youth varieties
to emerging varieties in other \ili{Germanic} languages like \ili{German} and \ili{Norwegian}. We
first of all argued that, unlike previously thought, V3\is{V3 word order} word orders can indeed
be found in urban youth varieties of \ili{Dutch} as well. We supported this with
evidence from a small dataset consisting mainly of interviews with teenagers
with a Moroccan heritage living in Gouda, in the west of the Netherlands. Some
further examples from Dutch-Moroccan teenagers from other parts of the country
presenting themselves on YouTube and online forums suggest this phenomenon is
not limited to this community in Gouda. The V3\is{V3 word order} patterns in our dataset share
most characteristics of the optional V3\is{V3 word order} innovations observed in other \ili{Germanic}
urban youth varieties: the sentence-initial constituent is a frame-setter of
any category and the preverbal constituent is mainly the subject that functions
as a familiar topic.\is{topic}

There are, however, a couple of examples in our current dataset that do
\emph{not} function as familiar topics. We adopted \citegen{Walkden:2017}
analysis and extended it by adding an additional FrameP so that preverbal
constituents that do not function as familiar topics could be accounted for as
well. This type of analysis fits well into \citegen{Wolfe:2017} typology of V2\is{V2 word order}
languages.  Following Wolfe's cline of possible V2-languages, we argued that
the \ili{Dutch} urban youth varieties can best be analysed as \enquote{Force-V2
system 1} grammars with \is{verb movement}V-to-Force movement + an additional FrameP. They thus
differ from Standard \ili{Dutch}, which is argued to be a \enquote{Force-V2 system 2}
based on the fact that only hanging or left-dislocated\is{left dislocation}
topics\is{topic} can be found in sentence-initial position of superficial V3\is{V3 word order}
patterns.

\printchapterglossary{}

%%%%%%%%%%%%%%%%%%%%%%%%%%%%%%%%%%%%%%%%%%%%%%%%%%%%%%%%%%%%%%%%%%%%%%%%%%%%%
\section*{Acknowledgements}

The research leading to these results has received funding from the European
Union's seventh \emph{framework programme} for research, technological development and
demonstration under grant agreement no.\ 613465 and from the European Research
Council advanced grant for the project \emph{Rethinking comparative syntax} (ID
269752).

\section*{Appendix}\label{sec:app}
%%%%%%%%%%%%%%%%%%%%%%%%%%%%%%%%%%%%%%%%%%%%%%%%%%%%%%%%%%%%%%%%%%%%%%%%%%%%%

\Cref{tab:16:appendix} shows the dates and locations of interviews in
conducted with young speakers of Moroccan \ili{Dutch} in Gouda. More details
about the speakers and the corpus in general can be found in
\citet{Mourigh:fc}.\il{Berber}\il{Arabic}

\begin{table}
\caption{Background of speakers from \citet{Mourigh:fc}\label{tab:16:appendix}}
\fittable{\begin{tabular}{lcllrc}
\lsptoprule
Speaker	&	Interview Date	&	Location	&	Heritage language	&	Duration	&	Age	\\
\midrule
MD-A	&	20-11-2014	&	City centre	&	Berber	&	45 min	&	18	\\
MD-B1	&	02-10-2014	&	Sports club	&	Berber	&	23 min	&	17	\\
MD-B2	&	02-10-2014	&	Sports club	&	(same speaker)	&	8 min	&	17	\\
MD-B3	&	16-10-2014	&	Sports club	&	(same speaker)	&	36 min	&	17	\\
MD-C	&	02-10-2014	&	Sports club	&	Arabic	&	23 min	&	15	\\
MD-E1	&	26-10-2014	&	Park	&	Arabic	&	35 min	&	17	\\
MD-E2	&	15-06-2015	&	City centre	&	(same speaker)	&	60 min	&	17	\\
MD-H	&	26-10-2014	&	Park	&	Arabic	&	40 min	&	17	\\
MD-I	&	15-06-2015	&	City centre	&	Berber	&	60 min	&	21	\\
MD-K	&	30-10-2014	&	Community centre	&	Berber	&	90 min	&	14	\\
MD-L	&	30-10-2014	&	Community centre	&	Berber	&	90 min	&	15	\\
\lspbottomrule
\end{tabular}}
\end{table}

{\sloppy\printbibliography[heading=subbibliography,notkeyword=this]}

\end{document}
