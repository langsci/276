\documentclass[output=paper]{langsci/langscibook}
\ChapterDOI{10.5281/zenodo.4280625}

\author{András Bárány\affiliation{Bielefeld University}\and
        Theresa Biberauer\affiliation{University of Cambridge, Stellenbosch
        University, University of the West Cape}\and
        Jamie Douglas\affiliation{University of Cambridge}\lastand
        Sten Vikner\affiliation{Aarhus University}}

\title{Introduction}

\abstract{}

\begin{document}

\maketitle

\noindent The three volumes of \emph{Syntactic architecture and its
consequences} present contributions to comparative generative linguistics that
\enquote{rethink} existing approaches to an extensive range of phenomena,
domains, and architectural questions in linguistic theory. At the heart of the
contributions is the tension between descriptive and explanatory adequacy which
has long animated generative linguistics and which continues to grow thanks to
the increasing amount and diversity of data available to us. As the three
volumes show, such data from a large number of understudied languages as well
as diatopic and diachronic varieties of well-known languages are being used to
test previously stated hypotheses, develop novel ideas and expand on our
understanding of linguistic theory.

The volumes feature a combination of squib- and regular-length discussions
addressing research questions with foci which range from micro to macro in
scale. We hope that together, they provide a valuable overview of issues that
are currently being addressed in generative linguistics, broadly defined,
allowing readers to make novel analogies and connections across a range of
different research strands. The chapters in Volume 1, \emph{Syntax inside the
grammar}, and Volume 3, \emph{Inside syntax}, address research topics both at
the syntactic interfaces and in syntax proper, such as language change,
complexity, and variation, as well as alignment types, case, agreement, and the
syntax of null elements.

The contributions to the present, second volume, \emph{Perspectives from
morphosyntax}, address research questions and developments in morphosyntax.
The volume is divided into two parts, dealing with architectural (Part I) and
structural issues in morphosyntax (Part II).

The chapters in Part I, \emph{Architectural issues in morphosyntax}, take on
classic issues in grammar and provide new perspectives on questions such as
universality and variation (Watumull \& Chomsky), language evolution and
variation (Grohmann \& Leivada), as well as the architectural underpinnings of
recent syntactic theory. These involve the role of the structure-building
operation Merge (Zeijlstra; Moro) as well as the structure-removing operation
Remove (Müller), and cross-linguistic questions relating to labelling
(Tsoulas), the nature of linearisation (Johnson), phases and cyclicity
(Gallego), phrase structure (Lasnik \& Stone), and constraints on extraction
from conjuncts and adjuncts (Bošković). Myler’s chapter explores how formal
syntax can make predictions about surface frequencies in word order variation,
while the age-old question of lexical and syntactic categories is addressed
from different perspectives in the chapters by Brandner, Kenesei, and Moro.

Part II, \emph{Structural issues in morphosyntax}, starts with chapters
reconsidering properties of relative pronouns and relative clauses (Daskalaki;
Douglas). The following chapters deal with second-position and third-position
effects in constituent order (Mitrović; Meelen, Mourigh \& Cheng). Several
contributions deal with the structure of and microvariation in noun phrases,
for example, with respect to demonstratives (Cinque; Ledgeway; Kinn), and the
properties and syntactic representation of person splits in Romance (Manzini
and Savoia), as well as microvariation in passives in varieties of Dutch
(Haegeman).

Taken together, then, the contributions to this volume, many of which have
clearly been influenced and inspired by
\textcite{Roberts2010,Roberts2012},~\textcite{RobRou2003},~\textcite{RobHol2010},
\textcite{BibRob2012,BibRob2015}, and \textcite{BibHolRob2014} give the reader
a sense of current research into morphosyntax and morphosyntactic variation.

{\sloppy
    \printbibliography[heading=subbibliography,notkeyword=this]
}

\end{document}
