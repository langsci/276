\documentclass[output=paper]{langsci/langscibook}
\ChapterDOI{10.5281/zenodo.4280669}

\author{Guglielmo Cinque\affiliation{Ca' Foscari University, Venice}}
\title[Notes on the Merge position of demonstratives]
{Preliminary notes on the Merge position of deictic, anaphoric, distal
and proximal demonstratives}

% \chapterDOI{} %will be filled in at production

\abstract{In many languages the same demonstrative forms can be used either
deictically (to point to some entity present in the speech act situation) or
anaphorically (to refer back to some entity already mentioned in the previous
discourse). In other languages deictic\is{deixis} and anaphoric \isi{demonstratives} are
expressed by different forms, and in a subset of the latter group of languages
the deictic\is{deixis} and anaphoric \isi{demonstratives} can co-occur, in a certain order. The
two thus appear to be merged in different positions of the nominal extended
projection, with deictic\is{deixis} \isi{demonstratives} arguably merged higher than anaphoric
demonstratives, as is more clearly evident in certain languages. I submit that
this is true of all languages even if most do not provide any overt indication
of a different Merge position. Some languages also appear to provide evidence
that distal and proximal \isi{demonstratives} are merged in distinct positions of the
nominal extended projection.}


\begin{document}
\maketitle\glsresetall
\multicolsep=.25\baselineskip
%\textbf{Keywords:} \isi{demonstratives}, nominal extended projection, deixis, anaphora

\section{Introduction}\largerpage

Demonstratives, whether used deictically or anaphorically,\footnote{Anaphoric
    \isi{demonstratives}, together with \enquote{cataphoric} and
    \enquote{recognitional} \isi{demonstratives} (the latter used for entities known
    from shared knowledge, \citealt{Diessel1999}), are often termed
    \enquote{endophoric}, and are opposed to \enquote{exophoric} (deictic)
    \isi{demonstratives}, though anaphoric \isi{demonstratives} may also show
    distal/proximal/etc. deictic\is{deixis} distinctions. For simplicity I will keep here
    to the traditional terms \enquote{anaphoric} and \enquote{deictic}.} are
    usually taken to be merged in the same position of the extended nominal
    projection.  While most languages do not provide evidence to the contrary,
    there are some that do show a distinct \isi{Merge} position for their deictic\is{deixis} and
    anaphoric \isi{demonstratives} (pointing to a higher Merge position for the
    deictic\is{deixis} ones).  Rather than taking this to be a
    parametric\is{parameters} difference among
    languages, I submit that all languages merge their deictic\is{deixis} and anaphoric
    \isi{demonstratives} in two distinct positions.  This will simply not be visible
    in those languages where the two cannot co-occur and/or where nothing
    raises between the position occupied by anaphoric \isi{demonstratives} and that
    occupied by deictic\is{deixis} ones.

\section{Languages where deictic and anaphoric demonstratives are formally
distinct and can co-occur}\largerpage

I consider first those languages where the two types of \isi{demonstratives} are
represented by different forms\footnote{\textcite[§5.5]{Diessel1999} states
    that anaphoric \isi{demonstratives} are morphologically more complex than deictic\is{deixis}
    \isi{demonstratives}, citing a number of languages where the former are formed by
    adding a morpheme to the latter. \textcite[76f]{Dixon2003} however,
    documents the opposite case, where the deictic\is{deixis} demonstrative is formed by
    adding a morpheme to the anaphoric one. For the internal complexity of
demonstratives, composed of a determiner and an adjectival deictic\is{deixis} adjective,
see \textcites{Leu2007}[§2.5]{Leu2015} (pace \citealt{Kleiber1986}).} and
overtly display their distinct \isi{Merge} position by occurring together.

One such language is \ili{Ngiti}, a Central Sudanic Nilo-Saharan
language.  Demonstrative, numeral and adjectival nominal modifiers precede the
head noun\linebreak\parencite[§9]{KutschLojenga1994} and deictic\is{deixis} \isi{demonstratives} are
formally distinct from anaphoric ones
\parencite[cf.][§§9.5.1--9.5.2]{KutschLojenga1994}. See (\ref{ex:09.1}a,b).\footnote{The
    question arises whether the “anaphoric” demonstrative of \ili{Ngiti} and that of the
    other languages mentioned below are distinct from determiners. In
    \ili{Loniu} at
    least (see \cref{fn:9.5} below) the post-nominal anaphoric and deictic\is{deixis}
    \isi{demonstratives} are distinct from the determiners, which are pre-nominal. In
    the other languages, which lack determiners, this is harder to tell, though
    the relevant grammatical descriptions seem not to assimilate the anaphoric
    \isi{demonstratives} to determiners. I thank Richard Kayne for raising this
    general question. Possibly some of the anaphoric \isi{demonstratives} discussed
below correspond to the “neutral” \isi{demonstratives} of \textcite[§11]{Kayne2014}.}

\ea\label{ex:09.1}Ngiti \parencite[373, 375]{KutschLojenga1994}\\
\begin{multicols}{2}
\ea\label{ex:09.1a}
		\gll yà dza\\
			Dem\tss{deictic} house\\
		\glt ‘this house’
    \ex\label{ex:09.1b}
		\gll  ndɨ          dza\\
			Dem\tss{anaphoric} house\\
		\glt ‘that house (mentioned before)’
	\z
\end{multicols}
\z

As apparent from \eqref{ex:09.2}, the two types of \isi{demonstratives} can
co-occur, with the deictic \isi{demonstratives} preceding the anaphoric
ones:{\interfootnotelinepenalty=10000\footnote{If nominal modifiers can move only as part of a constituent
    containing the N \citep{Cinque2005}, the possibility that the
deictic\is{deixis} demonstrative of \eqref{ex:09.2} is merged below the
anaphoric one and is raised above it is not viable.}}

\ea\label{ex:09.2}Ngiti \parencite[376]{KutschLojenga1994}\\
    \gll yà             ndɨ         dza   \\
            Dem\tss{deictic} Dem\tss{anaphoric} house\\
    \glt ‘this house (mentioned before)’
\z

As pre-nominal modifiers (as opposed to post-nominal ones) reflect the order of
\isi{Merge}, with elements to the left higher than those to the right
(\citealt{Kayne1994}, \citealt{Cinque2009,Cinque2017}), this language provides
direct evidence that deictic\is{deixis} \isi{demonstratives} are merged higher than anaphoric
demonstratives.

Another language showing the distinct \isi{Merge} position of deictic\is{deixis} and anaphoric
demonstratives, with the former arguably higher than the latter, is the Papuan
(Yam) language \ili{Komnzo}.

In addition to deictic\is{deixis} \isi{demonstratives}, Komnzo has one demonstrative, \emph{ane},
which \blockquote[{\citealt[128f]{Dohler2016}}]{has no spatial reference, but
    it is used for anaphoric reference. It marks a referent which has been
    established in the preceding context. [\dots] It may combine with the
proximal and the medial demonstrative identifier as can be seen in example
[(3)]} in the order N $>$ anaphoric demonstrative $>$ deictic\is{deixis} demonstrative:

\ea\label{ex:09.3} \ili{Komnzo} \parencite[129]{Dohler2016}\\
    \gll fintäth \textit{ane} \textit{z=iyé} \dots{}  yem=anme  dagon.\\
        \Propn{}  Dem\tss{anaphoric} \Prox=\Tsg.\M:\Npst{}.be {} cassowary=\Poss.\Nsg{} food\\
	\glt ‘This fintath (Semecarpus sp.) here is the cassowaries’ food.’
\z

The relative order of the two is with the anaphoric demonstrative closer to the
noun than the deictic\is{deixis} demonstrative, as was the case in \ili{Ngiti}.
The linear order, however, is the reverse, arguably due to the successive
\isi{raising} of the NP, with pied piping\is{pied-piping} of the
\emph{whose picture}-type, first above the lower anaphoric and then above the
higher deictic\is{deixis} demonstrative dragging along the lower anaphoric one,
with the result of reversing the order entirely
\parencite[cf.][]{Cinque2005,Cinque2017}.\largerpage[-1]

Identical to the \ili{Komnzo} situation is that of the Alor Pantar (Papuan)
language \ili{Kaere}, where the anaphoric demonstrative \emph{erang}
can combine with the deictic\is{deixis} \isi{demonstratives} \emph{ga} ‘this’ or \emph{gu}
‘that’ \parencite[§4]{Klamer2014} (see \ref{ex:09.4}), and that of the Oceanic language
\ili{Loniu} \parencite[§4.3.7]{Hamel1994}, where the anaphoric
demonstrative \emph{nropo} can co-occur with the deictic\is{deixis} demonstrative
\emph{itiyen} `that (relatively distant from speaker)' (see \ref{ex:09.5}), in both cases
with the order N Dem\tss{anaphoric} Dem\tss{deictic}:\footnote{\textquote{[T]he
    two together are equivalent to \ili{English} \enquote*{aforementioned}}
    \parencite[99]{Hamel1994}.  In addition to the anaphoric and deictic\is{deixis}
    \isi{demonstratives} in post-nominal position, \ili{Loniu} appears to also have
    determiners, in pre-nominal position.  \enquote{The order of constituents
        in the noun phrase is, generally, as shown in the formula in [(i)]
    below} \parencite[89]{Hamel1994}.

\begin{exe}
    \exi{(i)}
    (Det) Noun (Possessor NP) (Associated NP) (Descriptive Adjunct)
    (Quantifier) (Prepositional Phrase) (Relative Clause) (Demonstrative)
\end{exe}

\enquote{The personal pronouns which function as determiner are the same as
    those used as nominals for subject, object, and so on. Although they may
    co-occur with inanimate nouns, the majority of NPs in the data which
    contain personal pronoun determiners are animate.\is{animacy} [\dots] These
personal pronoun determiners, however, seem to be present only in NPs which are
definite} \parencite[90]{Hamel1994}. See the example in (ii):

\begin{exe}
    \exi{(ii)} Loniu\\
	\gll iy   pihin      iy    huti kawa\\
		\Tsg{} woman \Tsg{} take basket\\
	\glt ‘The woman takes the basket’
\end{exe}\label{fn:9.5}}

\ea \ili{Kaere} \parencite[120]{Klamer2014}\label{ex:09.4}\\
	\gll kunang   masik     utug \textit{erang} \textit{gu} \\
        child  male three {\Dem\tss{anaphoric}} {that}\\
	\glt ‘those three boys (mentioned earlier)’
\ex \ili{Loniu} \parencite[99]{Hamel1994}\label{ex:09.5}\\
    \gll \dots{} hetow    law        a  iy  \textit{nropo} \textit{itiyen} ... \\
		{} \Tpl.\Cl{} \Rel{} \Poss{} \Tsg{} {\Dem{}\tss{anaphoric}} {\Dem{}\tss{deictic}} {}\\
	\glt ‘\dots  to those aforementioned relatives of his \dots’
\z\largerpage

The Austronesian, Malayo-Polynesian, languages \ili{Gayo}
(\citealt{Eades2005}) and \ili{Nias} (\citealt{Brown2005}) and the
Niger-Congo languages \ili{Samba Leko} \citep{Fabre2004} and
\ili{Kitalinga} (\citealt{Paluku1998}) instead show post-nominally
the same order shown pre-nom\-i\-nal\-ly by \ili{Ngiti}: NP $>$ deictic\is{deixis} demonstrative
$>$ anaphoric demonstrative:\footnote{In (\ref{ex:09.6b}), the anaphoric
    demonstrative \emph{nomema} contains \emph{mema} ‘earlier’.  Adjectives and
    numerals follow the two \isi{demonstratives} in that order
    \parencite[412]{Brown2001}.  Another language with an anaphoric
    demonstrative meaning ‘earlier/before’ is Madurese:

        \begin{exe}\exi{(i)} \ili{Madurese} \parencite[192]{Davies2010}f)\\ %(i)
                \gll Reng   lake’ gella’ entar ka Sorbaja \\
                     person male before   go  to Surabaja \\
                \glt ‘That man (we were talking about just now) went to
            Surabaja’
        \end{exe}}

\ea\label{ex:09.6}
\ea\label{ex:09.6a}\ili{Gayo} \parencite[225]{Eades2005})\\
		\gll Serule-\textit{ni-ne}\\
			    Serule-this-earlier\\
		\glt ‘this Serule’ [Serule-this-\textsc{mentioned} earlier] (the aforementioned Serule)\\
        \ex\label{ex:09.6b}\ili{Nias} \parencite[579]{Brown2005})\\
		\gll Ba  si'ulu  wa  e  nama-da  \textit{andre} \textit{nomema'e}!?\\
            \Cnj{} noble \Dptcl{} \Dptcl{} father:\Mut{}-\Fpl.\Incl.\Poss{} {\Dem{}\tss{Deictic}} {\Dem{}\tss{anaphoric}}\\
		\glt ‘And you mean that ancestor you’ve been talking about was a noble!?’
        \ex\label{ex:09.6c}\ili{Samba Leko} \parencite[173]{Fabre2004}\\
		\gll b\=a?–\=a \textit{yê} \textit{d\=o}\\
			    iron {\Dem{}\tss{deictic}} {\Dem{}\tss{anaphoric}}\\
		\glt ‘that iron we talked about’ [our translation]
    \ex\label{ex:09.6d}Kitalinga\il{Kitalinga} \parencite[203]{Paluku1998}\\
		\gll omumelo  ɤú-\textit{nì-lá}\\
            throat  \textsc{?}-{\Dem{}\tss{deictic}}-{\Dem{}\tss{anaphoric}}\\
		\glt ‘this aforementioned throat’, orig.\ French \enquote*{gorge celui-ci – en question}
	\z
\z

My interpretation of the orders in \eqref{ex:09.6} is that they are derived
by \isi{raising} the NP (or constituents containing the NP) above the two
\isi{demonstratives} in one fell swoop (without pied piping\is{pied-piping})
(cf.\ \citealt{Cinque2005,Cinque2017}).\footnote{For evidence that constituents
appearing to the right of N/V/etc. cannot be taken to be merged there, but come
to be there as a function of the N(P)/V(P)/etc.\ moving above them, see
\citet{Cinque2009}.}

\section{Languages where deictic and anaphoric demonstratives are formally
distinct, occupy different positions, but cannot co-occur}

In the Trans-New Guinea Alor-Pantar language \ili{Abui}
\parencites[§3.5.2]{Kratochvil2007}{Kratochvil2011} \blockquote{[t]he deictic\is{deixis}
demonstratives precede the head noun while the anaphoric \isi{demonstratives} follow
it} \parencite[156]{Kratochvil2007}. See the overall structure of Abui
determiner phrases in \eqref{ex:09.7} \parencite[156]{Kratochvil2007}, and the illustrative
examples of the order of the two types of \isi{demonstratives} in \eqref{ex:09.8}:

\ea\label{ex:09.7}
(\textsc{dem}\tss{deictic}) \textsc{(nposs proposs-) n (nmod) (adj/v) (quant)} (\textsc{dem}\tss{anaphoric})
\ex\label{ex:09.8}
	\ea Abui \parencite[111]{Kratochvil2007}\\
		\gll \textit{oro} fala\\
			    {\Dem{}\tss{deictic}} house\\
		\glt ‘that house over there (far from us)’
	\ex Abui \parencite[114]{Kratochvil2007}\\
		\gll fala \textit{to}\\
			    house {\Dem{}\tss{anaphoric}}\\
		\glt ‘the house (you just talked about)’
	\z
\z

If deictic\is{deixis} \isi{demonstratives} are merged higher than anaphoric
\isi{demonstratives}, the Abui DP internal order Dem\tss{deictic} N A Num
Dem\tss{anaphoric} can be analysed as involving successive raisings\is{raising}
of the NP, with pied piping\is{pied-piping} of the \emph{whose picture}-type
above the lower anaphoric demonstrative but not above the higher
deictic\is{deixis} demonstrative.\footnote{The situation in \ili{Topoke}
    (Bantu, C53) is only slightly different, as \enquote{the anaphoric
    demonstrative always follows the noun, whereas other \isi{demonstratives}
can either precede or follow} \parencite[§2.4]{vandeVelde2005b}. This suggests
that anaphoric \isi{demonstratives} are obligatorily crossed over by the NP,
while deictic\is{deixis} \isi{demonstratives} are crossed over by the NP only
optionally. Only slightly different is the case of \ili{Rama} (Chibchan;
\citealt[§6.6]{CraigGrinevald1988}), where the deictic\is{deixis} demonstrative
is only pre-nominal while the anaphoric one \enquote{meaning ‘previously
mentioned’ [\dots{}] is found either pre- or post-nominally} (p.\ 15).}

In the Dogon language \ili{Jamsay}, where the deictic\is{deixis} demonstrative follows
the noun (cf.~\ref{ex:09.9}a)\footnote{\enquote{\emph{núŋò} is deictic\is{deixis}, and may be
accompanied by pointing or a similar gesture} \parencite[162]{Heath2008}.} and
the anaphoric one precedes it (cf.~\ref{ex:09.9}b),\footnote{\enquote{Unlike deictic\is{deixis}
[noun $+$ \emph{núŋò}], the phrase [\emph{kò} $+$ noun] is discourse anaphoric
\dots{}} \parencite[164]{Heath2008}.} within the overall order
〈Dem\tss{anaphoric}〉 N A Num 〈Dem\tss{deictic}〉, the derivation must be
different, involving \isi{raising} of the constituent [Dem\tss{anaphoric} N A Num]
(itself obtained via \isi{raising} of the NP around A and Num) above the higher
deictic demonstrative (cf.\  \citealt{Cinque2005,Cinque2017}).

\ea\label{ex:09.9}
	\ea Jamsay \parencite[161]{Heath2008}\\
		\gll èjù  \textit{núŋò}\\
                field.\Ll{} {\Dem{}\tss{deictic}}\\
		\glt ‘this/that field’
	\ex Jamsay \parencite[164]{Heath2008}\\
		\gll \textit{kò}  kùmàndâw kù\textsuperscript{n} bé\\
            {\Dem{}\tss{anaphoric}} Major \textsc{def} \textsc{pl}\\
		\glt ‘those (aforementioned) Majors’
	\z
\z

\section{Languages where deictic and anaphoric demonstratives are formally
identical, occupy different positions, but cannot co-occur}

The same pattern is instantiated by a number of other languages, modulo the
formal identity of the deictic\is{deixis} and the anaphoric \isi{demonstratives}.

\textcite[142]{Migdalski2001} notes that \blockquote{demonstratives may either
    precede or follow a noun in \ili{Polish}. The latter option is
    stylistically marked and is used only when the noun followed by a
    demonstrative has been previously mentioned, [\dots{}] as in
    [(10)]}:\footnote{The \ili{Polish} situation recalls the semantic difference
    between pre- and post-nominal \isi{demonstratives} in \ili{Spanish} and Modern
    \ili{Greek}
    (modulo the obligatory presence of a determiner in pre-nominal position
    when the demonstrative is post-nominal). As observed by
    \citet{Bernstein1997} and \citet{Taboada2007} for \ili{Spanish} and
    \citet{Panagiotidis2000} for Modern \ili{Greek}, a post-nominal
    demonstrative is only interpreted anaphorically (unless a demonstrative
    reinforcer is added), while a pre-nominal one can be interpreted
    deictically. But see \textcite[50, n.\ 27]{Bruge2002} and \textcite[§2.5.3,
p.\ 167, n.\ 51]{Bruge2000} for discussion of a number of complexities and of
differences among the \ili{Spanish} distal and proximal \isi{demonstratives}.}

\ea Polish\label{ex:09.10}
	\ea
		\textit{Ta} ksiazka\\
		‘\textit{this} book’\\
	\ex
		\gll Ksiazka \textit{ta} (acceptable if the book has been mentioned previously)\\
			book this\\
	\z
\z

Here too it is possible to analyse the pattern in Dem\tss{deictic} NP
Dem\tss{anaphoric} as involving raising of the NP (with possible pied piping)
above the lower anaphoric demonstrative but not above the higher deictic\is{deixis}
one.\footnote{In \ili{Italian}, where no evidence exists of a different \isi{Merge}
    position of deictic\is{deixis} and anaphoric \isi{demonstratives}, there is still a
    difference between the two in the possibility for the former but not for
    the latter, in its neuter usage (presumably with a silent head noun \textsc{thing};
    cf.\ \citealt{KayPol2009}), to take a locative \enquote{reinforcer}. See
    (i):

\begin{exe}
\exi{(i)}
\begin{xlist}
	\ex
		\gll Questo (*qui) non lo so\\
			 This (here) not it I.know\\
		\glt ‘This I don’t know’
	\ex
		\gll Quello (*lì) me lo sono chiesto anch’io\\
			That (there) to.me it am asked even-I\\
		\glt ‘That I wondered myself’
\end{xlist}
\end{exe}}

The opposite pattern Dem\tss{anaphoric} NP Dem\tss{deictic} is instantiated by
Thimbukushu (Bantu language of Namibia; \citealt{Fisch1998}), where
\enquote{[u]sually \isi{demonstratives} [\dots{}] occur as postpositive determiners after
the nouns to which they refer} \parencite[50]{Fisch1998}, see \eqref{ex:09.11}:

\ea Thimbukushu\label{ex:09.11} \parencite[50]{Fisch1998}\\
	\gll {}[ Mugenda \textit{oyu}] na haka\\
            {} guest this I like\\
	\glt ‘I like this guest’
\z

\enquote{If the demonstrative preposes the noun, it carries the meaning of
    \enquote*{this aforementioned}, \enquote*{this one mentioned}}
    \parencite[50]{Fisch1998}, see \eqref{ex:09.12}:\footnote{Romanian
        appears to be similar. Post-nominal \isi{demonstratives} have a
    deictic\is{deixis} interpretation while pre-nominal ones, which belong to a
non-colloquial style \parencite[cf.][n.\ 32]{Bruge2002}, have an anaphoric
interpretation \parencites[31]{Giusti2005}[299f]{Nicolae2013}.}

\ea Thimbukushu \parencite[50]{Fisch1998}\\\label{ex:09.12}
	\gll [ \textit{oyu} ngombe]\\
		    {} the.aforementioned cow\\
	\glt ‘this cow’
\z

This pattern can be taken to involve no \isi{movement} of the NP above the lower
anaphoric demonstrative (or possibly \isi{movement} of the NP in the
\emph{picture of whom}-mode, which has the effect of not changing the relative
order of the two elements), and raising of the NP (or of larger constituents
containing the NP) above the higher deictic\is{deixis} demonstrative.

\section{Languages where distal and proximal demonstratives occupy different
positions}

In \ili{Nawdm} (Niger-Congo, Gur; \citealt[§2.4]{Albro1998})

\begin{quote}there are
    two basic \isi{demonstratives} [\dots{}], corresponding to \enquote*{this} and
    \enquote*{that} in \ili{English}. Their distribution within the DP is different.
    The demonstrative corresponding to \enquote*{this} appears at the end of
    the DP [\dots{}], and the demonstrative corresponding to \enquote*{that}
appears at the beginning of the DP.
\end{quote}

\noindent See \eqref{ex:09.13}:\largerpage[1]

\ea Nawdm \parencite[6]{Albro1998}\label{ex:09.13}
	\ea
		\gll \textit{làɁà} bà  hɔˊlˋə  té  tèréɁété:\\
			that dog black \Cl.\Pl{} \Cl{}-two-\Cl{}\\
		\glt ‘those two black (big) dogs’
	\ex
		\gll bà hɔˊlˋə té tèréɁètèn \textit{tènté}\\
			dog black \Cl.\Pl{} \Cl{}-two-\Cl{} {\Cl{}-this-\Cl{}}\\
		\glt ‘these two black (big) dogs’
	\z
\z

According to \textcite[66ff]{Apronti1971}, the same distribution (Dem\tss{that}
N A Num and N A Num Dem\tss{this}) is found in the Kwa language
\ili{Dangme}.

It is thus tempting to assume that the distal and proximal deictic\is{deixis}
demonstratives occupy two distinct \isi{Merge} positions, with distal \isi{demonstratives}
higher than proximal \isi{demonstratives}, as shown in \eqref{ex:09.14}:

\ea\label{ex:09.14}
\begin{tikzpicture}[baseline, level distance=1cm, level/.style={sibling
    distance=2.5cm}]

    \node {}
        child { node {\Dem{}P\tss{distaldeictic}} }
        child { node { \dots{} }
            child { node { } }
            child { node {\dots}
                child { node {\Dem{}P\tss{proximaldeictic}} }
                child { node {\dots}
                    child { node { } }
                    child { node {\dots}
                        child { node {\Dem{}P\tss{anaphoric}} }
                        child { node {\dots}
                            child { node {} }
                            child { node {} }
                        }
                    }
                }
            }
        }
    ;

\end{tikzpicture}
\z

The order in \ili{Nawdm} and \ili{Dangme} would then involve raising of the NP
with pied piping of the \emph{whose picture}-type around A, Num and the lower
proximal demonstrative, but not above the higher distal one, which then appears
pre-nominally.

As in the case of \ili{Jamsay} above, a different derivation must be involved
to yield the order Dem\tss{proximal} (Num) N (A) Dem\tss{distal} of \ili{Tigre}
(Afro-Asiatic, Semitic), where it is the proximal demonstrative that precedes
the noun and the distal one that follows it (see \ref{ex:09.15}):

\ea \ili{Tigre} (\citealt{Dryer2013}, after \citealt[45]{Raz1983})\label{ex:09.15}
    \begin{multicols}{2}
	\ea
		\gll \textit{ʔəllan}  ʔamʕəl\=at\\
			{this.\glossF.\Pl{}} days\\
		\glt ‘these days’
	\ex
		\gll ʔəb laʔawk\=ad  \textit{lahay}\\
            at time {that.\M{}}\\
		\glt ‘at that time’
	\z
	\end{multicols}
\z

The NP must raise around A with pied piping\is{pied-piping} of the \emph{whose
picture}-type (or with no pied piping), and then around Num and the lower
proximal demonstrative with pied piping\is{pied-piping} of the \emph{picture of
whom}-type, after which it raises around the higher distal demonstrative again
with pied piping\is{pied-piping} of the \emph{whose picture}-type (a mixture of
movements typically involved in the derivation of non-consistent languages; see
\citealt{Cinque2017}).

The fact that the two positions are presumably close to each other may give the
impression in those languages where no material raises between them that they
are one and the same position.

\printchapterglossary{}

\section*{Acknowledgements}

To Ian, to whom \emph{nihil alienum est} in things linguistic, with fond
memories and admiration. For helpful comments to a previous draft of this squib
I am indebted to Laura Brugè, Richard Kayne and two anonymous reviewers.

{\sloppy
\printbibliography[heading=subbibliography,notkeyword=this]
}

\end{document}
