\documentclass[output=paper]{langsci/langscibook}

\author{Evangelia Daskalaki\affiliation{University of Alberta}}
\title{Types of relative pronouns}

% \chapterDOI{} %will be filled in at production

\abstract{In this paper, I explore the possibility that relative pronouns, like
personal pronouns, show different degrees of strength/deficiency. I show that,
at least in \ili{Greek}, the \gls{RR} pronoun \emph{o opios} is semantically
deficient compared to its \gls{FR} counterpart \emph{opjos} in two interrelated
respects: (i) it is referentially deficient and (ii) it does not license its
own range. After showing that both \gls{FR} and \gls{RR} pronouns\is{relative pronouns!restrictive relative pronouns} behave like
transitive Ds, I propose that their differences lie in their featural
composition, rather than in their structural make-up: \gls{FR} Determiners,
unlike \gls{RR} Determiners, are semantically definite.}
\maketitle

\begin{document}\glsresetall

%\textbf{Keywords:} relative pronouns, restrictive relatives, free relatives,
%\ili{Greek}

\section{Introduction}\label{sec:key:01.1}

That pronouns may show a different cluster of properties -- diachronically,
synchronically, and cross-linguistically -- is a well-established fact in the
literature. Existing accounts, focusing primarily on the different classes of
personal pronouns, suggest two main lines of approach.\footnote{For a detailed
overview and application to personal pronouns in Greek, see
\citet{Mavrogiorgos2010}.} The first one attributes the different properties of
(personal) pronouns to their external category
\parencite{CarSta1999,DecWil2002}. The second type of analyses treats all
pronouns as Determiners projecting a DP and derives their differences from
their internal structure and/or featural composition
(\citealt{Abney1987,Cardinaletti1994,Uriagereka1995}, among others).

The aim of this paper is to explore whether similar claims can be made for the
class of relative pronouns.\footnote{See also \citet{Sportiche2011} for
    \ili{French} restrictive relative pronouns,\is{relative
    pronouns!restrictive relative pronouns} and \citet{Wiltschko1998} for
\ili{German} restrictive relative pronouns.} I argue that, at least in
\ili{Greek}, \gls{RR} pronouns\is{relative pronouns!restrictive relative pronouns} can be shown to be semantically deficient
compared to \gls{FR} pronouns\is{relative pronouns!free relative pronouns} in two (interrelated) respects: (i) \gls{RR}
pronouns are not inherently definite/referential, and (ii) \gls{RR} pronouns\is{relative pronouns!restrictive relative pronouns} do
not license their own range. After showing that both \gls{FR} and \gls{RR}
pronouns behave like transitive Ds, and are therefore categorially equivalent,
I propose that their differences derive from their featural composition:
\gls{FR} Determiners, unlike \gls{RR} Determiners, are semantically
definite/referential. Because they are definite/referential Determiners, they
need a range that may take the form of a lexical NP complement or of an \isi{animacy}
restrictor.

The paper is structured as follows: \Cref{sec:key:01.2} provides some
background information concerning (\ili{Greek}) Relative clauses and pronouns.
\Cref{sec:key:01.3} establishes at an empirical level the semantic deficiency
of \gls{RR} pronouns and \Cref{sec:key:01.4} develops an analysis that
capitalizes on the featural composition of the \gls{FR} and \gls{RR} D head.
Finally, \Cref{sec:key:01.5} concludes the discussion.

\section{Background information on relative clauses and
pronouns}\label{sec:key:01.2}

\subsection{(\ili{Greek}) relative clauses}\label{sub:key:01.2.1}

Restrictive and Free Relatives are A$'$ \isi{movement} dependencies with
different functions. Whereas Restrictive Relatives function as modifiers of
nominal heads, Free Relatives function as arguments/adjuncts of lexical
predicates \parencite{AleLawMeiWil2000,Bianchi2002,GroLan1998}.  This is
illustrated below with \ili{Greek}:\footnote{On \ili{Greek} \glspl{RR} see
\citet{Alexopoulou2006}; on \ili{Greek} \glspl{FR} see \textcite{AleVar1997}.}

\ea Greek\label{ex:key:01.1}\\
	\gll ðjaleksa tus maθites\tss{i} [ {tus opius\tss{i}} protines t\tss{i} ].\\
        chose.\Fsg{} the students.\M.\Pl.\Acc{} {} which.\M.\Pl.\Acc{} recommended.\Ssg{} {}\\
	\glt ‘I chose the students who you recommended.'
\z

\ea Greek\label{ex:key:01.2}\\
	\gll ðjaleksa [ opjus\tss{i} protines t\tss{i} ].\\
        chose.\Fsg{} {} who.\M.\Pl.\Acc{} recommended.\Ssg{} {}\\
	\glt \enquote*{I chose who you recommended.}
\z

In~\eqref{ex:key:01.1}, the \gls{RR} modifies the nominal head \emph{maθites}
‘students’. In~\eqref{ex:key:01.2}, the \gls{FR} complements the verbal head \emph{ðjaleksa}
‘chose’.

As far as their semantic interpretation is concerned, \glspl{FR} in DP position
are semantically equivalent with strong DPs \citep{Jacobson1995}. For instance,
the \gls{FR} in~\eqref{ex:key:01.2} can be paraphrased with an \gls{RR} headed
by a demonstrative~\eqref{ex:key:01.3}:

\ea \ili{Greek}\label{ex:key:01.3}\\
	\gll ðjaleksa [ aftus [ {tus opius\tss{i}} protines t\tss{i} ]].\\
    chose.\Fsg{} {} those.\M.\Pl.\Acc{} {} which.\M.\Pl.\Acc{} recommended.\Ssg{} {}\\
	\glt \enquote*{I chose those ones you recommended.}
\z

\subsection{(\ili{Greek}) relative pronouns}\label{sub:key:01.2.2}

With respect to restrictive and free relative pronouns, languages differ as to
whether they draw them from the same paradigm. Thus, \ili{English} draws both
\gls{RR} and \gls{FR} pronouns\is{relative pronouns!free relative pronouns} from the paradigm of interrogative pronouns.
\ili{German}, on the other hand, uses interrogative pronouns to introduce \glspl{FR} and
morphologically definite determiners to introduce \glspl{RR} \citep{Wiltschko1998}.

\ili{Greek} stands somewhere in between: \gls{RR} and \gls{FR} pronouns\is{relative pronouns!free relative pronouns} are similar
in that they both combine interrogative and definite morphology.\footnote{
    Thus, the \gls{RR} pronoun \emph{o opios} consists of the morphologically
    definite determiner \emph{o} and the word \emph{opios}. The latter,
    being itself complex, can be decomposed into the determiner-like prefix
    \emph{o-} and the interrogative \emph{pios} ‘who’ (on the morphological
    decomposition of the \gls{RR} \emph{o opios}, see
    \citealt{Alexiadou1998}). A similar pattern is shown by the \gls{FR} pronoun
    \emph{opjos}. Like its \gls{RR} counterpart, it is a complex word,
    consisting of the determiner-like prefix \emph{o-} and the interrogative
    \emph{pjos} ‘who’. Unlike its \gls{RR} counterpart though, it is not
    introduced by a free determiner (on the etymological decomposition of the
    \gls{FR} \emph{opjos}, see \citealt{Chila-Markopoulou1994}).} However,
    they are not identical and replacing one with the other leads to strong
    ungrammaticality:

\ea \ili{Greek}\label{ex:key:01.4}\\
    \gll \llap{*}ðjaleksa tus maθites\tss{i} [ opjus\tss{i} protines t\tss{i} ].\\
    chose.\Fsg{} the students.\Acc{} {} who.\Acc{} recommended.\Ssg{} {}\\
    \trans *\enquote*{I chose the students whoever you recommended.}
\z

\ea \ili{Greek}\label{ex:key:01.5}\\
    \gll \llap{*}ðjaleksa [ {tus opius\tss{i}} protines t\tss{i} ].\\
    chose.\Fsg{} {} which.\Acc{} recommended.\Ssg{} {}\\
    \trans *\enquote*{I chose which you recommended.}
\z

Furthermore, both types of pronouns are inflected for the same range of
categories. Thus, they inflect for number (singular, plural),\is{number
features} gender
(masculine, feminine, neuter),\is{φ-features} and case (nominative, accusative\is{accusative case},
genitive\is{genitive case}), displaying in this respect the main features
characterizing \ili{Greek} nominal inflection. The complete morphological
paradigm of \emph{opjos} and \emph{o opios} is provided in
\Cref{tab:key:1,tab:key:2}, respectively \parencite[100]{HolMacPhiWar2004}.

\begin{table}
\caption{The morphological paradigm of the \gls{FR} pronoun
\emph{opjos-a-o}}\label{tab:key:1}
\begin{tabularx}{\textwidth}{XXXXXXX}
\lsptoprule
           & \multicolumn{3}{c}{Singular}                & \multicolumn{3}{c}{Plural}  \\
           & Masc         & Fem            & Neut        & Masc         & Fem          & Neut\\
\midrule
\Nom{}     & \emph{opjos} & \emph{opja}    & \emph{opjo} & \emph{opji}  & \emph{opjes} & \emph{opja}\\
\Acc{}     & \emph{opjon} & \emph{opja(n)} & \emph{opju} & \emph{opjus} &
\emph{opjes} & \emph{opja}\\
\Gen{}     & \emph{opju}  & \emph{opjas}   & \emph{opjo} & \emph{opjon} &
\emph{opjon} & \emph{opjon}\\
\lspbottomrule
\end{tabularx}
\end{table}

\begin{table}[htpb]
\caption{The morphological paradigm of the \gls{RR} pronoun \emph{o opios-i
opia-to opio}}\label{tab:key:2}
\begin{tabularx}{\textwidth}{XXXXXXX}
\lsptoprule
            & \multicolumn{3}{c}{Singular}                & \multicolumn{3}{c}{Plural}  \\
            & Masc            & Fem              & Neut           & Masc             & Fem              & Neut\\
\midrule
\Nom{}      & \emph{o opios}  & \emph{i opia}    & \emph{to opio} & \emph{i opii}    & \emph{i opies}   & \emph{ta opia}\\
\Acc{}      & \emph{ton opio} & \emph{tin opia}  & \emph{tu opiu} & \emph{tus opius} & \emph{tis opies} & \emph{ta opia}\\
\Gen{}      & \emph{tu opju}  & \emph{tis opias} & \emph{to opio} & \emph{ton opion} & \emph{ton opion} & \emph{ton opion}\\
\lspbottomrule
\end{tabularx}
\end{table}

\section{On the deficiency of RR pronouns}\label{sec:key:01.3}

Despite being amenable to a similar etymological decomposition and despite
being marked for the same range of morphological features, \gls{RR} pronouns
can be shown to be deficient compared to their \gls{FR} counterparts in a
number of ways that recall the differences identified between strong and weak
personal pronouns. Let us consider them in turn:

\paragraph*{(i) Contrastive Focus:} To begin with, only \gls{FR}
pronouns\is{relative pronouns!free relative pronouns} may bear contrastive
focus\is{focus!contrastive focus}. This is shown by the contrast in
grammaticality between \eqref{ex:key:01.6} and~\eqref{ex:key:01.7}.

\ea \ili{Greek}\label{ex:key:01.6}\\
	\gll kalese mono OPJUS tu protines oxi opjes tu protines\\
		invited.\Tsg{} only who.\M{}.\Pl{} \Cl.\Tsg.\M{} recommended.\Ssg{} not who.\glossF.\Pl{} \Cl.\Tsg.\M{} recommended.\Ssg{}\\
	\glt \enquote*{He only invited whichever MEN you recommended to him, not whichever women you recommended to him.}
\z

\ea \ili{Greek}\label{ex:key:01.7}\\
    \gll \llap{*}kalese mono aftus {TUS OPIUS} tu protines oxi aftes {tis opies} tu protines.\\
    invited.\Tsg{} only those.\M.\Pl.\Acc{} which.\M.\Pl{} \Cl.\Tsg{}
    recommended.\Ssg{} not those.\glossF.\Pl.\Acc{} which.\glossF.\Pl{} \Cl.\Tsg{} recommended.\Ssg{}\\
    \glt *\enquote*{He only invited those men WHO you recommended, not
    those women who you recommended.}
\z

Thus, in~\eqref{ex:key:01.6}, the \gls{FR} pronoun \emph{opjus}, encoding masculine gender,
can be contrastively focused with the \gls{FR} pronoun \emph{opjes}, encoding
feminine gender. Crucially, in the same contrastive configuration, the
\gls{RR} pronoun \emph{tus opius} is not permissible with contrastive stress
\eqref{ex:key:01.7}:\footnote{The \ili{English} translation in \eqref{ex:key:01.6} and
    \eqref{ex:key:01.7} in the main text fails to convey the
    contrast between \gls{FR} and \gls{RR} pronouns\is{relative pronouns!restrictive relative pronouns} with respect to focus. This
    is because \ili{English} relative pronouns do not encode gender
    distinctions (that is \emph{who} can be used to refer to both female and
    male entities). The same effect, though, can be conveyed with the
    \ili{English} \gls{FR} pronouns\is{relative pronouns!free relative pronouns} \emph{who} (a \gls{FR} pronoun used for
    animate entities) and \emph{what} (a \gls{FR} pronoun used for inanimate
    entities).

\begin{exe}
    \exi{(i)} Greek\\
    \gll \llap{*}ðen θeli mono afta {ta opia} exis ala ke aftus {TUS OPIUS} exis.\\
    \Neg{} want.\Tsg{} only those.\glossN.\Pl.\Acc{} which.\glossN.\Pl{} have.\Ssg{} but and those.\M.\Pl.\Acc{} who.\M{}.\Pl{} have.\Ssg{}\\
    \glt intended: \enquote*{He doesn’t only want those (things) which you
            have, but also those (persons) WHO you have.}
\end{exe}

\begin{exe}
    \exi{(ii)} Greek\\
	\gll ðen θeli mono oti exis ala ke OPJON exis.\\
		\Neg{} want.\Tsg{} only what have.\Ssg{} but and who.\M.\Sg{} have.\Ssg{}\\
	\glt \enquote*{He doesn’t only want what you have but also WHO you have.}
\end{exe}}

\paragraph*{(ii) Null Counterparts:} Secondly, only \gls{FR} pronouns\is{relative pronouns!free relative pronouns} are
obligatorily realized \parencite[22]{AleLawMeiWil2000}. To this end,
example~\eqref{ex:key:01.8} shows that replacing a \gls{FR} pronoun with the
uninflected complementizer\is{complementizers} \emph{pu} ‘that’ leads to strong ungrammaticality:

\ea \ili{Greek}\label{ex:key:01.8}
    \ea[]{
		\gll ðjaleksa [ opjus protines ].\\
        chose.\Fsg{} {} who.\Acc{} recommended.\Ssg{}\\
    \glt \enquote*{I chose whoever you recommended.}}
    \ex[*]{
		\gll ðjaleksa [ pu protines].\\
        chose.\Fsg{} {} that recommended.\Ssg{}\\
        \glt *\enquote*{I chose that you recommended.}}
	\z
\z

By contrast, complementizer\is{complementizers} \glspl{RR} (\ref{ex:key:01.9}b)  are a very common alternative to
pronominal \glspl{RR} (\ref{ex:key:01.9}a) in \ili{Greek} and in other languages:

\ea \ili{Greek}\label{ex:key:01.9}
	\ea
		\gll ðjaleksa aftus [ {tus opius} protines ].\\
        chose.\Fsg{} those.\M.\Pl.\Acc{} {} which recommended.\Ssg{}\\
		\glt \enquote*{I chose those ones who you recommended.}
	\ex
		\gll ðjaleksa aftus [ pu protines ].\\
        chose.\Fsg{} those.\M.\Pl.\Acc{} {} that recommended.\Ssg{}\\
		\glt \enquote*{I chose those ones that you recommended.}
	\z
\z

\paragraph*{(iii) Animacy:} Furthermore, only \gls{FR} pronouns\is{relative pronouns!free relative pronouns} appear to
license an \isi{animacy} restriction.

Thus, \gls{FR} pronouns\is{relative pronouns!free relative pronouns} marked for masculine/feminine gender licence by default
a [+animate] interpretation, whereas \gls{FR} pronouns\is{relative pronouns!free relative pronouns}  marked for neuter
gender license a [−animate] interpretation. For example, the masculine \gls{FR}
pronoun \emph{opjus} in (\ref{ex:key:01.10}a), under its more natural interpretation, refers to
a male animate entity, whereas the neuter \gls{FR} \emph{opja} in (\ref{ex:key:01.10}b), evokes
a [−animate] entity.

\ea \ili{Greek}\label{ex:key:01.10}
	\ea
		\gll ðjaleksa opjus protines.\\
			chose.\Fsg{} who.\M{}.\Pl{} recommended.\Ssg{}\\
		\glt \enquote*{I chose who you recommended.}
	\ex
		\gll ðjaleksa opja protines.\\
			chose.\Fsg{} what.\glossN.\Pl{} recommended.\Ssg{}\\
		\glt \enquote*{I chose what you recommended}
	\z
\z

A similar point is made by the minimal pair in~\eqref{ex:key:01.11}: whereas
the neuter \gls{FR} pronoun \emph{opjo} is perfectly grammatical as the subject
of verbs that typically take thematic/inanimate subjects (\ref{ex:key:01.11}a),
it sounds awkward, when it occupies the subject position of verbs that
typically require agentive/animate subjects (\ref{ex:key:01.11}b).

\ea \ili{Greek}\label{ex:key:01.11}
    \ea[]{
		\gll opjo espase\\
			what.\glossN.\Sg{} broke.\Tsg{}\\
        \glt \enquote*{What(ever) broke.}}
    \ex[\#\#]{
		\gll opjo eγrapse tin epistoli\\
			what.\glossN.\Sg{} wrote.\Tsg{} the letter.\Acc{}\\
        \glt \#\#\enquote*{What(ever) wrote the letter.}}
	\z
\z

The distribution of \gls{RR} pronouns, on the other hand, does not appear to be
regulated by \isi{animacy} considerations. To illustrate, \gls{RR} pronouns\is{relative pronouns!restrictive relative pronouns} are
admissible with both animate and inanimate antecedents, independently of
whether they are marked for masculine~\eqref{ex:key:01.12} or neuter
gender~\eqref{ex:key:01.13}.

\ea \ili{Greek}\label{ex:key:01.12}
	\ea
		\gll ðjaleksa tus maθites {tus opius} protines.\\
			chose.\Fsg{} the students.\Acc{} which.\M{}.\Pl{} recommended.\Ssg{}\\
		\glt ‘I chose the students who you recommended.'
	\ex
		\gll ðjaleksa tus pinakes {tus opius} protines.\\
			chose.\Fsg{} the paintings.\Acc{} which.\M{}.\Pl{} recommended.\Ssg{}\\
		\glt \enquote*{I chose the paintings which you recommended.}
	\z
\z

\ea \ili{Greek}\label{ex:key:01.13}
	\ea
		\gll ðjaleksa ta peðja {ta opia} protines.\\
			chose.\Fsg{} the kids.\Acc{} which.\glossN.\Pl{} recommended.\Ssg{}\\
		\glt ‘I chose the kids who you recommended.'
	\ex
		\gll ðjaleksa ta pexniðja {ta opia} protines.\\
			chose.\Fsg{} the toys.\Acc{} which.\glossN.\Pl{} recommended.\Ssg{}\\
		\glt ‘I chose the toys which you recommended.'
	\z
\z

\paragraph*{(iv) Referentiality:} A further difference between \gls{FR} and
\gls{RR} pronouns concerns their ability to introduce new referents.
Consider in this regard the examples in~\eqref{ex:key:01.14} illustrating
coordination of \glspl{FR}:\newpage

\ea \ili{Greek}\label{ex:key:01.14}
	\ea
		\gll kalesa opjon simbaθi i Maria ke opjon adipaθi i Lina.\\
			 invited.\Fsg{} who.\Acc{} like.\Tsg{} the Maria.\Nom{} and who.\Acc{} dislike.\Tsg{} the Lina.\Nom{}\\
		\glt \enquote*{I invited whoever Maria likes and whoever Lina dislikes.}\\
            {}[\ding{51} Maria likes X \& Lina dislikes Y; \ding{51} Maria likes X \& Lina dislikes X]
	\ex
		\gll kalesa opjon simbaθi i Maria ke adipaθi i Lina.\\
			 invited.\Fsg{} who.\Acc{} like.\Tsg{} the Maria.\Nom{} and dislike.\Tsg{} the Lina.\Nom{}\\
		\glt \enquote*{I invited whoever Maria likes and Lina dislikes.}\\
        {}[*Maria likes X \& Lina dislikes Y; \ding{51} Maria likes X \& Lina dislikes X]
	\z
\z

When coordination takes place at the \gls{FR} pronoun level, the coordinated
phrases may either refer to two distinct discourse referents or to a single
participant (\ref{ex:key:01.14}a). Of the two possible readings, the first one
is the preferred one. However, when coordination takes place below the \gls{FR}
pronoun, the coordinated phrases may only refer to a single participant
(\ref{ex:key:01.14}b). In other words, there appears to be a correlation
between the number of \gls{FR} pronouns and the number of
referents.\footnote{In this respect the \gls{FR} pronoun \emph{opjos} behaves
    like the definite determiner \emph{o} ‘the’ in argumental DPs.
    \textcite[67--68]{AleHaeSta2007}, replicating a point originally made by
    \citet{Longobardi1994} for Italian, show that there appears to be a
    correlation between the number of definite determiners in coordinated DPs
    and the number of referents. Thus, whereas there is only one referent in
    (i), there are two referents in (ii):

\begin{exe}
    \exi{(i)} Greek\\
	\gll irθ-e/*-an o antiprosopos tis {dikastikis arxis} ke proedros tis eforeftikis epitropis.\\
        came-\Tsg/\Pl{} the delegate of.the court and chair of.the elective committee\\
	\glt \enquote*{The representative of the court and chair of the elective committee have arrived.}
\end{exe}

\begin{exe}
    \exi{(ii)} Greek\\
    \gll irθ-an/*-e o antiprosopos tis {dikastikis arxis} ke o proedros tis eforeftikis epitropis.\\
        came-\Tpl/\Sg{} the delegate of.the court and the chair of.the elective committee\\
	\glt \enquote*{The representative of the court and the chair of the elective committee has arrived.}
\end{exe}
}

The correlation between number of pronouns and number of referents is not
replicated by \glspl{RR}:

\ea \ili{Greek}\label{ex:key:01.15}
	\ea
		\gll kalesa afton {ton opio} simbaθi i Maria ke {ton opio} adipaθi i Lina.\\
			invited.\Fsg{} this.one.\Acc{} which.\Acc{} like.\Tsg{} the Maria.\Nom{} and  which.\Acc{} dislike.\Tsg{} the Lina.\Nom{}\\
		\glt \enquote*{I invited this one who Maria likes and who Lina dislikes.}\\
        {}[*Maria likes X \& Lina dislikes Y; \ding{51} Maria likes X \& Lina dislikes X]
	\ex
		\gll kalesa afton {ton opio} simbaθi i Maria ke {ton opio} adipaθi i Lina.\\
			invited.\Fsg{} this.one.\Acc{} which.\Acc{} like.\Tsg{} the Maria.\Nom{} and  which.\Acc{} dislike.\Tsg{} the Lina.\Nom{}\\
		\glt \enquote*{I invited this one who Maria likes and Lina dislikes.}\\
        {}[*Maria likes X \& Lina dislikes Y; \ding{51} Maria likes X \& Lina dislikes X]
	\z
\z

What the above examples serve to show is that multiple occurrences of an
\gls{RR} pronoun do not produce a multiple index interpretation.

\paragraph*{(v) Overt NP complement:} Finally, only \gls{FR} pronouns\is{relative pronouns!free relative pronouns} may
licence overt NP complements. This is shown by the contrast in grammaticality
between~\eqref{ex:key:01.16} and~\eqref{ex:key:01.17}:\footnote{It is only in
    appositive relatives that \emph{o opios} may take an overt NP complement:

\begin{exe}
    \exi{(i)} Greek\\
	\gll to computer, {to opio} computer epemenes na aγoraso, ðen ðulevi.\\
		the computer, which computer insisted.\Ssg{} \Sbjv{} buy.\Fsg{} \Neg{} work.\Tsg{}\\
	\glt \enquote*{The computer, which you insisted that I buy, is not working.}
\end{exe}
}

\ea \ili{Greek}\label{ex:key:01.16}\\
    \gll ðjaleksa opjus \llap{(}ipopsifius) protines.\\
		chose.\Fsg{} who.\Acc{} candidates recommended.\Ssg{}\\
	\glt \enquote*{I chose whichever candidates you recommended.}
\z\newpage

\ea \ili{Greek}\label{ex:key:01.17}
    \ea[*]{
        \gll ðjaleksa {tus opius} ipopsifius protines.\\
			chose.\Fsg{} which.\Acc{} candidates recommended.\Ssg{}\\
        \glt *\enquote*{I chose which candidates you recommended.}}
    \ex[*]{
        \gll ðjaleksa tus ipopsifius {tus opius} ipopsifius protines.\\
			chose.\Fsg{} the candidates which.\Acc{} candidates recommended.\Ssg{}\\
		\glt *\enquote*{I chose the candidates which candidates you recommended.}}
	\z
\z

Crucially, \gls{FR} pronouns\is{relative pronouns!free relative pronouns} with overt NP complements (complex \gls{FR}
pronouns, henceforth) differ from the simple \gls{FR} pronouns\is{relative pronouns!free relative pronouns} discussed so
far, in two respects: First, they cannot bear contrastive stress. In instances
of contrastive focus\is{focus!contrastive focus} it is their complement that is
focused~\eqref{ex:key:01.18}:

\ea \ili{Greek}\label{ex:key:01.18}
    \ea[*]{
        \gll kalese mono OPJUS maθites tu protines oxi opjes maθitries tu protines.\\
			 invited.\Tsg{} only which.\M{}.\Pl{}  students.\M{}.\Pl{}
             \Cl.\Tsg.\M{} recommended.\Ssg{} not which.\glossF.\Pl{}
             students.\glossF.\Pl{} \Cl.\Tsg.\M{} recommended.\Ssg{}\\
		\glt intended: \enquote*{He only invited WHICHEVER male students you
        recommended to him, not whichever female students you recommended to
him.}}
    \ex[]{
		\gll kalese mono opjus MAθITES tu protines oxi opjes maθitries tu protines.\\
			 invited.\Tsg{} only which students.\M{}.\Pl{} \Cl.\Tsg.\M{}
             recommended.\Ssg{} not which students.\glossF.\Pl{} \Cl.\Tsg.\M{}
             recommended.\Ssg{}\\
		\glt \enquote*{He only invited whichever MALE students you recommended
    to him, not whichever female students you recommended to him.}}
	\z
\z

Second, they may take both animate and inanimate complements, independently of
whether they are marked for masculine/feminine gender, as in~\eqref{ex:key:01.19}, or for
neuter gender, as in~\eqref{ex:key:01.20}:

\ea \ili{Greek}\label{ex:key:01.19}
	\ea
		\gll ðjaleksa opjus maθites protines.\\
			chose.\Fsg{} which.\M{}.\Pl{} students.\Acc{} recommended.\Ssg{}\\
		\glt ‘I chose whichever students you recommended.'
	\ex
		\gll ðjaleksa opjus pinakes protines.\\
			chose.\Fsg{} which.\M{}.\Pl{} paintings.\Acc{} recommended.\Ssg{}\\
		\glt ‘I chose whichever paintings you recommended.'
	\z
\z

\ea \ili{Greek}\label{ex:key:01.20}
	\ea
		\gll ðjaleksa opja peðja protines.\\
			chose.\Fsg{} which.\glossN.\Pl{} kids.\Acc{} recommended.\Ssg{}\\
		\glt ‘I chose whichever kids you recommended.'
	\ex
		\gll ðjaleksa opja pexniðja protines.\\
			chose.\Fsg{} which.\glossN.\Pl{} toys.\Acc{} recommended.\Ssg{}\\
		\glt ‘I chose whichever toys you recommended.'
	\z
\z

A schematic summary of the differences between Restrictive and Free Relative
pronouns (simple and complex) is provided in \tabref{tab:key:3}:

\begin{table}[htpb]
\caption{The properties of \gls{RR} and \gls{FR} pronouns.}\label{tab:key:3}
{\small
\begin{tabularx}{\textwidth}{XXXX}
\lsptoprule
                                             & \gls{FR} pronouns\newline (simple) & \gls{FR} pronouns\newline (complex) & \gls{RR} pronouns\\
\midrule
Contrastive Focus                            & Yes                                & No (only their\newline complement)  & No \\
Null counterparts                            & No                                 & No                                  & Yes \\
Animacy                                      & Yes                                & No                                  & No \\
Disjoint reference\newline under conjunction & Yes                                & Yes                                 & No \\
Overt NP\newline complement                  & No                                 & Yes                                 & No \\
\lspbottomrule
\end{tabularx}
}
\end{table}

The list of differences between Free and Restrictive Relative pronouns can be
narrowed down into two main points of divergence:

First, \gls{FR} pronouns\is{relative pronouns!free relative pronouns} (simple/complex), unlike \gls{RR} pronouns, are
referential. This explains the correlation between the number of \gls{FR}
pronouns and the number of referents~\eqref{ex:key:01.14}, a correlation that does not hold in
the case of \gls{RR} pronouns~\eqref{ex:key:01.15}.

Second, \gls{FR} pronouns\is{relative pronouns!free relative pronouns}
(simple/complex), unlike \gls{RR} pronouns, may license their own range. The
range may take the form of an \isi{animacy} restriction licensed by the
\gls{FR} pronoun \eqref{ex:key:01.10} and \eqref{ex:key:01.11} (in the case of
simple \gls{FR} pronouns), or the form of a lexical NP complementing the
\gls{FR} pronoun \eqref{ex:key:01.19} and~\eqref{ex:key:01.20} (in the case of
complex \gls{FR} pronouns). This explains why simple \gls{FR} pronouns can be
contrastively focused. Being inherently specified as [+animate] or
[−animate],\is{animacy} they can bear contrastive focus\is{focus!contrastive
    focus} with respect to animacy~\eqref{ex:key:01.6}.  \gls{RR} pronouns, on
    the other hand, not being specified for \isi{animacy} cannot bear
    contrastive focus\is{focus!contrastive focus} for a property they
    lack~\eqref{ex:key:01.7}.

Under this view, \gls{FR} pronouns\is{relative pronouns!free relative pronouns} lack null counterparts because their
deletion would result in unrecoverable loss of both referentiality and range
\eqref{ex:key:01.8}.

\section{Towards an analysis}\label{sec:key:01.4}

Having established at an empirical level that \gls{RR} pronouns\is{relative
pronouns!restrictive relative pronouns} are deficient compared to \gls{FR}
pronouns, I will now consider the question of theoretical implementation.
After showing that both types of pronouns are transitive Determiners
(\Cref{sub:key:01.4.1,sub:key:01.4.2}), I will suggest that their differences
lie in their featural composition: whereas both \gls{RR} and \gls{FR}
Determiners are morphologically definite, only the latter ones are semantically
definite (\Cref{sub:key:01.4.3}).

\subsection{Both free and restrictive relative pronouns are
DPs}\label{sub:key:01.4.1}

It is possible that the referential deficiency of \gls{RR} pronouns\is{relative
pronouns!restrictive relative pronouns} is reflective of some sort of
structural deficiency. Thus, adopting and adapting \citegen{DecWil2002} account
of personal pronouns, we could assume that whereas \gls{FR} pronouns are Ds
projecting a DP, \gls{RR} pronouns\is{relative pronouns!restrictive relative
pronouns} are the mere spell out of phi features\is{φ-features} (phi Ps).
Within this approach, \gls{RR} pronouns\is{relative pronouns!restrictive
relative pronouns} fail to refer because they lack an external D layer, which
is typically taken to be the locus of definiteness/referentiality.

There are two main issues with this approach. First, as mentioned in the
introduction, both Free and Restrictive Relative pronouns incorporate a
morphologically definite determiner (\emph{o} ‘the’). Thus, morphological
considerations suggest that they are both Ds. The second issue is syntactic in
nature and concerns their distribution. Even though both pronouns surface in
[Spec,CP], they can be theta related to all the major argument positions,
including the subject of (in)transitive verbs, the subject of primary and
secondary predication, the (in)direct object, and the prepositional object
position. The latter is illustrated in~\eqref{ex:key:01.21}
and~\eqref{ex:key:01.22} with a \gls{FR} and \gls{RR} pronoun, respectively:

\ea \ili{Greek}\label{ex:key:01.21}\\
    \gll jia opjus \llap{(}maθites / pinakes) mu milises\\
        about which students {} paintings \Cl.\Fsg.\Gen{} talked.\Ssg{}\\
	\glt \enquote*{About whichever (students/paintings) you talked to me.}
\z\newpage

\ea \ili{Greek}\label{ex:key:01.22}\\
	\gll o maθitis / pinakas jia {ton opio} mu milises\\
        the student.\Nom{} {} painting.\Nom{} about who.\Pl.\Acc{} \Cl.\Fsg.\Gen{} talked.\Ssg{}\\
	\glt \enquote*{The student/painting about whom you talked to me.}
\z

On the assumption that argumenthood is a property of DPs
\citep{Longobardi1994}, it follows that both \emph{opjos}-phrases and
\emph{o opios} phrases are associated with a DP projection.

\subsection{Both free and restrictive relative pronouns are transitive Ds}\label{sub:key:01.4.2}

Furthermore, it can be argued that in addition to showing the external
distribution of DPs, both types of pronouns show the internal syntax of
determiners. Complex \gls{FR} pronouns\is{relative pronouns!free relative pronouns} clearly behave like transitive
determiners, since they allow an NP complement. The latter can be overt, as in
\eqref{ex:key:01.23} repeated from~\eqref{ex:key:01.16} above, or elided under
identity with a discourse antecedent, as
in~\eqref{ex:key:01.24}.\footnote{Evidence suggesting that the \gls{FR} pronoun
    in (\ref{ex:key:01.24}b) is a transitive determiner with a deleted NP
    restrictor comes from its similarities with other instances of nominal
    subdeletion attested in \ili{Greek}, such as the one illustrated in (i):

\begin{exe}
    \exi{(i)} Greek
    \begin{xlist}
        \ex
		\gll pja fusta aγorases?\\
			which.\glossF.\Sg{} skirt bought.\Ssg{}\\
		\glt \enquote*{Which skirt did you buy?}
        \ex
		\gll tin kokini.\\
			the.\glossF.\Sg{} red.\glossF.\Sg{}\\
    \end{xlist}
\end{exe}

In this regard, see \citet{Daskalaki2009} who shows how the conditions on
nominal subdeletion identified by  \textcite{GiaSta1999} can be
replicated for \gls{FR} phrases.}

\ea \ili{Greek}\label{ex:key:01.23}\\
	\gll kalesa [ opjus ipopsifius protines ].\\
    invited.\Fsg{} {} who.\Acc{} candidates recommended.\Ssg{} {}\\
	\glt \enquote*{I invited whichever candidates you recommended.}
\z

\ea \ili{Greek}\label{ex:key:01.24}
	\ea
		\gll pjus ipopsifius kaleses?\\
			which candidates invited.\Ssg{}\\
		\glt \enquote*{Which candidates did you invite?}
	\ex
		\gll opjus mu protines.\\
			who \Cl.\Fsg.\Gen{} recommended.\Ssg{}\\
		\glt \enquote*{Whoever you recommended to me.}
	\z
\z

In the absence of a salient discourse antecedent, we saw that \gls{FR} pronouns
(simple \gls{FR} pronouns\is{relative pronouns!free relative pronouns} in our
terms) receive a [$\pm$animate] interpretation, depending on their gender
specification (\ref{ex:key:01.10}--\ref{ex:key:01.11}). One way to implement
this observation is to assume that they bear interpretable phi features that
are responsible for licensing a null complement. Thus, an interpretable
masculine/feminine gender licenses an empty [+animate] NP complement, whereas
an interpretable neuter gender licenses a [−animate] NP complement. Within this
account, the difference between complex and simple \gls{FR}
pronouns\is{relative pronouns!free relative pronouns} does not lie in their
(in)transitivity. Rather it depends on whether the \gls{FR} determiner has
entered the derivation with an uninterpretable set of phi features (that will
be valued by an overt lexical NP) or with an interpretable set of phi features
that is responsible for licensing a null, [$\pm$animate] NP
complement.\footnote{Alternatively, it could be the case that the phi features
    of the \gls{FR} determiner are always uninterpretable. In the case of
    complex \gls{FR} pronouns\is{relative pronouns!free relative pronouns} they
    get valued through agreement with an overt lexical NP, whereas in the case
    of simple \gls{FR} pronouns\is{relative pronouns!free relative pronouns}
    they get valued through agreement with the gender specification of a null
    NP meaning ‘man’, ‘woman’, or ‘thing’. An analysis along these lines would
    be compatible with \textcite{Panagiotidis2003} and would allow us to treat
    homogeneously complex and (apparently) simple \gls{FR} pronouns. However,
    it is not clear how it would derive the contrast between the two types of
    \gls{FR} pronouns\is{relative pronouns!free relative pronouns} with respect
    to contrastive focus\is{focus!contrastive focus}. In other words, if both
    simple and complex \gls{FR} pronouns bear uninterpretable gender it is not
    clear why only the former ones can bear contrastive
    focus\is{focus!contrastive focus} (compare \eqref{ex:key:01.6} with
\eqref{ex:key:01.18}).}

Let us, finally, consider the \gls{RR} pronoun \emph{o opios}. At a first
approximation its treatment as a transitive Determiner seems implausible, given
that, at least in its restrictive use, it never surfaces with an overt NP
complement~\eqref{ex:key:01.17}. However, this would be incompatible with both
the Raising Analysis\is{relative clauses!raising analysis} (\citealt{Kayne1994}; for \ili{Greek} \glspl{RR}, see
\citealt{AleAna2000}, among others) and the Matching Analysis
(\citealt{Sauerland1998}; for \ili{Greek} \glspl{RR}, see \citealt{KotzVar2005}
of \isi{relative clauses}). Motivated by independent considerations, such as
reconstruction effects, both analyses maintain the claim that the \gls{RR}
pronoun is a Determiner taking an NP complement. In the case of the Raising\is{relative clauses!raising analysis}
Analysis, the NP complement is raised to the antecedent position, whereas in
the case of the Matching Analysis it is deleted under identity with an
externally Merged antecedent.\footnote{Thanks to an anonymous reviewer for
    pointing this out to me.} In view of these independent considerations, I
    will be assuming that \gls{RR} pronouns, like \gls{FR} pronouns, are
    transitive Determiners.\footnote{Within this analysis,
        (\ref{ex:key:01.17}a) is ungrammatical not because there is no NP
        position projected in syntax, but because the \gls{RR} determiner,
        being expletive\is{expletives} (see \Cref{sub:key:01.4.3}) cannot introduce a clause
        that functions as an argument. Accordingly, (\ref{ex:key:01.17}b) is
    ungrammatical because due to some economy consideration the complement of
the \gls{RR} determiner needs to be deleted under identity with a c-commanding
antecedent.}

\subsection{RR pronouns, unlike FR pronouns, have an expletive\is{expletives} D head}\label{sub:key:01.4.3}

If both \gls{FR} and \gls{RR} pronouns\is{relative pronouns!restrictive relative pronouns} are transitive Ds, then the referential
deficiency of \gls{RR} pronouns\is{relative pronouns!restrictive relative pronouns} cannot be treated as an instance of structural
deficiency. A conceivable alternative would be to treat it as an instance of
featural deficiency. Under this view, the difference between \gls{FR} and
\gls{RR} pronouns depends on whether their D head is semantically
definite/referential, as in the case of \gls{FR} pronouns, or semantically
inert, as in the case of \gls{RR} pronouns.

That the definite morphology of \gls{RR} pronouns\is{relative pronouns!restrictive relative pronouns} is void of any semantic
contribution is not a novel claim (see, among others,
\citealt[80]{Bianchi1999}; for \ili{Greek}, see \citealt{Alexiadou1998}). Independent
evidence in support of this analysis comes from the expletive\is{expletives} uses of the \ili{Greek}
definite determiner in contexts other than \glspl{RR}.  Consider, for example, the
phenomenon of polydefiniteness, illustrated in~\eqref{ex:key:01.25}:

\ea \ili{Greek}\label{ex:key:01.25}\\
	\gll to    spiti   to   megalo\\
		the house the big\\
    \trans \enquote*{the big house}
\z

In~\eqref{ex:key:01.25}, a noun (\emph{spiti} ‘house’) is modified by an adjective
(\emph{megalo} ‘big’), and noun and adjective are each introduced by a
morphologically definite determiner (\emph{to} ‘the’).  Despite the multiple
occurrences of the definite article, the construction does not receive a
multiple reference interpretation. Thus,~\eqref{ex:key:01.25} refers to a single entity at the
intersection of the set of houses and the set of big entities
\parencite{LekSze2012}. This fact has been taken to show that the definite
determiner in \ili{Greek}, at least in some contexts, can be used as an expletive\is{expletives}
(for an overview of the proposed analyses, see \citealt{Alexiadou2014b}). It is
this claim that we reiterate here for the \gls{RR} determiner.

Our second claim, that the \gls{FR} pronoun encodes
definiteness/referentiality, has been more controversial in the literature.
Recall from \Cref{sub:key:01.2.1} that \glspl{FR} can be paraphrased with
definite DPs. One group of analyses derives the referentiality/definiteness of
\glspl{FR} from the referentiality/definiteness of \gls{FR} pronouns\is{relative pronouns!free relative pronouns} (see, for
instance, \citealt{Jacobson1995} and \citealt{Pancheva2000}, among others). A
different group of analyses suggests that the reason why \glspl{FR} are
interpreted like definite DPs is because of a null c-commanding
Determiner/element that turns them into referential expressions
(\citealt{GrovanRie1981,Caponigro2003,GroLan1998}, among others).

One of the main semantic arguments in favor of the null D analysis is that many
languages use the same range of relative pronouns both in definite \glspl{FR}
and in irrealis \glspl{FR} \citep{Caponigro2003}. Irrealis \glspl{FR} differ
from definite \glspl{FR} in a number of ways (\citealt{Caponigro2003};
\citealt{Pancheva2000}; \citealt{GroLan1998}): Irrealis \glspl{FR} always
complement existential predicates (mainly the existential \emph{have} or
\emph{be}), they include irrealis verbal morphology, and, crucially, they
cannot be paraphrased by definite DPs.  Rather they appear to be semantically
equivalent with weak NPs. As an illustrative example, we may consider the
\ili{Polish} examples below, illustrating a standard and an irrealis \gls{FR},
respectively:

\ea \ili{Polish} \citep[27]{Caponigro2003}\label{ex:key:01.26}\\%26
	\gll Posmakowalam [ co ugotowałes ].\\
            tasted.\Fsg{} {} what cooked.\Ssg{}\\
	\glt    \enquote*{I tasted what you cooked.}
\z

\ea \ili{Polish} \citep[88]{Caponigro2003}\label{ex:key:01.27}\\%27
    \gll (Nie) mam [ co robić ].\\
            \hphantom{(}not have.\Fsg{} {} what do.\Inf{}\\
	\glt    \enquote*{There \{is something, isn’t anything\} I can do.}
\z

As pointed out by \citet{Caponigro2003} the fact that the same range of
pronouns is used both in standard/definite~\eqref{ex:key:01.26} and in irrealis
\glspl{FR}~\eqref{ex:key:01.27}, is problematic for the claim that these
pronouns are inherently definite.  Significantly, though, this counterargument
does not apply to the \ili{Greek} data.  As illustrated below, \gls{FR}
pronouns fail to introduce irrealis \glspl{FR} (\ref{ex:key:01.28}a).  Rather
an interrogative pronoun is used  for this purpose (\ref{ex:key:01.28}b):

\ea \ili{Greek}\label{ex:key:01.28}
    \ea[*]{
    \gll ðen exo se opjon na miliso.\\
    \Neg{} have.\Fsg{} to who \Sbjv{} talk.\Fsg{}\\}
    \ex[]{
		\gll ðen exo se pjon na miliso.\\
			\Neg{} have.\Fsg{} to who \Sbjv{} talk.\Fsg{}\\
        \glt intended: \enquote*{I don’t have anyone to talk to.}}
	\z
\z

If \emph{opjos} is not semantically definite, it is not clear what rules out
its use in (\ref{ex:key:01.28}a).

An additional challenge for the extension of the null D analysis to \ili{Greek} is
posed by the fact that the presumed null definite D fails to be replaced by the
overt definite Determiner \emph{o} ‘the’ that independently exists in the
language~\eqref{ex:key:01.29}:

\ea \ili{Greek}\label{ex:key:01.29}\\
    \gll \llap{*}Kalese ton opjon θes.\\
    invite.\Ssg{} the who want.\Ssg{}\\
    \trans *\enquote*{Invite the whoever you want.}
\z

Of course, it could be the case that the morphologically definite determiner is
always expletive\is{expletives} and that definiteness is always provided by a null
c-commanding functional head.\footnote{ This has actually been proposed by
\textcite{LekSze2012} on the basis of polydefinites.} Even in this case
though, one would expect that \emph{o opios} would be able to introduce a
\gls{FR} (when embedded under the null definite D) and that \emph{opjos}
would be able to introduce an \gls{RR} (when not embedded) under the null D).
As shown below, neither of the two predictions is borne out:

\ea \ili{Greek}\label{ex:key:01.30}\\
    \gll    \llap{*}Kalese [ ${\varnothing}$ [ {ton opio} maθiti θes ]].\\
            invite.\Ssg{} {} {} {} which student want.\Ssg{}\\
    \trans intended: \enquote*{Invite which student you want.}
\z

\ea \ili{Greek}\label{ex:key:01.31}\\
    \gll    \llap{*}Kalese afton / ton maθiti [ opjon θes ].\\
            invite.\Ssg{} this.one {} the student {} who want.\Ssg{}\\
    \trans  *\enquote*{Invite him/the student whoever you want.}
\z

In view of the above facts, I conclude that, at least in \ili{Greek}, the
\gls{FR} determiner, unlike the \gls{RR} determiner, is semantically
definite/referential.\footnote{ If this conclusion is on the right track, then
    it seems that the semantic import of \gls{FR} pronouns\is{relative pronouns!free relative pronouns} could be subject to
    cross-linguistic variation. On the one hand, there are \gls{FR} pronouns
    like the \ili{Greek} \emph{opjos} that may take an NP complement and encode
    definiteness. On the other hand, there are \gls{FR} pronouns\is{relative pronouns!free relative pronouns} like the
    \ili{Polish} \emph{co} or the \ili{English} \emph{who} that may not take an NP
    complement, and, according to \citeauthor{Caponigro2003}’s convincing
    analysis (\citeyear{Caponigro2003}),
encode \isi{animacy} (they are mere set restrictors).} Thus, whereas the \gls{RR}
    determiner \emph{o opios} is [−def, +rel], the \gls{FR} determiner
    \emph{opjos} is [+def, + rel]. Because it is semantically definite, it
    needs a range that is provided by its NP complement. The latter can be an
    overt NP, a deleted NP, or an empty NP that receives a [$\pm$animacy]
    interpretation.\is{animacy}

\section{Conclusions}\label{sec:key:01.5}

In this paper, I explored the possibility that relative pronouns, like personal
pronouns, show different degrees of strength/deficiency. I showed that, at
least in \ili{Greek}, the \gls{RR} pronoun \emph{o opios} is semantically deficient
compared to its \gls{FR} counterpart \emph{opjos} in two interrelated respects:
(i) it is referentially deficient and (ii) it does not license its own range.
After showing that both \gls{FR} and \gls{RR} pronouns\is{relative pronouns!restrictive relative pronouns} behave like transitive
Ds, I proposed that their differences lie in their featural composition:
\gls{FR} Determiners, unlike \gls{RR} Determiners, are semantically definite. This
analysis suggests that, at least in some cases, referential deficiency can be
indicative of featural rather than structural deficiency
\parencite[cf.][]{CarSta1999,DecWil2002}.  Furthermore, it opens up the
possibility of attributing the distribution of Free and Restrictive Relative
clauses to the properties of their introductory determiners. \gls{FR}
determiners, being [+Def], turn a clause into a referential DP. \gls{RR}
determiners, on the other hand, being expletive\is{expletives}, turn a clause into a predicate
that can function as a nominal modifier. The implications of these conclusions
for existing analyses of Free and Restrictive Relatives can be the topic of
future research.

\printchapterglossary{}

\section*{Acknowledgements}

In this paper, I revisit a puzzle that I briefly discussed in my PhD
dissertation (University of Cambridge, 2009). I would like to thank Prof. Ian
Roberts, my PhD supervisor, for all his help and support during those years.
Moreover, I would like to thank two anonymous reviewers for helpful comments
and suggestions. All remaining errors are, of course, my own.

{\sloppy
\printbibliography[heading=subbibliography,notkeyword=this]
}

\end{document}
